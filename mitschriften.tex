\documentclass[a4paper,11pt,oneside,final,notitlepage,onecolumn]{article}
\usepackage[top=2.5cm, bottom=3cm, left=2cm, right=2cm]{geometry}

\usepackage{ucs}
\usepackage[utf8x]{inputenc}
\usepackage{amsmath}
\usepackage{amsfonts}
\usepackage{amssymb}
\usepackage{mathtools}
\usepackage{mathrsfs}
\usepackage{bbm}
\usepackage{ulem}
\usepackage{caption}
\usepackage{multicol}
\usepackage[table]{xcolor}
\usepackage[ngerman]{babel}
\usepackage[T1]{fontenc}
\usepackage[pdftex]{graphicx}
\usepackage{tabularx}
\usepackage{eurosym}
\usepackage{pifont}
\newcommand{\cmark}{\ding{51}}
\usepackage{ziffer}
\usepackage{enumerate}
\usepackage{stmaryrd}
\usepackage{soul}
\usepackage{xifthen}
\usepackage{imakeidx}
\makeindex

\DeclareFontFamily{U}{futm}{}
\DeclareFontShape{U}{futm}{m}{n}{
  <-> s * [.92] fourier-bb
  }{}
\DeclareSymbolFont{Ufutm}{U}{futm}{m}{n}
\DeclareSymbolFontAlphabet{\mathbb}{Ufutm}

\usepackage{tikz}
\usetikzlibrary{positioning}
\usetikzlibrary{arrows}
\usetikzlibrary{shapes.misc}
\usetikzlibrary{angles}
\usetikzlibrary{quotes}
\usetikzlibrary{babel}
\usetikzlibrary{3d}
\usetikzlibrary{decorations.pathreplacing}
\tikzset{cross/.style={cross out, draw=black, minimum size=2*(#1-\pgflinewidth), inner sep=0pt, outer sep=0pt},cross/.default={1pt}}

\usepackage{titlesec}
\usepackage{lipsum}

\usepackage{natbib}

\usepackage{textcomp}
\usepackage{pgfplots}
\pgfplotsset{compat=1.12}

\bibliographystyle{plainnat}

\newcommand{\termin}[1]{{\newpage\Large \textbf{#1:}}}
\newcommand{\myparagraph}[1]{\paragraph{#1}\mbox{}\\}

\titleformat{\section}{\LARGE\bfseries}{\textbf{Kapitel \thesection:}}{1em}{}
\titleformat{\subsection}{\normalfont\bfseries}{\textbf{\Roman{subsection})}}{1em}{}
\titleformat{\subsubsection}{\normalfont}{\textbf{\alph{subsubsection})}}{1em}{}

\usepackage{xspace}
\newcommand{\titledot}[0]{\texorpdfstring{$\cdot$}{;}\xspace}

\newcommand{\leadingzero}[1]{\ifnum #1<10 0\the#1\else\the#1\fi}
\newcommand{\todaynew}{\leadingzero{\day}.\leadingzero{\month}.\the\year}

\newcounter{MathCounter}
\setcounter{MathCounter}{0}
\newcommand{\mdf}[1]{\vspace{0.2cm}\textbf{#1 \stepcounter{MathCounter}\arabic{MathCounter}:}}

\newcommand{\kapitel}[1]{\setcounter{MathCounter}{0}
\section{#1}}

\newcommand{\norm}[1]{\left\lVert#1\right\rVert}

\newcommand{\begr}[2][]{\ifthenelse{\isempty{#1}}{\index{#2}}{\index{#1}}\textbf{#2}}

% \setcounter{secnumdepth}{0}

\setlength{\parindent}{0cm}

% \rowcolors{2}{gray!15}{white}

\usepackage{url}
\def\UrlBreaks{\do\/\do-}
\usepackage[pdftex,breaklinks]{hyperref}
\hypersetup{
  pdftitle    = {MG2 Mitschrift},
  pdfsubject  = {Mitschrift},
  pdfauthor   = {},
  pdfkeywords = {},
  pdfcreator  = {pdfLaTeX},
  pdfproducer = {LaTeX},
  colorlinks = false
}

\renewcommand{\maketitle}[3]{
  \begin{center}
    \title{#1}
    \Large
    \textbf{#1}
    
    \normalsize
    \textit{#2}

    \date{\today}
    \textit{#3}
  \end{center}
  \normalsize
  
  \hypersetup{
    pdftitle    = {#1},
    pdfsubject  = {#1},
    pdfauthor   = {},
    pdfkeywords = {},
    pdfcreator  = {pdfLaTeX},
    pdfproducer = {LaTeX},
  }
}

\begin{document}
\maketitle{Mathematische Grundlagen 2 -- Mitschriften}{Mathematische Grundlagen 2: Lineare Algebra und Differential- und Integralrechnung}{Universität Bremen \titledot Sommersemester 2016}
\section*{Anmerkungen}
\textbf{Diese Mitschriften können die persönliche Anwesenheit in der Vorlesung nicht ersetzen!} Bei diesem Dokument handelt es sich hauptsächlich um mehr oder weniger exakte Abschriften von der Tafel, teilweise mit persönlichen Anmerkungen und Notizen. Eventuelle Fehler in diesem Dokument sind nicht ganz auszuschließen. Ich bemühe mich jedoch, ein möglichst gutes Skript zu erstellen. Fehler können mir gerne gemeldet werden, zum Beispiel persönlich an mich oder im zugehörigen GitHub Repo als Issue (\url{https://github.com/ccoors/Mathe2-Mitschriften}).

Am Ende dieses Dokuments finden sich viele definierte wichtige Begriffe im \textbf{Index}. Die Seitenzahlen sind verlinkt.

\tableofcontents

\termin{05.04.2016}
\section{Zahlen}
\subsection{Zahlenmengen}
\begin{itemize}
\item{Die Menge $\mathbb{N} = \{0,1,2,3,...\}$ heißt \underline{Menge der natürlichen Zahlen}. Für uns beinhalten die natürlichen Zahlen die $0$.}
\item{Die Menge $\mathbb{Z} = \{...,-3,-2,-1,0,1,2,3,...\}$ heißt \underline{Menge der ganzen Zahlen}.}
\item{Die Menge $\mathbb{Q} = \left\{\frac{p}{q}\,|\,p,q \in \mathbb{Z}, q \neq 0\right\}$ heißt \underline{Menge der rationalen Zahlen}.}
\end{itemize}

\subsection{Satz von Euklid}
Es gibt keine rationale Zahl $q \in \mathbb{Q}$ mit $q^2 = 2$.

\subsection{Reelle Zahlen}
Die Menge $\mathbb{R}$ der reellen Zahlen ist die Vereinigung der Menge der rationalen Zahlen mit allen Zahlen, die sich durch rationalen Zahlen beliebig approximieren lassen.

\subsection{Wurzel}
Ist $b^n = a$ für $a, b \in \mathbb{R} > 0, n \in \mathbb{N}$, so heißt b die \underline{$n$-te Wurzel} von $a$: $b = \sqrt[n]{a}=a^\frac{1}{n}$. $b$ ist für alle $a > 0$ eindeutig. Der Vorgang des Wurzelziehens heißt auch \underline{radizieren}.

\subsection{Kurzschreibweisen}
Wir führen Kurzschreibweisen ein:
\begin{align*}
[a,b] &:= \{x\in\mathbb{R}\,|\,a \leq x \leq b\}\text{ heißt abgeschlossenes Intervall,} \\
[a,b) &:= \{x\in\mathbb{R}\,|\,a \leq x < b\}\text{ und }(a,b] := \{x\in\mathbb{R}\,|\,a < x \leq b\}\text{ heißen halboffene Intervalle und} \\
(a,b) &:= \{x\in\mathbb{R}\,|\,a < x < b\}\text{ heißt offenes Intervall.} \\
&\text{Sonderfälle:} \\
[a, \infty) &:= \{x\in\mathbb{R}\,|\,a \leq x\} \\
(-\infty, b] &:= \{x\in\mathbb{R}\,|\,x \leq b\} \\
(a, \infty) &:= \{x\in\mathbb{R}\,|\,a < x\} \\
(-\infty, b) &:= \{x\in\mathbb{R}\,|\,x < b\}
\end{align*}

\subsection{Beschränktheit nach oben}
Eine Menge $M \subseteq \mathbb{R}$ heißt nach oben beschränkt, falls es ein $k \in \mathbb{R}$ gibt, so dass $x \le k$ für alle $x \in M$. $k$ heißt obere Schranke von $M$. Ist $M$ nach oben beschränkt, gibt es mehrere obere Schranken von $M$: $M = [3,5)$. Offenbar ist $k = 17$ eine obere Schranke von $M$. Die kleinste obere Schranke von $M$ heißt \underline{Supremum} von $M$, kurz, $\text{sup}(M)$.

\subsection{Satz über die Vollständigkeit von $\mathbb{R}$}
Jede nach oben beschränkte Menge $M \subseteq \mathbb{R}$ besitzt ein Supremum in $\mathbb{R}$. Für $\mathbb{Q}$ gilt dies nicht. Das Supremum der Menge $[0,\sqrt{3}) \subseteq \mathbb{Q}$ wäre in $\mathbb{R}$ $\sqrt{3}$, diese Zahl liegt aber nicht in $\mathbb{Q}$. Dies unterscheidet die reellen Zahlen fundamental von den rationalen Zahlen.

\subsection{Beschränktheit nach unten}
Siehe Beschränktheit nach oben. Wichtig: Die größte untere Schranke heißt \underline{Infimum}, kurz $\text{inf}(M)$. Es gilt: $\text{inf}(M) = -\text{sup}(-M)$. ($-M = \{-m\,|\,m \in M\}$).

\subsection{Satz zu Supremum und Infimum} Supremum und Infimum einer Menge M müssen nicht in der Menge liegen. Ist $\text{sup}(M) \in M$, so ist es auch das größte Element (Maximum) von $M$, kurz: $\text{max}(M)$. Analog dazu: Ist $\text{inf}(M) \in M$, so ist es auch das kleinste Element (Minimum) von $M$, kurz: $\text{min}(M)$.

\subsection{Literatur}
Alles bis hierher ist zu finden in:

\href{http://www.mat.univie.ac.at/~gerald/ftp/book-mfi/mfi1.pdf}{Mathematik für Informatiker -- Band 1: Diskrete Mathematik und Lineare Algebra. Springer. Gerald Teschl, Susanne Teschl. Kapitel 2.1.}



\termin{08.04.2016}
\subsection{Betrag}
Der Betrag einer reellen Zahl x ist gegeben durch
\begin{align*}
|x| &= \left\{\begin{array}{cl} x, & \mbox{falls }x \geq 0\\ -x, & \mbox{falls } x < 0 \mbox{/sonst} \end{array}\right.
\end{align*}

\subsection{Körper}
Ein Körper $K$ heißt angeordnet, falls es auf $K$ eine totale Ordnung $\leq$ gibt, so dass gelten:
\begin{description}
\item[(A1)]{Für alle $x, y, z \in K : x \leq y \Rightarrow x \leq y + z$}
\item[(A2)]{Für alle $x, y, z \in K : x \leq y, z > 0 \Rightarrow x\cdot z \leq y\cdot z$}
\end{description}
(Monotonie der Addition und Multiplikation)

\subsection{Vollständig angeordneter Körper}
Die reellen Zahlen bilden den bis auf Isomrphie eindeutig bestimmten \underline{vollständig angeordneten Körper}.

Endliche Körper können nicht angeordnet werden, z.B. $Z_7$.

\begin{align*}
&\overline{0} \leq \overline{1} \leq \overline{2} \leq \overline{3} \leq \overline{4} \leq \overline{5} \leq \overline{6} \\
&\overline{5} \leq \overline{6}\text{, aber }\overline{5} + \overline{1} = \overline{6} \neq \overline{6} + \overline{1} = \overline{0}
\end{align*}

\subsection{Zwischenfazit Zahlenmengen}
\begin{align*}
	\mathbb{N} \subseteq \mathbb{Z} \subseteq \mathbb{Q} \subseteq \mathbb{R}
\end{align*}

\subsection{Reelle Zahlen auf der Zahlengeraden}
Man kann sich die reellen Zahlen als Zahlengerade vorstellen. Diese Zahlengerade hat keine Lücken mehr.

\begin{center}
\begin{tikzpicture}[>=triangle 45,font=\sffamily]
    \draw [<->] (0,0) -- (4,0);
    \draw (2 cm, 4pt) -- (2 cm, -4pt) node[anchor=north] {$0$};
    \draw (2.5 cm, 4pt) -- (2.5 cm, -4pt) node[anchor=north] {$\frac{1}{2}$};
    \draw (3 cm, 4pt) -- (3 cm, -4pt) node[anchor=north] {$1$};
\end{tikzpicture}
\end{center}

Aber: Es gibt Gleichungen, die wir nicht lösen können.

$x^2 + 1 = 0$ hat keine Lösung in den reellen Zahlen, da es kein $x \in \mathbb{R}$ gibt, so dass $x^2 = -1$.

\subsection{Komplexe Zahlen}
Die Menge $\mathbb{C} = \{x + i\cdot y\,|\,x, y \in \mathbb{R}\}$ heißt \underline{Menge der komplexen Zahlen}.

Die Zahl $i = 0 + 1\cdot i$ ist definiert als $i^2 = -1$ und heißt \underline{imaginäre Einheit}. Für eine Zahl $z = x + i\cdot y$ heißt $x$ \underline{Realteil von z} ($\text{Re}(z) := x$), $y$ heißt \underline{Imaginärteil von z} ($\text{Im}(z) := y$).

\subsubsection{Beispiel}
$z = 3 - 2\cdot i$, dann $\text{Re}(z) = 3, \text{Im}(z) = -2$

\subsection{Komplexe Zahlen bilden einen Körper}
Die komplexen Zahlen bilden einen Körper.

\begin{itemize}
\item{\textbf{Addition}: Seien $z_1 = x_1 + i \cdot y_1$ und $z_2 = x_2 + i \cdot y_2$, dann ist $z_1 + z_2 = (x_1 + i \cdot y_1) + (x_2 + i \cdot y_2) = (x_1 + x_2) + i \cdot (y1 + y2)$.}
\item{\textbf{Multiplikation}: $z_1 \cdot z_2 = (x_1 + i \cdot y_1) \cdot (x_2 + i \cdot y_2) = x_1 \cdot x_2 + x_1 \cdot i \cdot y_2 + i \cdot y_1 \cdot x_2 + i^2 \cdot y_1 \cdot y_2 = x_1 \cdot x_2 - y_1 \cdot y_2 + i \cdot y_1 \cdot x_2 + i \cdot y_2 \cdot x_1 = (x_1 \cdot x_2 - y_1 \cdot y_2) + i \cdot (x_1 \cdot y_2 + x_2 \cdot y_1)$}
\end{itemize}

\subsubsection{Beweis}
\begin{enumerate}
\item{Assiziativität der Addition: \cmark}
\item{Neutralelement der Addition: $0 + 0 \cdot i = 0$ \cmark}
\item{Inverses bezüglich Addition: $z = x + i \cdot y \Rightarrow -z = -x - i \cdot y$,

Denn: $(x + i \cdot y) + (-x - i \cdot y) = (x - x) + i (y - y) = 0 + i \cdot 0 = 0$ \cmark}
\item{Kommutativität der Addition: \cmark}
\end{enumerate}

\myparagraph{Assoziativgesetz der Multiplikation}
Siehe P2 auf Blatt 2

\myparagraph{Neutralelement der Multiplikation}
$1 + 0 \cdot i = 1$

\myparagraph{Inverse bezüglich Multiplikation}
Ist $z = x + i \cdot y$, dann ist
\begin{align*}
	z^{-1} = \frac{x}{x^2 + y^2} + i \cdot \frac{-y}{x^2 + y^2}
\end{align*}
denn:
\begin{align*}
(x + i \cdot y) \cdot \left(\frac{x}{x^2 + y^2} + i \cdot \frac{-y}{x^2 + y^2}\right) = \left(\frac{x^2 + y^2}{x^2 + y^2}\right) + i \cdot \frac{x \cdot y - y \cdot x}{x^2 + y^2} = 1 + i \cdot 0
\end{align*}

\myparagraph{Distributivgesetz}
Siehe H4 auf Blatt 2

\subsection{Komplex konjugierte Zahlen}
Für eine komplexe Zahl $z = x + i \cdot y$ heißt $\overline{z} = x - i \cdot y$ die zu $z$ komplex konjugierte Zahl.

Damit $\text{Re}(z) = \frac{z+\overline{z}}{2}, \text{Im}(z)=\frac{z-\overline{z}}{2\cdot i}$

Es gilt: $\overline{z_1 + z_2} = \overline{z_1 + z_2}, \overline{z_1 \cdot z_2} = \overline{z_1 \cdot z_2}$
\begin{align*}
\overline{z^{-1}} = \overline{z}^{-1}
\end{align*}

\subsection{Der Betrag von komplexen Zahlen}
Der Betrag einer komplexen Zahl $z = x+iy$ ist gegeben durch $|z| = \sqrt{z \cdot \overline{z}} = \sqrt{x^2 + y^2}$.

Wir können die komplexen Zahlen geometrisch durch die Gaußsche Zahlenebene veranschaulichen.

\begin{center}
\begin{tikzpicture}[>=triangle 45,font=\sffamily]
% 	See https://de.sharelatex.com/blog/2013/08/27/tikz-series-pt1.html
%     \draw[step=1cm,gray,very thin] (-1.9,-1.9) grid (5.9,5.9);
    \draw[thick,->] (-1,0) -- (4.5,0);
	\draw[thick,->] (0,-1) -- (0,4.5);
	\draw[thick,->] (0,0) -- (4.5,0) node[anchor=north west] {Realteil};
	\draw[thick,->] (0,0) -- (0,4.5) node[anchor=south east] {Imaginärteil};
	\foreach \x in {1,2,3,4}
		\draw (\x cm,4pt) -- (\x cm,-4pt) node[anchor=north] {$\x$};
	\foreach \y in {1,2,3,4}
		\draw (4pt,\y cm) -- (-4pt,\y cm) node[anchor=east] {$\y$};
	
	\draw[thick,->] (0,0) -- (2,3);
	\draw (2,3) node [cross=5pt,red] {};
	\node [align=left] at (3,3) {$(x+iy)$};
\end{tikzpicture}
\end{center}

\subsection{Dreiecksungleichung}
Für alle $x, w \in \mathbb{C}$ gilt:
$|z + w| \leq |z| + |w|$ (\underline{Dreiecksungleichung})

Anschaulich:

\begin{center}
\begin{tikzpicture}[>=triangle 45,font=\sffamily]
    \draw[thick,->] (-1,0) -- (4.5,0);
	\draw[thick,->] (0,-1) -- (0,4.5);
	\draw[thick,->] (0,0) -- (4.5,0) node[anchor=north west] {Realteil};
	\draw[thick,->] (0,0) -- (0,4.5) node[anchor=south east] {Imaginärteil};
	\foreach \x in {1,2,3,4}
		\draw (\x cm,4pt) -- (\x cm,-4pt) node[anchor=north] {$\x$};
	\foreach \y in {1,2,3,4}
		\draw (4pt,\y cm) -- (-4pt,\y cm) node[anchor=east] {$\y$};
	
	\draw[thick,->] (0,0) -- (2,1);
	\draw[thick,->] (2,1) -- (3,3);
	\draw[thick,->] (0,0) -- (3,3);
	\draw (2,1) node [cross=5pt,red] {};
	\draw (3,3) node [cross=5pt,red] {};
	\node [align=left] at (2.4,1) {$z$};
	\node [align=left] at (3.4,3) {$w$};
\end{tikzpicture}
\end{center}

\subsection{Quadrieren von komplexen Zahlen}
Zu jeder Zahl $z \in \mathbb{C}$ gibt es ein $w \in \mathbb{C}$ mit $z = w^2$.

\subsubsection{Beweis}
Sei $z = x + i \cdot y$:
\begin{enumerate}
\item{$z = 0$; dann $z = 0^2$, also $w = 0$}
\item{$z \neq 0$ und $x > 0$. Dann setze $u := \sqrt{(1/2) \cdot (x + \sqrt{x^2 + y^2})}$. Dann ist $u \in \mathbb{R}$.

Mit $v := \frac{y}{2 \cdot u}$ gilt: $(u + v \cdot i^2) = w^2 = x + i \cdot y = z$}
\item{$z \neq 0$ und $x \leq 0$. Dann setze $u := \sqrt{\frac{1}{2} \cdot (-x + \sqrt{x^2 + y^2})}$ Dann ist $v \in \mathbb{R}$.

Mit $u := \frac{y}{2 \cdot v}$ gilt: $(u + v \cdot i^2) = w^2 = x + iy = z$}

\end{enumerate}

\subsection{$\mathbb{C}$ anordnen}
$\mathbb{C}$ lässt sich nicht anordnen.


\termin{12.04.2016}
\kapitel{Geometrie Teil 1}
\begin{center}
\begin{tikzpicture}[>=triangle 45,font=\sffamily]
\draw[step=1cm,gray,very thin] (0.1,0.1) grid (3.9,3.9);
\draw[thick,->] (0,0) -- (4.5,0) node[anchor=north west] {$x$};
\draw[thick,->] (0,0) -- (0,4.5) node[anchor=south east] {$y$};
\foreach \x in {1,2,3,4}
        \draw (\x cm,4pt) -- (\x cm,-4pt) node[anchor=north] {$\x$};
\foreach \y in {1,2,3,4}
        \draw (4pt,\y cm) -- (-4pt,\y cm) node[anchor=east] {$\y$};

\draw (3,2) node [cross=5pt,red] {};
\node [align=left] at (3.4,2) {$P$};
\end{tikzpicture}
\end{center}

Punkte in der Ebene können wie folgt beschrieben werden:
\begin{enumerate}
	\item{Als geordnetes Paar: $P = (3, 2)$.}
	\item{Wir wählen \glqq{}grundlegende Basiselemente\grqq{} z.B. $a$ = \glqq{}gehe\grqq{} 1 Schritt in $x$-Richtung und $b$ = \glqq{}gehe\grqq{} 1 Schritt in $y$-Richtung. Dann können wir $P$ schreiben als $3a + 2b$.}
	\item{Als Vektor: $P = \begin{pmatrix}3 \\ 2\end{pmatrix}$, wie ein geordnetes Paar untereinander.}
\end{enumerate}

Offenbar hängt die Darstellung von $P$ von der Achsenbeschriftung ab, bzw. von den \glqq{}Basiselementen\grqq{}.

Z.B. Setze $a$ = \glqq{}gehe\grqq{} 3 Schritte in $x$-Richtung, $b$ wie eben, dann ist $P = a + 2b$. Oder $a$ = \glqq{}gehe\grqq{} 1 Schritt entgegen der $x$-Richtung, dann $P = -3a + 2b$.

\subsection{Die reelle Ebene}
\mdf{Definition}
Die \underline{reelle Ebene} ist das kartesische Produkt $\mathbb{R}^2 = \mathbb{R} \times \mathbb{R}$ und enthält alle Punkte $(x, y)$ mit $x, y \in \mathbb{R}$, also $\mathbb{R}^2 = \{(x,y)\,|\,x,y \in \mathbb{R}\}$. Statt $(x, y)$ schreiben wir in Zukunft $\begin{pmatrix}x \\ y\end{pmatrix} \in \mathbb{R}^2$ für Punkte in der reellen Ebene.

$\begin{pmatrix}x \\ y\end{pmatrix}$ heißt (2-dimensionaler) \underline{Vektor} oder \underline{Spaltenvektor}.

Analog können wir den dreidimensionalen Raum beschreiben als
\begin{align*}
	\mathbb{R}^3 = \mathbb{R} \times \mathbb{R} \times \mathbb{R} = \{(x,y,z)\,|\,x,y,z \in \mathbb{R}\}
\end{align*}

Offenbar ist $\mathbb{R}^3 = \mathbb{R}^2 \times \mathbb{R}$.

Analog definiert man den $\mathbb{R}^n$, den $n$-dimensionalen reellen Raum:

\begin{align*}
	\mathbb{R}^n = \left\{\begin{pmatrix}x_1 \\ x_2 \\ x_3 \\ \vdots \\ x_n \end{pmatrix}\,\Bigg|\, x_1, x_2, \dots, x_n \in \mathbb{R} \right\}
\end{align*}

\subsection{Rechnen mit Vektoren}
\mdf{Definition}
\begin{align*}
	\text{Für 2 Vektoren }x = \begin{pmatrix}x_1 \\ \vdots \\ x_n \end{pmatrix},\enspace y = \begin{pmatrix}y_1 \\ \vdots \\ y_n \end{pmatrix}\text{ ist } x + y\text{ definiert als }\begin{pmatrix}x_1 \\ \vdots \\ x_n \end{pmatrix} + \begin{pmatrix}y_1 \\ \vdots \\ y_n \end{pmatrix} = \begin{pmatrix}x_1 + y_1 \\ \vdots \\ x_n + y_n \end{pmatrix}
\end{align*}

\mdf{Beispiel}
\begin{align*}
	\begin{pmatrix} -1 \\ 0 \\ 2 \\ 1,5 \end{pmatrix} + \begin{pmatrix} 1 \\ \pi \\ 0 \\ -3 \end{pmatrix} = \begin{pmatrix} 0 \\ \pi \\ 2 \\ -1,5 \end{pmatrix}
\end{align*}
Beachte: Man kann nur gleichdimensionale Vektoren addieren.

\subsection{Die Skalarmultiplikation}
\mdf{Definition}
Die \underline{Skalarmultiplikation} eines Vektors $x \in \mathbb{R}^n$ mit einer festen Zahl (Skalar genannt) $\lambda \in \mathbb{R}$ ist gegeben durch
\begin{align*}
	\lambda \cdot x = \begin{pmatrix} \lambda \cdot x_1 \\ \vdots \\ \lambda \cdot x_n \end{pmatrix}
\end{align*}

\mdf{Beispiel}
\begin{align*}
	\lambda = 2, \enspace x = \begin{pmatrix}2 \\ 1\end{pmatrix} \quad \lambda \cdot x = \begin{pmatrix}4 \\ 2\end{pmatrix}
\end{align*}

\subsection{Rechnen mit Vektoren}
Mit Vektoren kann man fast wie mit Zahlen rechnen.

Für $x,y,z \in \mathbb{R}^n$ gilt
\begin{itemize}
	\item{$(x+y)+z = x+(y+z)$ Assoziativgesetz}
	\item{$x+y = y+x$ Kommutativgesetz}
	\item{Der Nullvektor $0 = \begin{pmatrix}0 \\ \vdots \\ 0\end{pmatrix} \in \mathbb{R}^n$ ist neutrales Element der Addition}
	\item{Zu jedem $x \in \mathbb{R}^n$ gibt es ein \glqq{}negatives\grqq{} $-x \in \mathbb{R}^n$ mit $x + (-x) = 0 \in \mathbb{R}^n$}
	\item{Für $\lambda,\mu \in \mathbb{R}$ gilt $(\lambda + \mu)\cdot x = \lambda \cdot x + \mu \cdot x$ Distributivgesetz}
	\item{Für $\lambda,\mu \in \mathbb{R}$ gilt $(\lambda \cdot \mu) \cdot x = \lambda \cdot (\mu \cdot x)$ Assoziativgesetz der Skalarmultiplikation}
\end{itemize}

\mdf{Bemerkung}
Für Vektoren gibt es kein \glqq{}sinnvolles\grqq{} Produkt, so dass $x\cdot y$ in $\mathbb{R}^n$ liegt und die Eigenschaften gelten, die man von dem Produkt erwartet.

\subsection{Lineare Unabhängigkeit}
\mdf{Definition}
Eine Menge von Vektoren $v_1,\dots,v_k \in \mathbb{R}^n$ heißt \underline{linear abhängig} falls die Gleichung

\begin{align*}
	\lambda _1 \cdot v_1 +\dots+\lambda _k \cdot v_k = 0 (\lambda _1,\dots,\lambda _k \in \mathbb{R})
\end{align*}

nur die Lösung $\lambda _1,\dots,\lambda _k = 0$ hat.

Nicht linear unabhängige Vektoren heißen linear abhängig.

Für Vektoren $v_1,\dots,v_k \in \mathbb{R}^n$ und Skalare $\lambda _1,\dots\lambda _k \in \mathbb{R}$ heißt $\lambda _1 \cdot v_1+\dots+\lambda _k \cdot v_k$ \underline{Linearkombination}.

\subsection{Unterraum}
\mdf{Definition}
Eine Teilmenge $U \subseteq \mathbb{R}^n$ heißt \underline{Unterraum} von $\mathbb{R}^n$ falls gilt:
\begin{description}
	\item[(U1)]{$0 \in U$}
	\item[(U2)]{$\forall u \in U,\, \forall \lambda \in \mathbb{R}\,:\,\lambda \cdot u \in U$}
	\item[(U3)]{$\forall u,v \in U\,:\,u+v \in U$}
\end{description}

\mdf{Beispiel}
Betrachte $\mathbb{R}^3$
\begin{align*}
	U = \left\{\begin{pmatrix}x \\ y \\ 0\end{pmatrix}\,\Bigg|\,x,y \in \mathbb{R}\right\}\text{ ist Unterraum des }\mathbb{R}^3
\end{align*}

Betrachte $\mathbb{R}^2$
\begin{align*}
	U = \left\{\begin{pmatrix}x \\ -x\end{pmatrix}\,\Bigg|\,x \in \mathbb{R}\right\}\,\subseteq\,\mathbb{R}^2\text{ ist Unterraum des }\mathbb{R}^2
\end{align*}

\begin{center}
\begin{tikzpicture}[>=triangle 45,font=\sffamily]
\draw[step=1cm,gray,very thin] (-3.9,-3.9) grid (3.9,3.9);
\draw[thick,->] (-4.5,0) -- (4.5,0) node[anchor=north west] {$x$};
\draw[thick,->] (0,-4.5) -- (0,4.5) node[anchor=south east] {$y$};
\foreach \x in {-4,-3,-2,-1,1,2,3,4}
        \draw (\x cm,4pt) -- (\x cm,-4pt) node[anchor=north] {$\x$};
\foreach \y in {-4,-3,-2,-1,1,2,3,4}
        \draw (4pt,\y cm) -- (-4pt,\y cm) node[anchor=east] {$\y$};

\draw[thick,red] (-3.5,3.5) -- (3.5,-3.5);
\end{tikzpicture}
\end{center}

Betrachte $\mathbb{R}^n$, $U = \left\{\begin{pmatrix}0 \\ \vdots \\ 0\end{pmatrix}\right\}$ ist Unterraum des $\mathbb{R}^n$.

\subsection{Lineare Hülle/Span}
\mdf{Definition}
Seien $v_1,\dots,v_k \in \mathbb{R}^n$. Die Menge
\begin{align*}
	\text{span}(v_1,\dots,v_k) = \{\lambda _1 \cdot v_1+\dots+\lambda _k \cdot v_k\,|\,\lambda _i \in \mathbb{R}\}
\end{align*}

heißt \underline{lineare Hülle} oder \underline{Span} von $v_1,\dots,v_k$.

Die Vektoren $v_1,\dots,v_k$ heißen Erzeugendensystem eines Unterraums $U$ falls $U = \text{span}(v_1,\dots,v_k)$.

Die Vektoren $v_1,\dots,v_k$ heißen Basis von $U$ falls $U = \text{span}(v_1,\dots,v_k)$ und $v_1,\dots,v_k$ sind linear unabhängig.


\termin{15.04.2016}

\mdf{Beispiel}
Die Standardbasis des $\mathbb{R}^n$ ist die Menge der Vektoren
\begin{align*}
	v_1 = \begin{pmatrix}1\\0\\\vdots\\0\end{pmatrix}, v_2 = \begin{pmatrix}0\\1\\0\\\vdots\\0\end{pmatrix}, \dots, v_n = \begin{pmatrix}0\\\vdots\\0\\1\end{pmatrix}
\end{align*}

$v_1,\dots,v_n$ sind linear unabhängig (trivial).

$v_1,\dots,v_n$ bilden ein Erzeugendensystem von $\mathbb{R}^n$, denn jeder Vektor $x = \begin{pmatrix}x_1\\\vdots\\x_n\end{pmatrix}$ lässt sich schreiben als
\begin{align*}
	x = \begin{pmatrix}x_1\\\vdots\\x_n\end{pmatrix} = x_1\begin{pmatrix}1\\0\\\vdots\\0\end{pmatrix} + x_2\begin{pmatrix}0\\1\\0\\\vdots\\0\end{pmatrix} +\dots+x_n\begin{pmatrix}0\\\vdots\\0\\1\end{pmatrix}
\end{align*}

Konkret: Standardbasis von $\mathbb{R}^2\,:\,v_1=\begin{pmatrix}1\\0\end{pmatrix},v_2=\begin{pmatrix}0\\1\end{pmatrix}$.

Eine andere Basis des $\mathbb{R}^2$ ist $v_1=\begin{pmatrix}2\\0\end{pmatrix},v_2=\begin{pmatrix}0\\1\end{pmatrix}$. Hier lässt sich jeder Vektor $x = \begin{pmatrix}x_1\\x_2\end{pmatrix}$ schreiben als $x = \begin{pmatrix}x_1\\x_2\end{pmatrix} = \frac{x_1}{2}\cdot\begin{pmatrix}2\\0\end{pmatrix} + x_2 \cdot\begin{pmatrix}0\\1\end{pmatrix}$.

\mdf{Beispiel}
Es seien $v_1 = \begin{pmatrix}3\\5\\2\end{pmatrix}, v_2 = \begin{pmatrix}4\\0\\1\end{pmatrix}, v_3 = \begin{pmatrix}2\\2\\1\end{pmatrix}$.
\begin{align*}
	\text{span}(v_2, v_3) = \left\{\lambda _1 \begin{pmatrix}4\\0\\1\end{pmatrix} + \lambda _2 \begin{pmatrix}2\\2\\1\end{pmatrix} \,\Bigg|\, \lambda _1, \lambda _2 \in \mathbb{R} \right\}
\end{align*}

Z.B. ist $\begin{pmatrix}4\\2\\1,5\end{pmatrix} \in \text{span}(v_2, v_3)$, da $\begin{pmatrix}4\\2\\1,5\end{pmatrix} = \frac{1}{2} \begin{pmatrix}4\\0\\1\end{pmatrix} + 1 \begin{pmatrix}2\\2\\1\end{pmatrix}$.

\vspace{0.3cm}

Frage: Liegt $v_1$ in $\text{span}(v_2,v_3)$?

Antwort: Nur dann, wenn es $\lambda,\mu \in \mathbb{R}$ gibt mit $v_1 = \lambda v_2 + \mu v_3$, also
\begin{alignat*}{8}
\text{I}\quad & 4 & \lambda & \enspace+\enspace & 2 & \mu & \enspace=\enspace & 3 \\
\text{II}\quad &  &         &   & 2 & \mu & = & 5 \\
\text{III}\quad & & \lambda & \enspace+\enspace &   & \mu & \enspace=\enspace & 2
\end{alignat*}

Das ist ein lineares Gleichungssystem (mehr dazu in Kapitel 3). Aus II folgt $\mu = \frac{5}{2}$, in III eingesetzt ergibt sich $\lambda = -\frac{1}{2}$. Einsetzen in I bestätigt unsere Lösung, also liegt $v_1 \in \text{span}(v_2,v_3)$.

\vspace{0.5cm}

Es sei folgende Basis von $\mathbb{R}^3$ gegeben:
\begin{align*}
	v_1 = \begin{pmatrix}1\\1\\1\end{pmatrix}, v_2 = \begin{pmatrix}1\\2\\3\end{pmatrix}, v_3 = \begin{pmatrix}2\\-1\\1\end{pmatrix}
\end{align*}

Wir wollen nun $x = \begin{pmatrix}1\\-2\\5\end{pmatrix}$ durch $v_1,v_2,v_3$ ausdrücken, also $\lambda _1, \lambda _2, \lambda _3 \in \mathbb{R}$ finden, so dass $\lambda _1 v_1 + \lambda _2 v_2 + \lambda _3 v_3 = x$.

\begin{alignat*}{8}
\text{I}\quad & \lambda _1 & \enspace+\enspace & \lambda _2 & \enspace+\enspace & 2 \lambda _3 & \enspace=\enspace & 1 \\
\text{II}\quad & \lambda _1 & \enspace+\enspace & 2 \lambda _2 & \enspace-\enspace & \lambda _3 & \enspace=\enspace & -2 \\
\text{III}\quad & \lambda _1 & \enspace+\enspace & 3 \lambda _2 & \enspace+\enspace & \lambda _3 & \enspace=\enspace & 5
\end{alignat*}

Es ergibt sich $\lambda _1 = -6, \lambda _2 = 3, \lambda _3 = 2$.
\begin{align*}
	x = \begin{pmatrix}1\\-2\\5\end{pmatrix} = (-6)\begin{pmatrix}1\\1\\1\end{pmatrix} + 3 \begin{pmatrix}1\\2\\3\end{pmatrix} + 2 \begin{pmatrix}2\\-1\\1\end{pmatrix}
\end{align*}

\vspace{0.5cm}

2 Vektoren sind genau dann linear unabhängig wenn einer der Vektoren ein Vielfaches des anderen Vektors ist.


\termin{19.04.2016}

\subsection{Rechtwinkliges Dreieck}
\mdf{Definition}
In einem rechtwinkligen Dreieck
\begin{center}
\begin{tikzpicture}[font=\sffamily,scale=1.5]
\coordinate (A) at (0,0);
\coordinate (B) at (4,0);
\coordinate (C) at (4,2);

\pic["$\alpha$", draw=red, thick, -, angle eccentricity=0.7, angle radius=2cm] {angle=B--A--C};
\pic["$\cdot$", draw=red, thick, -, angle eccentricity=0.5, angle radius=0.5cm] {angle=C--B--A};

\draw [-] (A) -- (B) -- (C) -- (A);
\node [right] at (4,1) {$a$};
\node [above] at (2,1) {$b$};
\node [below] at (2,0) {$c$};
\end{tikzpicture}
\end{center}

heißt $b$ \underline{Hypotenuse}, $c$ \underline{Ankathete} zu $\alpha$ und $a$ \underline{Gegenkathete} zu $\alpha$.

Das Verhältnis $\frac{c}{b}$ heißt \underline{Kosinus} von $a$:
\begin{align*}
    \text{cos}(\alpha) &= \frac{\text{Länge der Ankathete}}{\text{Länge der Gegenkathete}}
\end{align*}

Offenbar macht diese Definition vom Kosinus nur Sinn für Winkel $0^{\circ} \leq \alpha \leq 90^{\circ}$.
\begin{center}
\begin{tikzpicture}[font=\sffamily]
\coordinate (A) at (0,0);
\coordinate (B) at (10,0);
\coordinate (C) at (0.1,1.2);
\coordinate (D) at (0.1,-1.2);

\draw [-] (A) -- (B);
\draw [-] (C) -- (D);

\node [left] at (0,0) {$0$};
\draw (0.2,1) -- (0,1) node[left] {$1$};
\draw (0.2,-1) -- (0,-1) node[left] {$-1$};

\draw [blue,thick] plot[variable=\x,domain=0:10,smooth,samples=200] (\x,{cos(\x r)});
\end{tikzpicture}
\end{center}

\subsection{Skalarprodukt}
\mdf{Definition}
Seien $x = \begin{pmatrix}x_1\\\vdots\\x_n\end{pmatrix},\quad y = \begin{pmatrix}y_1\\\vdots\\y_n\end{pmatrix}$ Vektoren aus $\mathbb{R}^n$.

Das \underline{Skalarprodukt} $\langle x, y\rangle$ von $x$ und $y$ ist definiert durch
\begin{align*}
    \langle x, y\rangle &= x_1y_1 + x_2y_2 + \dots + x_ny_n
\end{align*}

Zwei Vektoren $x, y \in \mathbb{R}^n$ heißen \underline{orthogonal} oder \underline{senkrecht} zueinander, falls $\langle x, y\rangle = 0$.

\subsection{Norm}
\mdf{Definition}
Die \underline{Norm} eines Vektors $x \in \mathbb{R}^n$ ist definiert durch $\norm{x} := \sqrt{\langle x, x\rangle}$.

\subsection{Winkel zwischen Vektoren}
\mdf{Definition}
Der Winkel $\varphi$ (Phi) zwischen den Vektoren $x, y \in \mathbb{R}^n$ mit $x\neq 0, y\neq 0$ ist festgelegt durch
\begin{align*}
    \text{cos}(\varphi) &= \frac{\langle x, y\rangle}{\norm{x}\cdot\norm{y}}
\end{align*}

\begin{center}
\begin{tikzpicture}[font=\sffamily]
\draw [->] (0,0) -- (-1,1);
\draw [->,thick,red] (0,0) -- (2,3);
\draw [->] (0,0) -- (1,1.5);

\node [below left] at (-0.5,0.5) {$x$};
\node [below right] at (0.5,0.75) {$y$};
\node [below right,red] at (1.5,2.5) {$y_1 = \lambda y$};
\end{tikzpicture}
\end{center}
Die Vektoren müssen normiert werden, weil der Winkel sich sonst durch verschieden lange Vektoren verändern würde.

\mdf{Beispiel}
Im $\mathbb{R}^2$: $v_1 = \begin{pmatrix}1\\0\end{pmatrix},\quad v_2 = \begin{pmatrix}x\\y\end{pmatrix}$ mit $x, y \geq 0$.

\begin{center}
\begin{tikzpicture}[>=triangle 45,font=\sffamily,scale=1.5,decoration=brace]
\coordinate (A) at (0,0);
\coordinate (B) at (1,0);
\coordinate (C) at (2.5,2.2);

\pic["$\varphi$", draw=red, thick, -, angle eccentricity=0.7, angle radius=1.2cm] {angle=B--A--C};
	
\draw[->] (-0.5,0) -- (3.5,0) node[anchor=north west] {$x$};
\draw[->] (0,-0.5) -- (0,3.5) node[anchor=south east] {$y$};
\foreach \x in {1,2,3}
	\draw (\x cm,4pt) -- (\x cm,-4pt) node[anchor=north] {$\x$};
\foreach \y in {1,2,3}
	\draw (4pt,\y cm) -- (-4pt,\y cm) node[anchor=east] {$\y$};

\draw [thick,->] (0,0) -- (B);
\node [below] at (0.5,0) {$v_1$};

\draw [thick,->] (0,0) -- (C);
\node [above left] at (1.25,1.1) {$v_2$};

\draw [decorate, yshift=0.3cm] (0,2) -- node[above] {$x$} (2.5,2);
\draw [dashed] (0,2.2) -- (2.5,2.2);

\draw [rotate=-90, decorate, yshift=0.3cm] (-2.2,2.3) -- node[right] {$y$} (0,2.3);
\draw [dashed] (2.5,0) -- (2.5,2.2);
\end{tikzpicture}
\end{center}

Nach Definition 13:
\begin{align*}
	\text{cos}(\varphi) &= \frac{x}{\sqrt{x^2 + y^2}}
\end{align*}

Mit Skalarprodukt nach Definition 16:
\begin{align*}
	\frac{\left\langle \begin{pmatrix}1\\0\end{pmatrix}, \begin{pmatrix}x\\y\end{pmatrix} \right\rangle}{\norm{\begin{pmatrix}1\\0\end{pmatrix}}\cdot \norm{\begin{pmatrix}x\\y\end{pmatrix}}} &= \frac{1x + 0y}{\sqrt{1^2 + 0^2}\cdot\sqrt{x^2 + y^2}} = \frac{x}{\sqrt{x^2 + y^2}} = \text{cos}(\varphi)
\end{align*}

\subsection{Orthogonale Systeme}
\mdf{Definition}
Eine Menge von Vektoren $v_1,\dots,v_k \in \mathbb{R}^n$ heißt \underline{orthogonales System}, falls für alle $1\leq i, j \leq k, i\neq j$ gilt: $\langle v_i, v_j\rangle = 0$.

Ein orthogonales System $v_1,\dots,v_n \in \mathbb{R}^n$, das zusätzlich Basis von $\mathbb{R}^n$ ist und für alle $1\leq i\leq n \norm{v_i} = 1$ erfüllt, heißt \underline{Orthonormalbasis} von $\mathbb{R}^n$.

\begin{center}
\begin{tikzpicture}[>=triangle 45,font=\sffamily]
% Ja, das hat lange gedauert. Aber dafür ist es nun sehr flexibel :)
\coordinate (v1) at ({2*cos(1.1 r)},{2*sin(1.1 r)});
\coordinate (v2) at ({2*cos((1.1-pi/2) r)},{2*sin((1.1-pi/2) r)});
\coordinate (c) at (0,0);

\pic["$\cdot$", draw=red, thick, -, angle eccentricity=0.6, angle radius=1.2cm] {angle=v2--c--v1};

\draw [step=2cm,gray,very thin] (-2.5,-2.5) grid (2.5,2.5);
\draw [->] (-3,0) -- (3,0) node[anchor=north west] {$x$};
\draw [->] (0,-2.5) -- (0,3) node[anchor=south east] {$y$};

\draw (0,0) circle (2);

\draw [->] (0,0) -- (v1);
\draw [->] (0,0) -- (v2);

\node [below] at (0,-2.5) {Einheitskreis};
\end{tikzpicture}
\end{center}
Das ist auch eine Basis.

\mdf{Satz}
\begin{enumerate}[i)]
  \item{Sei $v \in \mathbb{R}^n$. Dann gilt: $\norm{v} = 0 \Leftrightarrow v = 0$}
  \item{Für alle Vektoren $v, w \in \mathbb{R}^n$ gilt: $\langle v, w\rangle = \langle w, v\rangle$ (\glqq{}symmetrisch\grqq{})}
  \item{$\forall x, y, z \in \mathbb{R}^n$ gilt: $\langle x+y, z\rangle = \langle x, z\rangle + \langle y, z\rangle$}
  \item{$\forall x, y \in \mathbb{R}^n,\enspace \lambda, \mu \in \mathbb{R}$ gilt: $\langle\lambda x, \mu y\rangle = \lambda\mu\langle x, y\rangle$}
  \item{Ist $v_1,\dots,v_k$ ein orthogonales System und alle $v_i \neq 0$, dann sind $v_1,\dots,v_k$ linear unabhängig.}
\end{enumerate}

\textbf{Zu i)}
\begin{align*}
	\norm{v} = \sqrt{\langle v, v\rangle} = \sqrt{v_1^2+v_2^2+\dots +v_n^2} = 0 &\Leftrightarrow v_1^2+v_2^2+\dots +v_n^2 = 0 \\
	&\Leftrightarrow v_1 = v_2 = \dots = v_n = 0 \\
	&\Leftrightarrow v = 0
\end{align*}

\textbf{Zu ii)}
\begin{align*}
	\langle v, w\rangle &= v_1w_1+v_2w_2+\dots +v_nw_n\quad|\text{ Kommutativgesetz} \\
	&= w_1v_1+w_2v_2+\dots +w_nv_n = \langle w, v\rangle
\end{align*}

\textbf{Zu iii) und iv): Siehe Übung H8}

\textbf{Zu v)}
Zu zeigen: $\lambda_1v_1+\lambda_2v_2+\dots +\lambda_kv_k = 0 \Rightarrow \lambda_1 = \lambda_2 = \dots = \lambda_k = 0$

Sei also $\lambda_1v_1+\lambda_2v_2+\dots +\lambda_kv_k = 0$. Bilde das Skalarprodukt
\begin{align*}
	\langle 0, v_1\rangle &= 0 = \langle \lambda_1v_1+\dots +\lambda_kv_k, v_1\rangle \\
	\text{Nach iii)}\quad &= \langle \lambda_1v_1, v_1\rangle +\langle \lambda_2v_2, v_1\rangle+\dots +\langle \lambda_kv_k, v_1\rangle \\
	\text{Nach iv)}\quad &=\lambda_1\langle v_1, v_1\rangle + \lambda_2\langle v_2, v_1\rangle + \dots + \lambda_k\langle v_k, v_1\rangle = \lambda_1\langle v_1, v_1\rangle \\
	&= \lambda_1 \langle v_1, v_1\rangle = \lambda_1 \norm{v_1}^2\quad\text{Da } v_1\neq 0 \Rightarrow \lambda_1 = 0
\end{align*}

Analog für $i=2,\dots,k$.
\begin{align*}
	0 = \langle 0, v_i\rangle &= \langle \lambda_1v_1 + \dots + \lambda_kv_k, v_i\rangle \\
	&= \lambda_1 \langle v_1, v_i\rangle + \dots + \lambda_k \langle v_k, v_i\rangle \\
	&= \lambda_i \langle v_i, v_i\rangle\quad\text{(Da $v_1,\dots,v_k$ orth. System)} \\
	&= \lambda_i \norm{v_i}^2\quad\text{Da }v_i\neq 0\text{, folgt }\lambda_i = 0\text{.}
\end{align*}
Also $\lambda_1 = \lambda_2 = \dots = \lambda_k = 0$. Damit sind $v_1,\dots,v_k$ unabhängig.


\termin{22.04.2016}
\kapitel{Lineare Gleichungssysteme}
\mdf{Definition}
Ein \begr{lineares Gleichungssystem} aus $m$ Gleichungen und $n$ Unbekannten $x_1,\dots,x_n$ hat die Form
\begin{align*}
    a_{11}x_1 + a_{12}x_2 + \dots + a_{1n}x_n &= b_1 \\
    a_{21}x_1 + a_{22}x_2 + \dots + a_{2n}x_n &= b_2 \\
    \vdots & \\
    a_{m1}x_1 + a_{m2}x_2 + \dots + a_{mn}x_n &= b_m
\end{align*}

Dabei sind die $a_{ij}$ und $b_i$ reelle oder komplexe Zahlen. Die $a_{ij}$ heißen \begr{Koeffizienten} des LGS. Sind alle $b_i$ gleich Null, so heißt das LGS \begr{homogen}, sonst \begr{inhomogen}.

Ein LGS zu lösen bedeutet, Zahlen $x_1,\dots,x_n$ zu finden, so dass alle Gleichungen des LGS erfüllt werden.

\mdf{Satz}
Ein inhomogenes LGS hat entweder keine, genau eine oder unendlich viele Lösungen.

\mdf{Beispiel}

\textbf{a)}
\begin{alignat*}{4}
\text{I}\quad & x+y & = 2 & \quad|\,\text{Aus II: } x = y \\
\text{II}\quad & x-y & = 0 & \quad|\,\text{Aus I: } 2x = 2
\end{alignat*}
Also genau eine Lösung: $x = y = 1$. \\

\textbf{b)}
\begin{alignat*}{4}
\text{I}\quad & x+y & = 2 & \quad|\,\text{Aus II: } x = -y \\
\text{II}\quad & x+y & = 0 & \quad|\,-y+y = 0 \Rightarrow 0 = 2 \enspace\lightning
\end{alignat*} \\

\textbf{c)}
\begin{alignat*}{4}
\text{I}\quad & x+y & = 2 & \quad|\,\text{Aus II: } y = 2 - x \\
\text{II}\quad & 2x + 2y & = 4 & \quad|\,2x + 2(2-x) = 4
\end{alignat*}
Also $4 = 4$. Daher kann $x$ oder $y$ frei gewählt werden.
\begin{align*}
    \text{Also }\begin{pmatrix}x\\y\end{pmatrix} = \begin{pmatrix}t\\2-t\end{pmatrix}\text{ mit }t\in \mathbb{R}
\end{align*}

\begin{samepage}
\textbf{Anschauung:} \\
\textbf{a)}
\begin{center}
\begin{tikzpicture}[>=triangle 45,font=\sffamily]
\draw[step=1cm,gray,very thin] (-2.9,-2.9) grid (3.9,3.9);
\draw[thick,->] (-3.5,0) -- (4.5,0) node[anchor=north west] {$x$-Achse};
\draw[thick,->] (0,-3.5) -- (0,4.5) node[anchor=south east] {$y$-Achse};
\foreach \x in {-2,-1,1,2,3,4}
        \draw (\x cm,4pt) -- (\x cm,-4pt) node[anchor=north] {$\x$};
\foreach \y in {-2,-1,1,2,3,4}
        \draw (4pt,\y cm) -- (-4pt,\y cm) node[anchor=east] {$\y$};

\draw[red,thick] (-2.5,-2.5) -- (3.5,3.5) node[anchor=south west] {$x=y$};
\draw[blue,thick] (-2.5,4.5) -- (3.5,-1.5) node[anchor=north west] {$y=2-x$};
\end{tikzpicture}
\end{center}
Die Lösung liegt genau in der Schnittmenge der beiden Geraden, hier also bei $(1, 1)$.
\end{samepage} \\

\begin{samepage}
\textbf{b)}
\begin{center}
\begin{tikzpicture}[>=triangle 45,font=\sffamily]
\draw[step=1cm,gray,very thin] (-2.9,-2.9) grid (3.9,3.9);
\draw[thick,->] (-3.5,0) -- (4.5,0) node[anchor=north west] {$x$-Achse};
\draw[thick,->] (0,-3.5) -- (0,4.5) node[anchor=south east] {$y$-Achse};
\foreach \x in {-2,-1,1,2,3,4}
        \draw (\x cm,4pt) -- (\x cm,-4pt) node[anchor=north] {$\x$};
\foreach \y in {-2,-1,1,2,3,4}
        \draw (4pt,\y cm) -- (-4pt,\y cm) node[anchor=east] {$\y$};

\draw[red,thick] (-2.5,2.5) -- (3.5,-3.5) node[anchor=south west] {$x=-y$};
\draw[blue,thick] (-2.5,4.5) -- (3.5,-1.5) node[anchor=north west] {$y=2-x$};
\end{tikzpicture}
\end{center}
Die Geraden sind hier parallel und schneiden sich nicht, es gibt also keine Lösung des LGS, die Lösungsmenge ist leer.
\end{samepage} \\

\begin{samepage}
\textbf{c)}
\begin{center}
\begin{tikzpicture}[>=triangle 45,font=\sffamily]
\draw[step=1cm,gray,very thin] (-2.9,-2.9) grid (3.9,3.9);
\draw[thick,->] (-3.5,0) -- (4.5,0) node[anchor=north west] {$x$-Achse};
\draw[thick,->] (0,-3.5) -- (0,4.5) node[anchor=south east] {$y$-Achse};
\foreach \x in {-2,-1,1,2,3,4}
        \draw (\x cm,4pt) -- (\x cm,-4pt) node[anchor=north] {$\x$};
\foreach \y in {-2,-1,1,2,3,4}
        \draw (4pt,\y cm) -- (-4pt,\y cm) node[anchor=east] {$\y$};


\draw[blue,thick] (-2.5,4.5) -- (3.5,-1.5) node[anchor=north west] {$y=2-x$};
\draw[red,thick,dashed] (-2.5,4.5) -- (3.5,-1.5) node[anchor=south west] {$x=4-2y$};
\end{tikzpicture}
\end{center}
Die Geraden liegen aufeinander, es gibt unendlich viele Lösungen.
\end{samepage}

\mdf{Bemerkung}
Ein homogenes LGS hat immer mindestens eine Lösung: $x_1 = \dots = x_n = 0$ (triviale Lösung)

\subsection{Erlaubte Umformungen eines LGS}
\mdf{Satz}
Folgende Umformungen verändern die Lösung eines LGS nicht:

\begin{description}
\item[i)]{Vertauschung zweier Zeilen}
\item[ii)]{Multiplikation einer Zeile mit einer Zahl ungleich Null}
\item[iii)]{Addition des Vielfachen einer Zeile zu einer anderen Zeile}
\end{description}

\subsection{Erweiterte Koeffizientenmatrix}
\mdf{Bemerkung}
Statt ein LGS als System von Gleichungen darzustellen nutzt man oft die \begr{erweiterte Koeffizientenmatrix}:
\begin{align*}
    \left(\begin{array}{cccc|c}
        a_{11} & a_{12} & \dots & a_{1n} & b_1 \\
        a_{21} & a_{22} & \dots & a_{2n} & b_2 \\
        \vdots & \vdots & \ddots & \vdots & \vdots \\
        a_{m1} & a_{m2} & \dots & a_{mn} & b_m
    \end{array}\right)
\end{align*}

\mdf{Definition}
Eine erweiterte Koeffizientenmatrix ist in \begr{Zeilenstufenform} (ZSF), falls

\begin{description}
\item[i)]{In jeder Zeile ist die erste Zahl ungleich Null eine Eins und steht weiter rechts als die erste nicht-Null-Zahl der Zeile darüber}
\item[ii)]{Alle Nullzeilen befinden sich am unteren Ende der Matrix}
\end{description}
Sie ist in reduzierter Zeilenstufenform, falls zusätzlich gilt:
\begin{description}
\item[iii)]{Über jeder führenden Eins stehen nur Nullen}
\end{description}

\begin{align*}
    \left(\begin{array}{ccccc|c}
        0 & 1 & 1 & 3 & 5 & 1 \\
        0 & 0 & 1 & 1 & 0 & 0 \\
        0 & 0 & 0 & 0 & 1 & 3
    \end{array}\right) \text{ ist in ZSF}
\end{align*}

\begin{align*}
    \left(\begin{array}{ccccc|c}
        1 & 0 & 0 & 0 & 0 & 2 \\
        0 & 0 & 0 & 1 & 0 & 3 \\
        0 & 0 & 0 & 0 & 1 & 1 \\
        0 & 0 & 0 & 0 & 0 & 0
    \end{array}\right) \text{ ist in red. ZSF}
\end{align*}

Wenn eine erweiterte Koeffizientenmatrix in ZSF oder red. ZSF ist (und damit auch das zugehörige LGS), lässt sich die Lösung des LGS leicht ablesen.

Der folgende Algorithmus überführt eine erweiterte Koeffizientenmatrix in ZSF:

\mdf{Satz}
Jede Matrix kann durch endlich viele Umformungen aus Satz 5 in ZSF gebracht werden.

\subsection{Gauß--Jordan--Algorithmus}
\mdf{Algorithmus}

Anmerkung: \texttt{a[x][y]} bezeichnet hier $a_{xy}$.
\begin{verbatim}
1   i = 1
2   j = 1
3   Gauß(i, j):
4       Falls i = m oder j = n+1:
5           Ende
6       Falls a[i][j] = 0:
7           Suche r > i mit a[r][j] != 0
8           Falls r existiert:
9               Tausche Zeilen r und i
10          Sonst:
11              Gauß(i, j+1)
12      Teile i-te Zeile durch a[i][j]
13      Für alle k > i: (Zeile k) - a[k][j] * (Zeile i)
14      Gauß(i+1, j+1)
\end{verbatim}

\mdf{Bemerkung}
Beim Lösen von Hand ist es oft einfacher, notwendige Umformungen durch \glqq{}scharfes Hingucken\grqq{} zu erkennen.

Beim Implementieren des Gauß--Jordan--Algorithmus treten oft Effekte auf, die mit der Repräsentation reeller Zahlen im Rechner zusammenhängen.

\mdf{Beispiel}
\begin{align*}
    \left(\begin{array}{ccccc|c}
        1 & 2 & 1 & 1 & 1 & 2 \\
        -1 & -2 & -2 & 2 & 1 & 1 \\
        2 & 4 & 3 & -1 & 0 & -1 \\
        1 & 2 & 2 & -2 & 1 & 1
    \end{array}\right)
\end{align*}
Löse das zugehörige LGS. Es gibt keine Lösung.


\termin{26.04.2016}

\kapitel{Matrizen}

\mdf{Definition}
Ein rechteckiges Schema von Skalaren $a_{ij} \in \mathbb{R}\text{ (oder }\mathbb{C}\text{)}$ in $m$ Zeilen und $n$ Spalten der Form
\begin{align*}
    A &=
    \begin{pmatrix}
        a_{11} & a_{12} & \dots & a_{1n} \\
        a_{21} & a_{22} & \dots & a_{2n} \\
        \vdots & \vdots & \ddots & \vdots \\
        a_{m1} & a_{m2} & \dots & a_{mn}
    \end{pmatrix}
\end{align*}
kurz $A = (a_{ij})$ heißt \begr[Matrix]{$m \times n$ Matrix}. $(m, n)$ heißt \begr{Dimension} der Matrix, $a_{ij}$ heißen \begr[Einträge (Matrix)]{Einträge}/\begr[Koeffizienten (Matrix)]{Koeffizienten}. Ist $A$ eine $n \times n$ Matrix (quadratische Matrix), so heißen die Koeffizienten $a_{ii}$ \begr{Diagonalelemente} von $A$.

Mit Matrizen kann man rechnen. Matrizen kann man addieren. Sind $A, B$ $m \times n$ Matrizen, man schreibt auch $A, B \in \mathbb{R}^{m \times n}$, mit $A = (a_{ij}), B = (b_{ij})$, dann
\begin{align*}
    A + B &= (a_{ij} + b_{ij}) =
    \begin{pmatrix}
        a_{11} + b_{11} & a_{12} + b_{12} & \dots & a_{1n} + b_{1n} \\
        a_{21} + b_{21} & a_{22} + b_{22} & \dots & a_{2n} + b_{2n} \\
        \vdots & \vdots & \ddots & \vdots \\
        a_{m1} + b_{m1} & a_{m2} + b_{m2} & \dots & a_{mn} + b_{mn}
    \end{pmatrix}
\end{align*}

\mdf{Lemma}
Bezüglich \glqq{}$+$\grqq{} bilden $m \times n$ Matrizen eine abelsche Gruppe, d.h. es gilt:
\begin{align*}
    A + B &= B + A \\
    A + (B + C) &= (A + B) + C
\end{align*}
Die Nullmatrix $0 = (0)\enspace i = 1,\dots,m\enspace j = 1,\dots,n$ ist neutrales Element der Addition.

Ist $A = (a_{ij})$, so gibt es ein eindeutig bestimmtes $-A = (-a_{ij})$ mit $A + (-A) = 0$. Beweis: Siehe Übungsblatt 5.

Matrizen sind ähnlich wie Vektoren. Für $k, h \in \mathbb{R}$ (oder $\mathbb{C}$):
\begin{itemize}
    \item{$kA = (k a_{ij})$}
    \item{$k(hA) = (kh)A$}
    \item{$k(A+B) = kA + kB$}
    \item{$(k+h)A = kA + hA$}
    \item{$1A = A$}
\end{itemize}

\mdf{Definition}
Zu einer $m \times n$ Matrix $A = (a_{ij})$ heißt die $n \times m$ Matrix $A^T = (a_{ji})$ \begr[Transponierte Matrix]{transponierte Matrix} zu $A$.

\mdf{Beispiel}
\begin{align*}
    A &= \begin{pmatrix}
        3 & 2 \\
        5 & -1 \\
        0 & 7
    \end{pmatrix} & A^T = \begin{pmatrix}
        3 & 5 & 0 \\
        2 & -1 & 7
    \end{pmatrix} \\
    B &= \begin{pmatrix}
        1 \\
        2 \\
        5
    \end{pmatrix} & B^T = \begin{pmatrix}
        1 & 2 & 5
    \end{pmatrix}
\end{align*}

\mdf{Definition}
Gilt für eine $n \times n$ Matrix $A$:
\begin{align*}
    A &= A^T
\end{align*}
so heißt $A$ \begr[Symmetrische Matrix]{symmetrische Matrix}.

\mdf{Definition}
Sei $A \in \mathbb{R}^{m \times n},\enspace B \in \mathbb{R}^{n \times r}$. Das Produkt $A \cdot B$ ist die $m \times r$ Matrix $C$ mit
\begin{align*}
    c_{ij} &= \sum_{k=1}^{n} a_{ik} b_{kj} = a_{i1} b_{1j} + a_{i2} b_{2j} + \dots + a_{in} b_{nj}
\end{align*}
Das Produkt von $A$ und $B$ ist nur definiert, falls $A$ so viele Spalten besitzt, wie $B$ Zeilen hat.

\mdf{Beispiel}
\begin{align*}
    A &= \begin{pmatrix}
        4 & 2 & 0 \\
        -1 & 3 & 5
    \end{pmatrix} \\
    B &= \begin{pmatrix}
        2 & 1 \\
        3 & 7 \\
        1 & 0
    \end{pmatrix} \\
    C &= \begin{pmatrix}
        7 & 1 \\
        2 & 5
    \end{pmatrix} \\
\end{align*}
Definiert sind nur $A\cdot B$, $B\cdot A$, $B\cdot C$ und $C\cdot A$. Nicht definiert sind hingegen $A\cdot C$ und $C\cdot B$.

Das \begr[Falk Schema]{Falk(sche) Schema} kann verwendet werden, um Matrizen einfach zu multiplizieren, hier am Beispiel $A\cdot B$. Man schreibt $A$ in die untere linke Zelle und $B$ in die obere rechte Zelle. Das Ergebnis $C$ steht in der unteren rechten Zelle.
\begin{center}
\begin{tabular}{r|l}
    & $\begin{pmatrix}2 & 1\\3 & 7\\1 & 0\end{pmatrix}$ \\ \hline
    $\begin{pmatrix}4 & 2 & 0\\-1 & 3 & 5\end{pmatrix}$ & $\begin{pmatrix} 14 & 18 \\ 12 & 20 \end{pmatrix}$
\end{tabular}
\end{center}
(Wird noch überarbeitet)

\mdf{Satz}
Seien $A, B, C$ Matrizen mit passender Dimension (passend für die Beispiele), $k \in \mathbb{R}$ (oder $\mathbb{C}$). Dann gelten:
\begin{itemize}
    \item{$(kA)B = k(AB)$}
    \item{$A(BC) = (AB)C$}
    \item{$(A+B)C = AC+BC$}
    \item{$A(B+C) = AB+AC$}
    \item{$(AB)^T = B^TA^T$}
\end{itemize}
Im Allgemeinen gilt $AB \neq BA$.

\mdf{Definition}
Die $n \times n$ Matrix
\begin{align*}
    \mathbbm{1} &= \begin{pmatrix}
        1 & 0 & \dots & 0 \\
        0 & 1 & \dots & 0 \\
        \vdots & \vdots & \ddots & \vdots \\
        0 & 0 & \dots & 1
    \end{pmatrix}
\end{align*}
heißt \begr[Einheitsmatrix]{$n \times n$ Einheitsmatrix}. Ist $A \in \mathbb{R}^{m \times n}$, so gilt $\mathbbm{1}_m A = A \mathbbm{1}_n = A$.

Lineare Gleichungssysteme kann man als Matrixgleichung schreiben. Statt
\begin{align*}
    a_{11}x_1 + a_{12}x_2 + \dots + a_{1n}x_n &= b_1 \\
    a_{21}x_1 + a_{22}x_2 + \dots + a_{2n}x_n &= b_2 \\
    \vdots &= \vdots \\
    a_{m1}x_1 + a_{m2}x_2 + \dots + a_{mn}x_n &= b_m
\end{align*}
schreiben wir
\begin{align*}
    A x = b\text{, mit } A \in \mathbb{R}^{m \times n},\quad x \in \mathbb{R}^{n(\times 1)},\quad b \in \mathbb{R}^{m (\times 1)}
\end{align*}



\termin{29.04.2016}
\subsection{Invertieren}
\mdf{Definition}
Ist $A \in \mathbb{R}^{n \times n}$ eine quadratische Matrix und gibt es zu $A$ eine Matrix $A^{-1}$ mit
\begin{align*}
	A^{-1} A = A A^{-1} = \mathbbm{1}_n
\end{align*}
so heißt $A$ \begr[Invertierbarkeit (Matrix)]{invertierbar} und die Matrix $A^{-1}$ heißt \begr[Inverse Matrix]{inverse Matrix} zu $A$.

\vspace{1cm}

Ist ein LGS mit $n$ Gleichungen und $n$ Unbekannten gegeben
\begin{align*}
	Ax = b\text{ mit }A \in \mathbb{R}^{n \times n},\quad x, b \in \mathbb{R}^n
\end{align*}
und A ist invertierbar, dann ist $x = A^{-1}b$ eindeutige Lösung des LGS.
\begin{align*}
	A^{-1}Ax = \mathbbm{1}_nx = x = A^{-1}b
\end{align*}

\mdf{Satz}
Sind $A, B \in \mathbb{R}^{n \times n}$ invertierbar, so ist auch $A \cdot B$ invertierbar und es gilt
\begin{align*}
	(AB)^{-1} &= B^{-1}A^{-1} \\
	\text{Denn: } (B^{-1}A^{-1})(AB) &= B^{-1}(A^{-1}A)B \\
	&= B^{-1} (\mathbbm{1}_n) B \\
	&= B^{-1}B \\
	&= \mathbbm{1}_n
\end{align*}

\mdf{Beispiel}
Ist $A = \begin{pmatrix}2&4\\-1&3\end{pmatrix}$ invertierbar? Gesucht ist also eine Matrix $B = \begin{pmatrix}b_{11}&b_{12}\\b_{21}&b_{22}\end{pmatrix}$ mit
\begin{align*}
	AB = \mathbbm{1}_2 &= \begin{pmatrix}1&0\\0&1\end{pmatrix} \\
	\begin{pmatrix}2&4\\-1&3\end{pmatrix}\begin{pmatrix}b_{11}&b_{12}\\b_{21}&b_{22}\end{pmatrix} &= \begin{pmatrix}1&0\\0&1\end{pmatrix} \\
	AB = \begin{pmatrix}2b_{11}+4b_{21}&2b_{12}+4b_{22}\\-b_{11}+3b_{21}&-b_{12}+3b_{22}\end{pmatrix} &= \begin{pmatrix}1&0\\0&1\end{pmatrix}
\end{align*}
Als LGS mit $4$ Gleichungen und $4$ Unbekannten
\begin{alignat*}{9}
2b_{11} & \enspace\enspace & & \enspace+\enspace & 4b_{12} & & & \enspace=\enspace & 1 \\
-b_{11} & & & \enspace+\enspace & 3b_{12} & & & \enspace=\enspace & 0 \\
 & \enspace\enspace & 2b_{12} & & & \enspace+\enspace & 4b_{22} & \enspace=\enspace & 0 \\
 & \enspace\enspace & -b_{12} & & & \enspace+\enspace & 3b_{22} & \enspace=\enspace & 1
\end{alignat*}
\begin{align*}
	\begin{pmatrix}
		2 & 0 & 4 & 0 \\
		-1 & 0 & 3 & 0 \\
		0 & 2 & 0 & 4 \\
		0 & -1 & 0 & 3
	\end{pmatrix} \begin{pmatrix}
		b_{11} \\
		b_{12} \\
		b_{21} \\
		b_{22}
	\end{pmatrix} &= \begin{pmatrix}
		1 \\
		0 \\
		0 \\
		1
	\end{pmatrix}
\end{align*}
Als $2$ LGS mit je $2$ Unbekannten
\begin{align*}
	\begin{pmatrix}2&4\\-1&3\end{pmatrix}\begin{pmatrix}b_{11}\\b_{21}\end{pmatrix} = \begin{pmatrix}1\\0\end{pmatrix}\text{ und }
	\begin{pmatrix}2&4\\-1&3\end{pmatrix}\begin{pmatrix}b_{12}\\b_{22}\end{pmatrix} = \begin{pmatrix}0\\1\end{pmatrix}
\end{align*}
Als erweiterte erweiterte (!) Koeffizientenmatrix
\begin{align*}
	\left(\begin{array}{cc|c|c}
		2 & 4 & 1 & 0 \\
		-1 & 3 & 0 & 1
	\end{array}\right) = \left(\begin{array}{c|c}A & \mathbbm{1}_2\end{array}\right)
\end{align*}
Mit Gauß--Jordan auf reduzierte ZSF bringen
\begin{align*}
	\left(\begin{array}{c|c}
		\begin{matrix}1 & 0\\0 & 1\end{matrix} & A^{-1}
	\end{array}\right) &= \left(\begin{array}{c|c}
		\mathbbm{1}_2 & A^{-1}
	\end{array}\right) \\
	\left(\begin{array}{cc|cc}
		2 & 4 & 1 & 0 \\
		-1 & 3 & 0 & 1
	\end{array}\right)\enspace &|\text{I} : 2 \\
	\left(\begin{array}{cc|cc}
		1 & 2 & \frac{1}{2} & 0 \\[0.3em]
		-1 & 3 & 0 & 1
	\end{array}\right)\enspace &|\text{II} + \text{I} \\
	\left(\begin{array}{cc|cc}
		1 & 2 & \frac{1}{2} & 0 \\[0.3em]
		0 & 5 & \frac{1}{2} & 1
	\end{array}\right)\enspace &|\text{II} : 5 \\
	\left(\begin{array}{cc|cc}
		1 & 2 & \frac{1}{2} & 0 \\[0.3em]
		0 & 1 & \frac{1}{10} & \frac{1}{5}
	\end{array}\right)\enspace &|\text{I} - 2\cdot\text{II} \\
	\left(\begin{array}{cc|cc}
		1 & 0 & \frac{3}{10} & -\frac{2}{5} \\[0.3em]
		0 & 1 & \frac{1}{10} & \frac{1}{5}
	\end{array}\right)
\end{align*}
Also $A^{-1} = \frac{1}{10}\begin{pmatrix}3 & -4 \\1 & 2\end{pmatrix}$.
\begin{align*}
	A^{-1}A = \frac{1}{10}\begin{pmatrix}3 & -4 \\1 & 2\end{pmatrix}\begin{pmatrix}2&4\\-1&3\end{pmatrix} = \frac{1}{10}\begin{pmatrix}10&0\\0&10\end{pmatrix} = \begin{pmatrix}1&0\\0&1\end{pmatrix}
\end{align*}

Also: Matrix invertieren durch simultanes Lösung von $n$ LGS. $\left(A|\mathbbm{1}_n\right)$ mit Gauß--Jordan--Algorithmus umformen zu $\left(\mathbbm{1}_n|A^{-1}\right)$. Ist dies möglich, so ist $A$ invertierbar.

\vspace{0.5cm}

\textbf{Frage:} Können wir einer Matrix \glqq{}ansehen\grqq{}, ob sie invertierbar ist?

Betrachte $2 \times 2$ LGS:
\begin{align*}
	a_{11}x_1 + a_{12}x_2 &= b_1\quad |\cdot a_{22} \\
	a_{21}x_1 + a_{22}x_2 &= b_2\quad |\cdot (-a_{12}) \\[0.5cm]
	a_{11}a_{22}x_1 + a_{12}a_{22}x_2 &= a_{22}b_1 \\
	-a_{12}a_{21}x_1 - a_{12}a_{22}x_2 &= -a_{12}b_2 \quad |\text{I} + \text{II} \\[0.5cm]
	a_{11}a_{22}x_1 + a_{12}a_{22}x_2 - a_{12}a_{22}x_2 - a_{12}a_{21}x_1 &= a_{22}b_1 - a_{12}b_2 \\
	\Leftrightarrow (a_{11}a_{22}-a_{12}a_{21}) x_1 &= a_{22}b_1 - a_{12}b_{2}
\end{align*}
Diese Gleichung enthält nur noch $x_1$. Analog erhält man
\begin{align*}
	(a_{11}a_{22}-a_{12}a_{21}) x_2 &= a_{11}b_1 - a_{21}b_{2}
\end{align*}

Ist $(a_{11}a_{22}-a_{12}a_{21}) \neq 0$, ist das LGS eindeutig lösbar.

\mdf{Definition}
Für eine $2 \times 2$ Matrix
\begin{align*}
    A &= \begin{pmatrix}a_{11} & a_{12} \\ a_{21} & a_{22}\end{pmatrix}
\end{align*}
heißt
\begin{align*}
    \text{det}(A) = \begin{vmatrix}a_{11} & a_{12} \\ a_{21} & a_{22}\end{vmatrix}
\end{align*}
\begr{Determinante} von $A$. Die Verallgemeinerung auf allgemeine Matrizen ist schwieriger. Die Herleitung geht über diese Vorlesung hinaus.


\printindex

\end{document}
