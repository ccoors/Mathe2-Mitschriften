\documentclass[a4paper,11pt,oneside,final,notitlepage,onecolumn]{article}
\usepackage[top=2.5cm, bottom=3cm, left=2cm, right=2cm]{geometry}

\usepackage{ucs}
\usepackage[utf8x]{inputenc}
\usepackage{amsmath}
\usepackage{amsfonts}
\usepackage{amssymb}
\usepackage{mathtools}
\usepackage{mathrsfs}
\usepackage{bbm}
\usepackage{ulem}
\usepackage{caption}
\usepackage{multicol}
\usepackage[table]{xcolor}
\usepackage[ngerman]{babel}
\usepackage[T1]{fontenc}
\usepackage[pdftex]{graphicx}
\usepackage{tabularx}
\usepackage{eurosym}
\usepackage{pifont}
\newcommand{\cmark}{\ding{51}}
\usepackage{ziffer}
\usepackage{enumerate}
\usepackage{stmaryrd}
\usepackage{soul}
\usepackage{xifthen}
\usepackage{imakeidx}
\usepackage{fancybox}
\makeindex

\usepackage{tikz}
\usetikzlibrary{positioning}
\usetikzlibrary{arrows}
\usetikzlibrary{shapes.misc}
\usetikzlibrary{angles}
\usetikzlibrary{quotes}
\usetikzlibrary{babel}
\usetikzlibrary{3d}
\usetikzlibrary{decorations.pathreplacing}
\usetikzlibrary{matrix}
\tikzset{cross/.style={cross out, draw=black, minimum size=2*(#1-\pgflinewidth), inner sep=0pt, outer sep=0pt},cross/.default={1pt}}

\usepackage{titlesec}
\usepackage{lipsum}

\usepackage{natbib}

\usepackage{textcomp}
\usepackage{pgfplots}
\pgfplotsset{compat=1.12}

\bibliographystyle{plainnat}

\newcommand{\termin}[1]{{\newpage\Large \textbf{#1:}}}
\newcommand{\myparagraph}[1]{\paragraph{#1}\mbox{}\\}

\newcommand{\sigalph}[0]{\sigma_{\alpha}}

\newcommand{\cosa}[0]{\text{cos}(\alpha)}
\newcommand{\sina}[0]{\text{sin}(\alpha)}

\newcommand{\cosah}[0]{\text{cos}\left(\frac{\alpha}{2}\right)}
\newcommand{\sinah}[0]{\text{sin}\left(\frac{\alpha}{2}\right)}

\titleformat{\section}{\LARGE\bfseries}{\textbf{Kapitel \thesection:}}{1em}{}
\titleformat{\subsection}{\normalfont\bfseries}{\textbf{\Roman{subsection})}}{1em}{}
\titleformat{\subsubsection}{\normalfont}{\textbf{\alph{subsubsection})}}{1em}{}

\usepackage{xspace}
\newcommand{\titledot}[0]{\texorpdfstring{$\cdot$}{;}\xspace}

\newcommand{\leadingzero}[1]{\ifnum #1<10 0\the#1\else\the#1\fi}
\newcommand{\todaynew}{\leadingzero{\day}.\leadingzero{\month}.\the\year}

\newcounter{MathCounter}
\setcounter{MathCounter}{0}
\newcommand{\mdf}[1]{\vspace{0.2cm}\pagebreak[2]\phantomsection\textbf{#1 \stepcounter{MathCounter}\arabic{MathCounter}:}\addcontentsline{toc}{subsection}{#1 \arabic{MathCounter}}}

\newcommand{\kapitel}[1]{\setcounter{MathCounter}{0}
\section{#1}}

\newcommand{\norm}[1]{\left\lVert#1\right\rVert}

\newcommand{\begr}[2][]{\ifthenelse{\isempty{#1}}{\index{#2}}{\index{#1}}\textbf{#2}}

% \setcounter{secnumdepth}{0}
\setcounter{tocdepth}{2}

\setlength{\parindent}{0cm}

% \rowcolors{2}{gray!15}{white}

\usepackage{url}
\def\UrlBreaks{\do\/\do-}
\usepackage[pdftex,breaklinks]{hyperref}
\hypersetup{
  pdftitle    = {MG2 Mitschrift},
  pdfsubject  = {Mitschrift},
  pdfauthor   = {Christian Friedrich Coors},
  pdfkeywords = {},
  pdfcreator  = {pdfLaTeX},
  pdfproducer = {LaTeX},
  colorlinks = false
}

\renewcommand{\maketitle}[3]{
  \begin{center}
    \title{#1}
    \Large
    \textbf{#1}
    
    \normalsize
    \textit{#2}

    \date{\today}
    \textit{#3}
  \end{center}
  \normalsize
  
  \hypersetup{
    pdftitle    = {#1},
    pdfsubject  = {#1},
    pdfauthor   = {},
    pdfkeywords = {},
    pdfcreator  = {pdfLaTeX},
    pdfproducer = {LaTeX},
  }
}

\begin{document}
\maketitle{Mathematische Grundlagen 2 -- Mitschriften}{Mathematische Grundlagen 2: Lineare Algebra und Differential- und Integralrechnung}{Universität Bremen \titledot Sommersemester 2016}
\section*{Anmerkungen}
\textbf{Diese Mitschriften können die persönliche Anwesenheit in der Vorlesung nicht ersetzen!} Bei diesem Dokument handelt es sich hauptsächlich um mehr oder weniger exakte Abschriften von der Tafel, teilweise mit persönlichen Anmerkungen und Notizen. Eventuelle Fehler in diesem Dokument sind nicht ganz auszuschließen. Ich bemühe mich jedoch, ein möglichst gutes Skript zu erstellen. Fehler können mir gerne gemeldet werden, zum Beispiel persönlich an mich oder im zugehörigen GitHub Repo als Issue (\url{https://github.com/ccoors/Mathe2-Mitschriften}).

\vspace{0.3cm}

\begin{center}
\shadowbox{\parbox{16cm}{Das Inhaltsverzeichnis enthält keine Begriffe/Überschriften (außer in Kapitel 1). Die definierten Begriffe befinden sich am Ende des Dokuments im \textbf{Index}. Dort sind die Seitenzahlen \textbf{verlinkt}.}}
\end{center}

\tableofcontents

\termin{05.04.2016}
\section{Zahlen}
\subsection{Zahlenmengen}
\begin{itemize}
\item{Die Menge $\mathbb{N} = \{0,1,2,3,...\}$ heißt \underline{Menge der natürlichen Zahlen}. Für uns beinhalten die natürlichen Zahlen die $0$.}
\item{Die Menge $\mathbb{Z} = \{...,-3,-2,-1,0,1,2,3,...\}$ heißt \underline{Menge der ganzen Zahlen}.}
\item{Die Menge $\mathbb{Q} = \left\{\frac{p}{q}\,|\,p,q \in \mathbb{Z}, q \neq 0\right\}$ heißt \underline{Menge der rationalen Zahlen}.}
\end{itemize}

\subsection{Satz von Euklid}
Es gibt keine rationale Zahl $q \in \mathbb{Q}$ mit $q^2 = 2$.

\subsection{Reelle Zahlen}
Die Menge $\mathbb{R}$ der reellen Zahlen ist die Vereinigung der Menge der rationalen Zahlen mit allen Zahlen, die sich durch rationalen Zahlen beliebig approximieren lassen.

\subsection{Wurzel}
Ist $b^n = a$ für $a, b \in \mathbb{R} > 0, n \in \mathbb{N}$, so heißt b die \underline{$n$-te Wurzel} von $a$: $b = \sqrt[n]{a}=a^\frac{1}{n}$. $b$ ist für alle $a > 0$ eindeutig. Der Vorgang des Wurzelziehens heißt auch \underline{radizieren}.

\subsection{Kurzschreibweisen}
Wir führen Kurzschreibweisen ein:
\begin{align*}
[a,b] &:= \{x\in\mathbb{R}\,|\,a \leq x \leq b\}\text{ heißt abgeschlossenes Intervall,} \\
[a,b) &:= \{x\in\mathbb{R}\,|\,a \leq x < b\}\text{ und }(a,b] := \{x\in\mathbb{R}\,|\,a < x \leq b\}\text{ heißen halboffene Intervalle und} \\
(a,b) &:= \{x\in\mathbb{R}\,|\,a < x < b\}\text{ heißt offenes Intervall.} \\
&\text{Sonderfälle:} \\
[a, \infty) &:= \{x\in\mathbb{R}\,|\,a \leq x\} \\
(-\infty, b] &:= \{x\in\mathbb{R}\,|\,x \leq b\} \\
(a, \infty) &:= \{x\in\mathbb{R}\,|\,a < x\} \\
(-\infty, b) &:= \{x\in\mathbb{R}\,|\,x < b\}
\end{align*}

\subsection{Beschränktheit nach oben}
Eine Menge $M \subseteq \mathbb{R}$ heißt nach oben beschränkt, falls es ein $k \in \mathbb{R}$ gibt, so dass $x \le k$ für alle $x \in M$. $k$ heißt obere Schranke von $M$. Ist $M$ nach oben beschränkt, gibt es mehrere obere Schranken von $M$: $M = [3,5)$. Offenbar ist $k = 17$ eine obere Schranke von $M$. Die kleinste obere Schranke von $M$ heißt \underline{Supremum} von $M$, kurz, $\text{sup}(M)$.

\subsection{Satz über die Vollständigkeit von $\mathbb{R}$}
Jede nach oben beschränkte Menge $M \subseteq \mathbb{R}$ besitzt ein Supremum in $\mathbb{R}$. Für $\mathbb{Q}$ gilt dies nicht. Das Supremum der Menge $[0,\sqrt{3}) \subseteq \mathbb{Q}$ wäre in $\mathbb{R}$ $\sqrt{3}$, diese Zahl liegt aber nicht in $\mathbb{Q}$. Dies unterscheidet die reellen Zahlen fundamental von den rationalen Zahlen.

\subsection{Beschränktheit nach unten}
Siehe Beschränktheit nach oben. Wichtig: Die größte untere Schranke heißt \underline{Infimum}, kurz $\text{inf}(M)$. Es gilt: $\text{inf}(M) = -\text{sup}(-M)$. ($-M = \{-m\,|\,m \in M\}$).

\subsection{Satz zu Supremum und Infimum} Supremum und Infimum einer Menge M müssen nicht in der Menge liegen. Ist $\text{sup}(M) \in M$, so ist es auch das größte Element (Maximum) von $M$, kurz: $\text{max}(M)$. Analog dazu: Ist $\text{inf}(M) \in M$, so ist es auch das kleinste Element (Minimum) von $M$, kurz: $\text{min}(M)$.

\subsection{Literatur}
Alles bis hierher ist zu finden in:

\href{http://www.mat.univie.ac.at/~gerald/ftp/book-mfi/mfi1.pdf}{Mathematik für Informatiker -- Band 1: Diskrete Mathematik und Lineare Algebra. Springer. Gerald Teschl, Susanne Teschl. Kapitel 2.1.}



\termin{08.04.2016}
\subsection{Betrag}
Der Betrag einer reellen Zahl x ist gegeben durch
\begin{align*}
|x| &= \left\{\begin{array}{cl} x, & \mbox{falls }x \geq 0\\ -x, & \mbox{falls } x < 0 \mbox{/sonst} \end{array}\right.
\end{align*}

\subsection{Körper}
Ein Körper $K$ heißt angeordnet, falls es auf $K$ eine totale Ordnung $\leq$ gibt, so dass gelten:
\begin{description}
\item[(A1)]{Für alle $x, y, z \in K : x \leq y \Rightarrow x \leq y + z$}
\item[(A2)]{Für alle $x, y, z \in K : x \leq y, z > 0 \Rightarrow x\cdot z \leq y\cdot z$}
\end{description}
(Monotonie der Addition und Multiplikation)

\subsection{Vollständig angeordneter Körper}
Die reellen Zahlen bilden den bis auf Isomrphie eindeutig bestimmten \underline{vollständig angeordneten Körper}.

Endliche Körper können nicht angeordnet werden, z.B. $Z_7$.

\begin{align*}
&\overline{0} \leq \overline{1} \leq \overline{2} \leq \overline{3} \leq \overline{4} \leq \overline{5} \leq \overline{6} \\
&\overline{5} \leq \overline{6}\text{, aber }\overline{5} + \overline{1} = \overline{6} \neq \overline{6} + \overline{1} = \overline{0}
\end{align*}

\subsection{Zwischenfazit Zahlenmengen}
\begin{align*}
	\mathbb{N} \subseteq \mathbb{Z} \subseteq \mathbb{Q} \subseteq \mathbb{R}
\end{align*}

\subsection{Reelle Zahlen auf der Zahlengeraden}
Man kann sich die reellen Zahlen als Zahlengerade vorstellen. Diese Zahlengerade hat keine Lücken mehr.

\begin{center}
\begin{tikzpicture}[>=triangle 45,font=\sffamily]
    \draw [<->] (0,0) -- (4,0);
    \draw (2 cm, 4pt) -- (2 cm, -4pt) node[anchor=north] {$0$};
    \draw (2.5 cm, 4pt) -- (2.5 cm, -4pt) node[anchor=north] {$\frac{1}{2}$};
    \draw (3 cm, 4pt) -- (3 cm, -4pt) node[anchor=north] {$1$};
\end{tikzpicture}
\end{center}

Aber: Es gibt Gleichungen, die wir nicht lösen können.

$x^2 + 1 = 0$ hat keine Lösung in den reellen Zahlen, da es kein $x \in \mathbb{R}$ gibt, so dass $x^2 = -1$.

\subsection{Komplexe Zahlen}
Die Menge $\mathbb{C} = \{x + i\cdot y\,|\,x, y \in \mathbb{R}\}$ heißt \underline{Menge der komplexen Zahlen}.

Die Zahl $i = 0 + 1\cdot i$ ist definiert als $i^2 = -1$ und heißt \underline{imaginäre Einheit}. Für eine Zahl $z = x + i\cdot y$ heißt $x$ \underline{Realteil von z} ($\text{Re}(z) := x$), $y$ heißt \underline{Imaginärteil von z} ($\text{Im}(z) := y$).

\subsubsection{Beispiel}
$z = 3 - 2\cdot i$, dann $\text{Re}(z) = 3, \text{Im}(z) = -2$

\subsection{Komplexe Zahlen bilden einen Körper}
Die komplexen Zahlen bilden einen Körper.

\begin{itemize}
\item{\textbf{Addition}: Seien $z_1 = x_1 + i \cdot y_1$ und $z_2 = x_2 + i \cdot y_2$, dann ist $z_1 + z_2 = (x_1 + i \cdot y_1) + (x_2 + i \cdot y_2) = (x_1 + x_2) + i \cdot (y1 + y2)$.}
\item{\textbf{Multiplikation}: $z_1 \cdot z_2 = (x_1 + i \cdot y_1) \cdot (x_2 + i \cdot y_2) = x_1 \cdot x_2 + x_1 \cdot i \cdot y_2 + i \cdot y_1 \cdot x_2 + i^2 \cdot y_1 \cdot y_2 = x_1 \cdot x_2 - y_1 \cdot y_2 + i \cdot y_1 \cdot x_2 + i \cdot y_2 \cdot x_1 = (x_1 \cdot x_2 - y_1 \cdot y_2) + i \cdot (x_1 \cdot y_2 + x_2 \cdot y_1)$}
\end{itemize}

\subsubsection{Beweis}
\begin{enumerate}
\item{Assiziativität der Addition: \cmark}
\item{Neutralelement der Addition: $0 + 0 \cdot i = 0$ \cmark}
\item{Inverses bezüglich Addition: $z = x + i \cdot y \Rightarrow -z = -x - i \cdot y$,

Denn: $(x + i \cdot y) + (-x - i \cdot y) = (x - x) + i (y - y) = 0 + i \cdot 0 = 0$ \cmark}
\item{Kommutativität der Addition: \cmark}
\end{enumerate}

\myparagraph{Assoziativgesetz der Multiplikation}
Siehe P2 auf Blatt 2

\myparagraph{Neutralelement der Multiplikation}
$1 + 0 \cdot i = 1$

\myparagraph{Inverse bezüglich Multiplikation}
Ist $z = x + i \cdot y$, dann ist
\begin{align*}
	z^{-1} = \frac{x}{x^2 + y^2} + i \cdot \frac{-y}{x^2 + y^2}
\end{align*}
denn:
\begin{align*}
(x + i \cdot y) \cdot \left(\frac{x}{x^2 + y^2} + i \cdot \frac{-y}{x^2 + y^2}\right) = \left(\frac{x^2 + y^2}{x^2 + y^2}\right) + i \cdot \frac{x \cdot y - y \cdot x}{x^2 + y^2} = 1 + i \cdot 0
\end{align*}

\myparagraph{Distributivgesetz}
Siehe H4 auf Blatt 2

\subsection{Komplex konjugierte Zahlen}
Für eine komplexe Zahl $z = x + i \cdot y$ heißt $\overline{z} = x - i \cdot y$ die zu $z$ komplex konjugierte Zahl.

Damit $\text{Re}(z) = \frac{z+\overline{z}}{2}, \text{Im}(z)=\frac{z-\overline{z}}{2\cdot i}$

Es gilt: $\overline{z_1 + z_2} = \overline{z_1 + z_2}, \overline{z_1 \cdot z_2} = \overline{z_1 \cdot z_2}$
\begin{align*}
\overline{z^{-1}} = \overline{z}^{-1}
\end{align*}

\subsection{Der Betrag von komplexen Zahlen}
Der Betrag einer komplexen Zahl $z = x+iy$ ist gegeben durch $|z| = \sqrt{z \cdot \overline{z}} = \sqrt{x^2 + y^2}$.

Wir können die komplexen Zahlen geometrisch durch die Gaußsche Zahlenebene veranschaulichen.

\begin{center}
\begin{tikzpicture}[>=triangle 45,font=\sffamily]
% 	See https://de.sharelatex.com/blog/2013/08/27/tikz-series-pt1.html
%     \draw[step=1cm,gray,very thin] (-1.9,-1.9) grid (5.9,5.9);
    \draw[thick,->] (-1,0) -- (4.5,0);
	\draw[thick,->] (0,-1) -- (0,4.5);
	\draw[thick,->] (0,0) -- (4.5,0) node[anchor=north west] {Realteil};
	\draw[thick,->] (0,0) -- (0,4.5) node[anchor=south east] {Imaginärteil};
	\foreach \x in {1,2,3,4}
		\draw (\x cm,4pt) -- (\x cm,-4pt) node[anchor=north] {$\x$};
	\foreach \y in {1,2,3,4}
		\draw (4pt,\y cm) -- (-4pt,\y cm) node[anchor=east] {$\y$};
	
	\draw[thick,->] (0,0) -- (2,3);
	\draw (2,3) node [cross=5pt,red] {};
	\node [align=left] at (3,3) {$(x+iy)$};
\end{tikzpicture}
\end{center}

\subsection{Dreiecksungleichung}
Für alle $x, w \in \mathbb{C}$ gilt:
$|z + w| \leq |z| + |w|$ (\underline{Dreiecksungleichung})

Anschaulich:

\begin{center}
\begin{tikzpicture}[>=triangle 45,font=\sffamily]
    \draw[thick,->] (-1,0) -- (4.5,0);
	\draw[thick,->] (0,-1) -- (0,4.5);
	\draw[thick,->] (0,0) -- (4.5,0) node[anchor=north west] {Realteil};
	\draw[thick,->] (0,0) -- (0,4.5) node[anchor=south east] {Imaginärteil};
	\foreach \x in {1,2,3,4}
		\draw (\x cm,4pt) -- (\x cm,-4pt) node[anchor=north] {$\x$};
	\foreach \y in {1,2,3,4}
		\draw (4pt,\y cm) -- (-4pt,\y cm) node[anchor=east] {$\y$};
	
	\draw[thick,->] (0,0) -- (2,1);
	\draw[thick,->] (2,1) -- (3,3);
	\draw[thick,->] (0,0) -- (3,3);
	\draw (2,1) node [cross=5pt,red] {};
	\draw (3,3) node [cross=5pt,red] {};
	\node [align=left] at (2.4,1) {$z$};
	\node [align=left] at (3.4,3) {$w$};
\end{tikzpicture}
\end{center}

\subsection{Quadrieren von komplexen Zahlen}
Zu jeder Zahl $z \in \mathbb{C}$ gibt es ein $w \in \mathbb{C}$ mit $z = w^2$.

\subsubsection{Beweis}
Sei $z = x + i \cdot y$:
\begin{enumerate}
\item{$z = 0$; dann $z = 0^2$, also $w = 0$}
\item{$z \neq 0$ und $x > 0$. Dann setze $u := \sqrt{(1/2) \cdot (x + \sqrt{x^2 + y^2})}$. Dann ist $u \in \mathbb{R}$.

Mit $v := \frac{y}{2 \cdot u}$ gilt: $(u + v \cdot i^2) = w^2 = x + i \cdot y = z$}
\item{$z \neq 0$ und $x \leq 0$. Dann setze $u := \sqrt{\frac{1}{2} \cdot (-x + \sqrt{x^2 + y^2})}$ Dann ist $v \in \mathbb{R}$.

Mit $u := \frac{y}{2 \cdot v}$ gilt: $(u + v \cdot i^2) = w^2 = x + iy = z$}

\end{enumerate}

\subsection{$\mathbb{C}$ anordnen}
$\mathbb{C}$ lässt sich nicht anordnen.


\termin{12.04.2016}
\kapitel{Geometrie Teil 1}
\begin{center}
\begin{tikzpicture}[>=triangle 45,font=\sffamily]
\draw[step=1cm,gray,very thin] (0.1,0.1) grid (3.9,3.9);
\draw[thick,->] (0,0) -- (4.5,0) node[anchor=north west] {$x$};
\draw[thick,->] (0,0) -- (0,4.5) node[anchor=south east] {$y$};
\foreach \x in {1,2,3,4}
        \draw (\x cm,4pt) -- (\x cm,-4pt) node[anchor=north] {$\x$};
\foreach \y in {1,2,3,4}
        \draw (4pt,\y cm) -- (-4pt,\y cm) node[anchor=east] {$\y$};

\draw (3,2) node [cross=5pt,red] {};
\node [align=left] at (3.4,2) {$P$};
\end{tikzpicture}
\end{center}

Punkte in der Ebene können wie folgt beschrieben werden:
\begin{enumerate}
	\item{Als geordnetes Paar: $P = (3, 2)$.}
	\item{Wir wählen \glqq{}grundlegende Basiselemente\grqq{} z.B. $a$ = \glqq{}gehe\grqq{} 1 Schritt in $x$-Richtung und $b$ = \glqq{}gehe\grqq{} 1 Schritt in $y$-Richtung. Dann können wir $P$ schreiben als $3a + 2b$.}
	\item{Als Vektor: $P = \begin{pmatrix}3 \\ 2\end{pmatrix}$, wie ein geordnetes Paar untereinander.}
\end{enumerate}

Offenbar hängt die Darstellung von $P$ von der Achsenbeschriftung ab, bzw. von den \glqq{}Basiselementen\grqq{}.

Z.B. Setze $a$ = \glqq{}gehe\grqq{} 3 Schritte in $x$-Richtung, $b$ wie eben, dann ist $P = a + 2b$. Oder $a$ = \glqq{}gehe\grqq{} 1 Schritt entgegen der $x$-Richtung, dann $P = -3a + 2b$.

\subsection{Die reelle Ebene}
\mdf{Definition}
Die \underline{reelle Ebene} ist das kartesische Produkt $\mathbb{R}^2 = \mathbb{R} \times \mathbb{R}$ und enthält alle Punkte $(x, y)$ mit $x, y \in \mathbb{R}$, also $\mathbb{R}^2 = \{(x,y)\,|\,x,y \in \mathbb{R}\}$. Statt $(x, y)$ schreiben wir in Zukunft $\begin{pmatrix}x \\ y\end{pmatrix} \in \mathbb{R}^2$ für Punkte in der reellen Ebene.

$\begin{pmatrix}x \\ y\end{pmatrix}$ heißt (2-dimensionaler) \underline{Vektor} oder \underline{Spaltenvektor}.

Analog können wir den dreidimensionalen Raum beschreiben als
\begin{align*}
	\mathbb{R}^3 = \mathbb{R} \times \mathbb{R} \times \mathbb{R} = \{(x,y,z)\,|\,x,y,z \in \mathbb{R}\}
\end{align*}

Offenbar ist $\mathbb{R}^3 = \mathbb{R}^2 \times \mathbb{R}$.

Analog definiert man den $\mathbb{R}^n$, den $n$-dimensionalen reellen Raum:

\begin{align*}
	\mathbb{R}^n = \left\{\begin{pmatrix}x_1 \\ x_2 \\ x_3 \\ \vdots \\ x_n \end{pmatrix}\,\Bigg|\, x_1, x_2, \dots, x_n \in \mathbb{R} \right\}
\end{align*}

\subsection{Rechnen mit Vektoren}
\mdf{Definition}
\begin{align*}
	\text{Für 2 Vektoren }x = \begin{pmatrix}x_1 \\ \vdots \\ x_n \end{pmatrix},\enspace y = \begin{pmatrix}y_1 \\ \vdots \\ y_n \end{pmatrix}\text{ ist } x + y\text{ definiert als }\begin{pmatrix}x_1 \\ \vdots \\ x_n \end{pmatrix} + \begin{pmatrix}y_1 \\ \vdots \\ y_n \end{pmatrix} = \begin{pmatrix}x_1 + y_1 \\ \vdots \\ x_n + y_n \end{pmatrix}
\end{align*}

\mdf{Beispiel}
\begin{align*}
	\begin{pmatrix} -1 \\ 0 \\ 2 \\ 1,5 \end{pmatrix} + \begin{pmatrix} 1 \\ \pi \\ 0 \\ -3 \end{pmatrix} = \begin{pmatrix} 0 \\ \pi \\ 2 \\ -1,5 \end{pmatrix}
\end{align*}
Beachte: Man kann nur gleichdimensionale Vektoren addieren.

\subsection{Die Skalarmultiplikation}
\mdf{Definition}
Die \underline{Skalarmultiplikation} eines Vektors $x \in \mathbb{R}^n$ mit einer festen Zahl (Skalar genannt) $\lambda \in \mathbb{R}$ ist gegeben durch
\begin{align*}
	\lambda \cdot x = \begin{pmatrix} \lambda \cdot x_1 \\ \vdots \\ \lambda \cdot x_n \end{pmatrix}
\end{align*}

\mdf{Beispiel}
\begin{align*}
	\lambda = 2, \enspace x = \begin{pmatrix}2 \\ 1\end{pmatrix} \quad \lambda \cdot x = \begin{pmatrix}4 \\ 2\end{pmatrix}
\end{align*}

\subsection{Rechnen mit Vektoren}
Mit Vektoren kann man fast wie mit Zahlen rechnen.

Für $x,y,z \in \mathbb{R}^n$ gilt
\begin{itemize}
	\item{$(x+y)+z = x+(y+z)$ Assoziativgesetz}
	\item{$x+y = y+x$ Kommutativgesetz}
	\item{Der Nullvektor $0 = \begin{pmatrix}0 \\ \vdots \\ 0\end{pmatrix} \in \mathbb{R}^n$ ist neutrales Element der Addition}
	\item{Zu jedem $x \in \mathbb{R}^n$ gibt es ein \glqq{}negatives\grqq{} $-x \in \mathbb{R}^n$ mit $x + (-x) = 0 \in \mathbb{R}^n$}
	\item{Für $\lambda,\mu \in \mathbb{R}$ gilt $(\lambda + \mu)\cdot x = \lambda \cdot x + \mu \cdot x$ Distributivgesetz}
	\item{Für $\lambda,\mu \in \mathbb{R}$ gilt $(\lambda \cdot \mu) \cdot x = \lambda \cdot (\mu \cdot x)$ Assoziativgesetz der Skalarmultiplikation}
\end{itemize}

\mdf{Bemerkung}
Für Vektoren gibt es kein \glqq{}sinnvolles\grqq{} Produkt, so dass $x\cdot y$ in $\mathbb{R}^n$ liegt und die Eigenschaften gelten, die man von dem Produkt erwartet.

\subsection{Lineare Unabhängigkeit}
\mdf{Definition}
Eine Menge von Vektoren $v_1,\dots,v_k \in \mathbb{R}^n$ heißt \underline{linear abhängig} falls die Gleichung

\begin{align*}
	\lambda _1 \cdot v_1 +\dots+\lambda _k \cdot v_k = 0 (\lambda _1,\dots,\lambda _k \in \mathbb{R})
\end{align*}

nur die Lösung $\lambda _1,\dots,\lambda _k = 0$ hat.

Nicht linear unabhängige Vektoren heißen linear abhängig.

Für Vektoren $v_1,\dots,v_k \in \mathbb{R}^n$ und Skalare $\lambda _1,\dots\lambda _k \in \mathbb{R}$ heißt $\lambda _1 \cdot v_1+\dots+\lambda _k \cdot v_k$ \underline{Linearkombination}.

\subsection{Unterraum}
\mdf{Definition}
Eine Teilmenge $U \subseteq \mathbb{R}^n$ heißt \underline{Unterraum} von $\mathbb{R}^n$ falls gilt:
\begin{description}
	\item[(U1)]{$0 \in U$}
	\item[(U2)]{$\forall u \in U,\, \forall \lambda \in \mathbb{R}\,:\,\lambda \cdot u \in U$}
	\item[(U3)]{$\forall u,v \in U\,:\,u+v \in U$}
\end{description}

\mdf{Beispiel}
Betrachte $\mathbb{R}^3$
\begin{align*}
	U = \left\{\begin{pmatrix}x \\ y \\ 0\end{pmatrix}\,\Bigg|\,x,y \in \mathbb{R}\right\}\text{ ist Unterraum des }\mathbb{R}^3
\end{align*}

Betrachte $\mathbb{R}^2$
\begin{align*}
	U = \left\{\begin{pmatrix}x \\ -x\end{pmatrix}\,\Bigg|\,x \in \mathbb{R}\right\}\,\subseteq\,\mathbb{R}^2\text{ ist Unterraum des }\mathbb{R}^2
\end{align*}

\begin{center}
\begin{tikzpicture}[>=triangle 45,font=\sffamily]
\draw[step=1cm,gray,very thin] (-3.9,-3.9) grid (3.9,3.9);
\draw[thick,->] (-4.5,0) -- (4.5,0) node[anchor=north west] {$x$};
\draw[thick,->] (0,-4.5) -- (0,4.5) node[anchor=south east] {$y$};
\foreach \x in {-4,-3,-2,-1,1,2,3,4}
        \draw (\x cm,4pt) -- (\x cm,-4pt) node[anchor=north] {$\x$};
\foreach \y in {-4,-3,-2,-1,1,2,3,4}
        \draw (4pt,\y cm) -- (-4pt,\y cm) node[anchor=east] {$\y$};

\draw[thick,red] (-3.5,3.5) -- (3.5,-3.5);
\end{tikzpicture}
\end{center}

Betrachte $\mathbb{R}^n$, $U = \left\{\begin{pmatrix}0 \\ \vdots \\ 0\end{pmatrix}\right\}$ ist Unterraum des $\mathbb{R}^n$.

\subsection{Lineare Hülle/Span}
\mdf{Definition}
Seien $v_1,\dots,v_k \in \mathbb{R}^n$. Die Menge
\begin{align*}
	\text{span}(v_1,\dots,v_k) = \{\lambda _1 \cdot v_1+\dots+\lambda _k \cdot v_k\,|\,\lambda _i \in \mathbb{R}\}
\end{align*}

heißt \underline{lineare Hülle} oder \underline{Span} von $v_1,\dots,v_k$.

Die Vektoren $v_1,\dots,v_k$ heißen Erzeugendensystem eines Unterraums $U$ falls $U = \text{span}(v_1,\dots,v_k)$.

Die Vektoren $v_1,\dots,v_k$ heißen Basis von $U$ falls $U = \text{span}(v_1,\dots,v_k)$ und $v_1,\dots,v_k$ sind linear unabhängig.


\termin{15.04.2016}

\mdf{Beispiel}
Die Standardbasis des $\mathbb{R}^n$ ist die Menge der Vektoren
\begin{align*}
	v_1 = \begin{pmatrix}1\\0\\\vdots\\0\end{pmatrix}, v_2 = \begin{pmatrix}0\\1\\0\\\vdots\\0\end{pmatrix}, \dots, v_n = \begin{pmatrix}0\\\vdots\\0\\1\end{pmatrix}
\end{align*}

$v_1,\dots,v_n$ sind linear unabhängig (trivial).

$v_1,\dots,v_n$ bilden ein Erzeugendensystem von $\mathbb{R}^n$, denn jeder Vektor $x = \begin{pmatrix}x_1\\\vdots\\x_n\end{pmatrix}$ lässt sich schreiben als
\begin{align*}
	x = \begin{pmatrix}x_1\\\vdots\\x_n\end{pmatrix} = x_1\begin{pmatrix}1\\0\\\vdots\\0\end{pmatrix} + x_2\begin{pmatrix}0\\1\\0\\\vdots\\0\end{pmatrix} +\dots+x_n\begin{pmatrix}0\\\vdots\\0\\1\end{pmatrix}
\end{align*}

Konkret: Standardbasis von $\mathbb{R}^2\,:\,v_1=\begin{pmatrix}1\\0\end{pmatrix},v_2=\begin{pmatrix}0\\1\end{pmatrix}$.

Eine andere Basis des $\mathbb{R}^2$ ist $v_1=\begin{pmatrix}2\\0\end{pmatrix},v_2=\begin{pmatrix}0\\1\end{pmatrix}$. Hier lässt sich jeder Vektor $x = \begin{pmatrix}x_1\\x_2\end{pmatrix}$ schreiben als $x = \begin{pmatrix}x_1\\x_2\end{pmatrix} = \frac{x_1}{2}\cdot\begin{pmatrix}2\\0\end{pmatrix} + x_2 \cdot\begin{pmatrix}0\\1\end{pmatrix}$.

\mdf{Beispiel}
Es seien $v_1 = \begin{pmatrix}3\\5\\2\end{pmatrix}, v_2 = \begin{pmatrix}4\\0\\1\end{pmatrix}, v_3 = \begin{pmatrix}2\\2\\1\end{pmatrix}$.
\begin{align*}
	\text{span}(v_2, v_3) = \left\{\lambda _1 \begin{pmatrix}4\\0\\1\end{pmatrix} + \lambda _2 \begin{pmatrix}2\\2\\1\end{pmatrix} \,\Bigg|\, \lambda _1, \lambda _2 \in \mathbb{R} \right\}
\end{align*}

Z.B. ist $\begin{pmatrix}4\\2\\1,5\end{pmatrix} \in \text{span}(v_2, v_3)$, da $\begin{pmatrix}4\\2\\1,5\end{pmatrix} = \frac{1}{2} \begin{pmatrix}4\\0\\1\end{pmatrix} + 1 \begin{pmatrix}2\\2\\1\end{pmatrix}$.

\vspace{0.3cm}

Frage: Liegt $v_1$ in $\text{span}(v_2,v_3)$?

Antwort: Nur dann, wenn es $\lambda,\mu \in \mathbb{R}$ gibt mit $v_1 = \lambda v_2 + \mu v_3$, also
\begin{alignat*}{8}
\text{I}\quad & 4 & \lambda & \enspace+\enspace & 2 & \mu & \enspace=\enspace & 3 \\
\text{II}\quad &  &         &   & 2 & \mu & = & 5 \\
\text{III}\quad & & \lambda & \enspace+\enspace &   & \mu & \enspace=\enspace & 2
\end{alignat*}

Das ist ein lineares Gleichungssystem (mehr dazu in Kapitel 3). Aus II folgt $\mu = \frac{5}{2}$, in III eingesetzt ergibt sich $\lambda = -\frac{1}{2}$. Einsetzen in I bestätigt unsere Lösung, also liegt $v_1 \in \text{span}(v_2,v_3)$.

\vspace{0.5cm}

Es sei folgende Basis von $\mathbb{R}^3$ gegeben:
\begin{align*}
	v_1 = \begin{pmatrix}1\\1\\1\end{pmatrix}, v_2 = \begin{pmatrix}1\\2\\3\end{pmatrix}, v_3 = \begin{pmatrix}2\\-1\\1\end{pmatrix}
\end{align*}

Wir wollen nun $x = \begin{pmatrix}1\\-2\\5\end{pmatrix}$ durch $v_1,v_2,v_3$ ausdrücken, also $\lambda _1, \lambda _2, \lambda _3 \in \mathbb{R}$ finden, so dass $\lambda _1 v_1 + \lambda _2 v_2 + \lambda _3 v_3 = x$.

\begin{alignat*}{8}
\text{I}\quad & \lambda _1 & \enspace+\enspace & \lambda _2 & \enspace+\enspace & 2 \lambda _3 & \enspace=\enspace & 1 \\
\text{II}\quad & \lambda _1 & \enspace+\enspace & 2 \lambda _2 & \enspace-\enspace & \lambda _3 & \enspace=\enspace & -2 \\
\text{III}\quad & \lambda _1 & \enspace+\enspace & 3 \lambda _2 & \enspace+\enspace & \lambda _3 & \enspace=\enspace & 5
\end{alignat*}

Es ergibt sich $\lambda _1 = -6, \lambda _2 = 3, \lambda _3 = 2$.
\begin{align*}
	x = \begin{pmatrix}1\\-2\\5\end{pmatrix} = (-6)\begin{pmatrix}1\\1\\1\end{pmatrix} + 3 \begin{pmatrix}1\\2\\3\end{pmatrix} + 2 \begin{pmatrix}2\\-1\\1\end{pmatrix}
\end{align*}

\vspace{0.5cm}

2 Vektoren sind genau dann linear unabhängig wenn einer der Vektoren ein Vielfaches des anderen Vektors ist.


\termin{19.04.2016}

\subsection{Rechtwinkliges Dreieck}
\mdf{Definition}
In einem rechtwinkligen Dreieck
\begin{center}
\begin{tikzpicture}[font=\sffamily,scale=1.5]
\coordinate (A) at (0,0);
\coordinate (B) at (4,0);
\coordinate (C) at (4,2);

\pic["$\alpha$", draw=red, thick, -, angle eccentricity=0.7, angle radius=2cm] {angle=B--A--C};
\pic["$\cdot$", draw=red, thick, -, angle eccentricity=0.5, angle radius=0.5cm] {angle=C--B--A};

\draw [-] (A) -- (B) -- (C) -- (A);
\node [right] at (4,1) {$a$};
\node [above] at (2,1) {$b$};
\node [below] at (2,0) {$c$};
\end{tikzpicture}
\end{center}

heißt $b$ \underline{Hypotenuse}, $c$ \underline{Ankathete} zu $\alpha$ und $a$ \underline{Gegenkathete} zu $\alpha$.

Das Verhältnis $\frac{c}{b}$ heißt \underline{Kosinus} von $a$:
\begin{align*}
    \text{cos}(\alpha) &= \frac{\text{Länge der Ankathete}}{\text{Länge der Gegenkathete}}
\end{align*}

Offenbar macht diese Definition vom Kosinus nur Sinn für Winkel $0^{\circ} \leq \alpha \leq 90^{\circ}$.
\begin{center}
\begin{tikzpicture}[font=\sffamily]
\coordinate (A) at (0,0);
\coordinate (B) at (10,0);
\coordinate (C) at (0.1,1.2);
\coordinate (D) at (0.1,-1.2);

\draw [-] (A) -- (B);
\draw [-] (C) -- (D);

\node [left] at (0,0) {$0$};
\draw (0.2,1) -- (0,1) node[left] {$1$};
\draw (0.2,-1) -- (0,-1) node[left] {$-1$};

\draw [blue,thick] plot[variable=\x,domain=0:10,smooth,samples=200] (\x,{cos(\x r)});
\end{tikzpicture}
\end{center}

\subsection{Skalarprodukt}
\mdf{Definition}
Seien $x = \begin{pmatrix}x_1\\\vdots\\x_n\end{pmatrix},\quad y = \begin{pmatrix}y_1\\\vdots\\y_n\end{pmatrix}$ Vektoren aus $\mathbb{R}^n$.

Das \underline{Skalarprodukt} $\langle x, y\rangle$ von $x$ und $y$ ist definiert durch
\begin{align*}
    \langle x, y\rangle &= x_1y_1 + x_2y_2 + \dots + x_ny_n
\end{align*}

Zwei Vektoren $x, y \in \mathbb{R}^n$ heißen \underline{orthogonal} oder \underline{senkrecht} zueinander, falls $\langle x, y\rangle = 0$.

\subsection{Norm}
\mdf{Definition}
Die \underline{Norm} eines Vektors $x \in \mathbb{R}^n$ ist definiert durch $\norm{x} := \sqrt{\langle x, x\rangle}$.

\subsection{Winkel zwischen Vektoren}
\mdf{Definition}
Der Winkel $\varphi$ (Phi) zwischen den Vektoren $x, y \in \mathbb{R}^n$ mit $x\neq 0, y\neq 0$ ist festgelegt durch
\begin{align*}
    \text{cos}(\varphi) &= \frac{\langle x, y\rangle}{\norm{x}\cdot\norm{y}}
\end{align*}

\begin{center}
\begin{tikzpicture}[font=\sffamily]
\draw [->] (0,0) -- (-1,1);
\draw [->,thick,red] (0,0) -- (2,3);
\draw [->] (0,0) -- (1,1.5);

\node [below left] at (-0.5,0.5) {$x$};
\node [below right] at (0.5,0.75) {$y$};
\node [below right,red] at (1.5,2.5) {$y_1 = \lambda y$};
\end{tikzpicture}
\end{center}
Die Vektoren müssen normiert werden, weil der Winkel sich sonst durch verschieden lange Vektoren verändern würde.

\mdf{Beispiel}
Im $\mathbb{R}^2$: $v_1 = \begin{pmatrix}1\\0\end{pmatrix},\quad v_2 = \begin{pmatrix}x\\y\end{pmatrix}$ mit $x, y \geq 0$.

\begin{center}
\begin{tikzpicture}[>=triangle 45,font=\sffamily,scale=1.5,decoration=brace]
\coordinate (A) at (0,0);
\coordinate (B) at (1,0);
\coordinate (C) at (2.5,2.2);

\pic["$\varphi$", draw=red, thick, -, angle eccentricity=0.7, angle radius=1.2cm] {angle=B--A--C};
	
\draw[->] (-0.5,0) -- (3.5,0) node[anchor=north west] {$x$};
\draw[->] (0,-0.5) -- (0,3.5) node[anchor=south east] {$y$};
\foreach \x in {1,2,3}
	\draw (\x cm,4pt) -- (\x cm,-4pt) node[anchor=north] {$\x$};
\foreach \y in {1,2,3}
	\draw (4pt,\y cm) -- (-4pt,\y cm) node[anchor=east] {$\y$};

\draw [thick,->] (0,0) -- (B);
\node [below] at (0.5,0) {$v_1$};

\draw [thick,->] (0,0) -- (C);
\node [above left] at (1.25,1.1) {$v_2$};

\draw [decorate, yshift=0.3cm] (0,2) -- node[above] {$x$} (2.5,2);
\draw [dashed] (0,2.2) -- (2.5,2.2);

\draw [rotate=-90, decorate, yshift=0.3cm] (-2.2,2.3) -- node[right] {$y$} (0,2.3);
\draw [dashed] (2.5,0) -- (2.5,2.2);
\end{tikzpicture}
\end{center}

Nach Definition 13:
\begin{align*}
	\text{cos}(\varphi) &= \frac{x}{\sqrt{x^2 + y^2}}
\end{align*}

Mit Skalarprodukt nach Definition 16:
\begin{align*}
	\frac{\left\langle \begin{pmatrix}1\\0\end{pmatrix}, \begin{pmatrix}x\\y\end{pmatrix} \right\rangle}{\norm{\begin{pmatrix}1\\0\end{pmatrix}}\cdot \norm{\begin{pmatrix}x\\y\end{pmatrix}}} &= \frac{1x + 0y}{\sqrt{1^2 + 0^2}\cdot\sqrt{x^2 + y^2}} = \frac{x}{\sqrt{x^2 + y^2}} = \text{cos}(\varphi)
\end{align*}

\subsection{Orthogonale Systeme}
\mdf{Definition}
Eine Menge von Vektoren $v_1,\dots,v_k \in \mathbb{R}^n$ heißt \underline{orthogonales System}, falls für alle $1\leq i, j \leq k, i\neq j$ gilt: $\langle v_i, v_j\rangle = 0$.

Ein orthogonales System $v_1,\dots,v_n \in \mathbb{R}^n$, das zusätzlich Basis von $\mathbb{R}^n$ ist und für alle $1\leq i\leq n \norm{v_i} = 1$ erfüllt, heißt \underline{Orthonormalbasis} von $\mathbb{R}^n$.

\begin{center}
\begin{tikzpicture}[>=triangle 45,font=\sffamily]
% Ja, das hat lange gedauert. Aber dafür ist es nun sehr flexibel :)
\coordinate (v1) at ({2*cos(1.1 r)},{2*sin(1.1 r)});
\coordinate (v2) at ({2*cos((1.1-pi/2) r)},{2*sin((1.1-pi/2) r)});
\coordinate (c) at (0,0);

\pic["$\cdot$", draw=red, thick, -, angle eccentricity=0.6, angle radius=1.2cm] {angle=v2--c--v1};

\draw [step=2cm,gray,very thin] (-2.5,-2.5) grid (2.5,2.5);
\draw [->] (-3,0) -- (3,0) node[anchor=north west] {$x$};
\draw [->] (0,-2.5) -- (0,3) node[anchor=south east] {$y$};

\draw (0,0) circle (2);

\draw [->] (0,0) -- (v1);
\draw [->] (0,0) -- (v2);

\node [below] at (0,-2.5) {Einheitskreis};
\end{tikzpicture}
\end{center}
Das ist auch eine Basis.

\mdf{Satz}
\begin{enumerate}[i)]
  \item{Sei $v \in \mathbb{R}^n$. Dann gilt: $\norm{v} = 0 \Leftrightarrow v = 0$}
  \item{Für alle Vektoren $v, w \in \mathbb{R}^n$ gilt: $\langle v, w\rangle = \langle w, v\rangle$ (\glqq{}symmetrisch\grqq{})}
  \item{$\forall x, y, z \in \mathbb{R}^n$ gilt: $\langle x+y, z\rangle = \langle x, z\rangle + \langle y, z\rangle$}
  \item{$\forall x, y \in \mathbb{R}^n,\enspace \lambda, \mu \in \mathbb{R}$ gilt: $\langle\lambda x, \mu y\rangle = \lambda\mu\langle x, y\rangle$}
  \item{Ist $v_1,\dots,v_k$ ein orthogonales System und alle $v_i \neq 0$, dann sind $v_1,\dots,v_k$ linear unabhängig.}
\end{enumerate}

\textbf{Zu i)}
\begin{align*}
	\norm{v} = \sqrt{\langle v, v\rangle} = \sqrt{v_1^2+v_2^2+\dots +v_n^2} = 0 &\Leftrightarrow v_1^2+v_2^2+\dots +v_n^2 = 0 \\
	&\Leftrightarrow v_1 = v_2 = \dots = v_n = 0 \\
	&\Leftrightarrow v = 0
\end{align*}

\textbf{Zu ii)}
\begin{align*}
	\langle v, w\rangle &= v_1w_1+v_2w_2+\dots +v_nw_n\quad|\text{ Kommutativgesetz} \\
	&= w_1v_1+w_2v_2+\dots +w_nv_n = \langle w, v\rangle
\end{align*}

\textbf{Zu iii) und iv): Siehe Übung H8}

\textbf{Zu v)}
Zu zeigen: $\lambda_1v_1+\lambda_2v_2+\dots +\lambda_kv_k = 0 \Rightarrow \lambda_1 = \lambda_2 = \dots = \lambda_k = 0$

Sei also $\lambda_1v_1+\lambda_2v_2+\dots +\lambda_kv_k = 0$. Bilde das Skalarprodukt
\begin{align*}
	\langle 0, v_1\rangle &= 0 = \langle \lambda_1v_1+\dots +\lambda_kv_k, v_1\rangle \\
	\text{Nach iii)}\quad &= \langle \lambda_1v_1, v_1\rangle +\langle \lambda_2v_2, v_1\rangle+\dots +\langle \lambda_kv_k, v_1\rangle \\
	\text{Nach iv)}\quad &=\lambda_1\langle v_1, v_1\rangle + \lambda_2\langle v_2, v_1\rangle + \dots + \lambda_k\langle v_k, v_1\rangle = \lambda_1\langle v_1, v_1\rangle \\
	&= \lambda_1 \langle v_1, v_1\rangle = \lambda_1 \norm{v_1}^2\quad\text{Da } v_1\neq 0 \Rightarrow \lambda_1 = 0
\end{align*}

Analog für $i=2,\dots,k$.
\begin{align*}
	0 = \langle 0, v_i\rangle &= \langle \lambda_1v_1 + \dots + \lambda_kv_k, v_i\rangle \\
	&= \lambda_1 \langle v_1, v_i\rangle + \dots + \lambda_k \langle v_k, v_i\rangle \\
	&= \lambda_i \langle v_i, v_i\rangle\quad\text{(Da $v_1,\dots,v_k$ orth. System)} \\
	&= \lambda_i \norm{v_i}^2\quad\text{Da }v_i\neq 0\text{, folgt }\lambda_i = 0\text{.}
\end{align*}
Also $\lambda_1 = \lambda_2 = \dots = \lambda_k = 0$. Damit sind $v_1,\dots,v_k$ unabhängig.


\termin{22.04.2016}
\kapitel{Lineare Gleichungssysteme}
\mdf{Definition}
Ein \begr{lineares Gleichungssystem} aus $m$ Gleichungen und $n$ Unbekannten $x_1,\dots,x_n$ hat die Form
\begin{align*}
    a_{11}x_1 + a_{12}x_2 + \dots + a_{1n}x_n &= b_1 \\
    a_{21}x_1 + a_{22}x_2 + \dots + a_{2n}x_n &= b_2 \\
    \vdots & \\
    a_{m1}x_1 + a_{m2}x_2 + \dots + a_{mn}x_n &= b_m
\end{align*}

Dabei sind die $a_{ij}$ und $b_i$ reelle oder komplexe Zahlen. Die $a_{ij}$ heißen \begr{Koeffizienten} des LGS. Sind alle $b_i$ gleich Null, so heißt das LGS \begr{homogen}, sonst \begr{inhomogen}.

Ein LGS zu lösen bedeutet, Zahlen $x_1,\dots,x_n$ zu finden, so dass alle Gleichungen des LGS erfüllt werden.

\mdf{Satz}
Ein inhomogenes LGS hat entweder keine, genau eine oder unendlich viele Lösungen.

\mdf{Beispiel}

\textbf{a)}
\begin{alignat*}{4}
\text{I}\quad & x+y & = 2 & \quad|\,\text{Aus II: } x = y \\
\text{II}\quad & x-y & = 0 & \quad|\,\text{Aus I: } 2x = 2
\end{alignat*}
Also genau eine Lösung: $x = y = 1$. \\

\textbf{b)}
\begin{alignat*}{4}
\text{I}\quad & x+y & = 2 & \quad|\,\text{Aus II: } x = -y \\
\text{II}\quad & x+y & = 0 & \quad|\,-y+y = 0 \Rightarrow 0 = 2 \enspace\lightning
\end{alignat*} \\

\textbf{c)}
\begin{alignat*}{4}
\text{I}\quad & x+y & = 2 & \quad|\,\text{Aus II: } y = 2 - x \\
\text{II}\quad & 2x + 2y & = 4 & \quad|\,2x + 2(2-x) = 4
\end{alignat*}
Also $4 = 4$. Daher kann $x$ oder $y$ frei gewählt werden.
\begin{align*}
    \text{Also }\begin{pmatrix}x\\y\end{pmatrix} = \begin{pmatrix}t\\2-t\end{pmatrix}\text{ mit }t\in \mathbb{R}
\end{align*}

\begin{samepage}
\textbf{Anschauung:} \\
\textbf{a)}
\begin{center}
\begin{tikzpicture}[>=triangle 45,font=\sffamily]
\draw[step=1cm,gray,very thin] (-2.9,-2.9) grid (3.9,3.9);
\draw[thick,->] (-3.5,0) -- (4.5,0) node[anchor=north west] {$x$-Achse};
\draw[thick,->] (0,-3.5) -- (0,4.5) node[anchor=south east] {$y$-Achse};
\foreach \x in {-2,-1,1,2,3,4}
        \draw (\x cm,4pt) -- (\x cm,-4pt) node[anchor=north] {$\x$};
\foreach \y in {-2,-1,1,2,3,4}
        \draw (4pt,\y cm) -- (-4pt,\y cm) node[anchor=east] {$\y$};

\draw[red,thick] (-2.5,-2.5) -- (3.5,3.5) node[anchor=south west] {$x=y$};
\draw[blue,thick] (-2.5,4.5) -- (3.5,-1.5) node[anchor=north west] {$y=2-x$};
\end{tikzpicture}
\end{center}
Die Lösung liegt genau in der Schnittmenge der beiden Geraden, hier also bei $(1, 1)$.
\end{samepage} \\

\begin{samepage}
\textbf{b)}
\begin{center}
\begin{tikzpicture}[>=triangle 45,font=\sffamily]
\draw[step=1cm,gray,very thin] (-2.9,-2.9) grid (3.9,3.9);
\draw[thick,->] (-3.5,0) -- (4.5,0) node[anchor=north west] {$x$-Achse};
\draw[thick,->] (0,-3.5) -- (0,4.5) node[anchor=south east] {$y$-Achse};
\foreach \x in {-2,-1,1,2,3,4}
        \draw (\x cm,4pt) -- (\x cm,-4pt) node[anchor=north] {$\x$};
\foreach \y in {-2,-1,1,2,3,4}
        \draw (4pt,\y cm) -- (-4pt,\y cm) node[anchor=east] {$\y$};

\draw[red,thick] (-2.5,2.5) -- (3.5,-3.5) node[anchor=south west] {$x=-y$};
\draw[blue,thick] (-2.5,4.5) -- (3.5,-1.5) node[anchor=north west] {$y=2-x$};
\end{tikzpicture}
\end{center}
Die Geraden sind hier parallel und schneiden sich nicht, es gibt also keine Lösung des LGS, die Lösungsmenge ist leer.
\end{samepage} \\

\begin{samepage}
\textbf{c)}
\begin{center}
\begin{tikzpicture}[>=triangle 45,font=\sffamily]
\draw[step=1cm,gray,very thin] (-2.9,-2.9) grid (3.9,3.9);
\draw[thick,->] (-3.5,0) -- (4.5,0) node[anchor=north west] {$x$-Achse};
\draw[thick,->] (0,-3.5) -- (0,4.5) node[anchor=south east] {$y$-Achse};
\foreach \x in {-2,-1,1,2,3,4}
        \draw (\x cm,4pt) -- (\x cm,-4pt) node[anchor=north] {$\x$};
\foreach \y in {-2,-1,1,2,3,4}
        \draw (4pt,\y cm) -- (-4pt,\y cm) node[anchor=east] {$\y$};


\draw[blue,thick] (-2.5,4.5) -- (3.5,-1.5) node[anchor=north west] {$y=2-x$};
\draw[red,thick,dashed] (-2.5,4.5) -- (3.5,-1.5) node[anchor=south west] {$x=4-2y$};
\end{tikzpicture}
\end{center}
Die Geraden liegen aufeinander, es gibt unendlich viele Lösungen.
\end{samepage}

\mdf{Bemerkung}
Ein homogenes LGS hat immer mindestens eine Lösung: $x_1 = \dots = x_n = 0$ (triviale Lösung)

\subsection{Erlaubte Umformungen eines LGS}
\mdf{Satz}
Folgende Umformungen verändern die Lösung eines LGS nicht:

\begin{description}
\item[i)]{Vertauschung zweier Zeilen}
\item[ii)]{Multiplikation einer Zeile mit einer Zahl ungleich Null}
\item[iii)]{Addition des Vielfachen einer Zeile zu einer anderen Zeile}
\end{description}

\subsection{Erweiterte Koeffizientenmatrix}
\mdf{Bemerkung}
Statt ein LGS als System von Gleichungen darzustellen nutzt man oft die \begr{erweiterte Koeffizientenmatrix}:
\begin{align*}
    \left(\begin{array}{cccc|c}
        a_{11} & a_{12} & \dots & a_{1n} & b_1 \\
        a_{21} & a_{22} & \dots & a_{2n} & b_2 \\
        \vdots & \vdots & \ddots & \vdots & \vdots \\
        a_{m1} & a_{m2} & \dots & a_{mn} & b_m
    \end{array}\right)
\end{align*}

\mdf{Definition}
Eine erweiterte Koeffizientenmatrix ist in \begr{Zeilenstufenform} (ZSF), falls

\begin{description}
\item[i)]{In jeder Zeile ist die erste Zahl ungleich Null eine Eins und steht weiter rechts als die erste nicht-Null-Zahl der Zeile darüber}
\item[ii)]{Alle Nullzeilen befinden sich am unteren Ende der Matrix}
\end{description}
Sie ist in reduzierter Zeilenstufenform, falls zusätzlich gilt:
\begin{description}
\item[iii)]{Über jeder führenden Eins stehen nur Nullen}
\end{description}

\begin{align*}
    \left(\begin{array}{ccccc|c}
        0 & 1 & 1 & 3 & 5 & 1 \\
        0 & 0 & 1 & 1 & 0 & 0 \\
        0 & 0 & 0 & 0 & 1 & 3
    \end{array}\right) \text{ ist in ZSF}
\end{align*}

\begin{align*}
    \left(\begin{array}{ccccc|c}
        1 & 0 & 0 & 0 & 0 & 2 \\
        0 & 0 & 0 & 1 & 0 & 3 \\
        0 & 0 & 0 & 0 & 1 & 1 \\
        0 & 0 & 0 & 0 & 0 & 0
    \end{array}\right) \text{ ist in red. ZSF}
\end{align*}

Wenn eine erweiterte Koeffizientenmatrix in ZSF oder red. ZSF ist (und damit auch das zugehörige LGS), lässt sich die Lösung des LGS leicht ablesen.

Der folgende Algorithmus überführt eine erweiterte Koeffizientenmatrix in ZSF:

\mdf{Satz}
Jede Matrix kann durch endlich viele Umformungen aus Satz 5 in ZSF gebracht werden.

\subsection{Gauß--Jordan--Algorithmus}
\mdf{Algorithmus}

Anmerkung: \texttt{a[x][y]} bezeichnet hier $a_{xy}$.
\begin{verbatim}
1   i = 1
2   j = 1
3   Gauß(i, j):
4       Falls i = m oder j = n+1:
5           Ende
6       Falls a[i][j] = 0:
7           Suche r > i mit a[r][j] != 0
8           Falls r existiert:
9               Tausche Zeilen r und i
10          Sonst:
11              Gauß(i, j+1)
12      Teile i-te Zeile durch a[i][j]
13      Für alle k > i: (Zeile k) - a[k][j] * (Zeile i)
14      Gauß(i+1, j+1)
\end{verbatim}

\mdf{Bemerkung}
Beim Lösen von Hand ist es oft einfacher, notwendige Umformungen durch \glqq{}scharfes Hingucken\grqq{} zu erkennen.

Beim Implementieren des Gauß--Jordan--Algorithmus treten oft Effekte auf, die mit der Repräsentation reeller Zahlen im Rechner zusammenhängen.

\mdf{Beispiel}
\begin{align*}
    \left(\begin{array}{ccccc|c}
        1 & 2 & 1 & 1 & 1 & 2 \\
        -1 & -2 & -2 & 2 & 1 & 1 \\
        2 & 4 & 3 & -1 & 0 & -1 \\
        1 & 2 & 2 & -2 & 1 & 1
    \end{array}\right)
\end{align*}
Löse das zugehörige LGS. Es gibt keine Lösung.


\termin{26.04.2016}

\kapitel{Matrizen}

\mdf{Definition}
Ein rechteckiges Schema von Skalaren $a_{ij} \in \mathbb{R}\text{ (oder }\mathbb{C}\text{)}$ in $m$ Zeilen und $n$ Spalten der Form
\begin{align*}
    A &=
    \begin{pmatrix}
        a_{11} & a_{12} & \dots & a_{1n} \\
        a_{21} & a_{22} & \dots & a_{2n} \\
        \vdots & \vdots & \ddots & \vdots \\
        a_{m1} & a_{m2} & \dots & a_{mn}
    \end{pmatrix}
\end{align*}
kurz $A = (a_{ij})$ heißt \begr[Matrix]{$m \times n$ Matrix}. $(m, n)$ heißt \begr{Dimension} der Matrix, $a_{ij}$ heißen \begr[Einträge (Matrix)]{Einträge}/\begr[Koeffizienten (Matrix)]{Koeffizienten}. Ist $A$ eine $n \times n$ Matrix (quadratische Matrix), so heißen die Koeffizienten $a_{ii}$ \begr{Diagonalelemente} von $A$.

Mit Matrizen kann man rechnen. Matrizen kann man addieren. Sind $A, B$ $m \times n$ Matrizen, man schreibt auch $A, B \in \mathbb{R}^{m \times n}$, mit $A = (a_{ij}), B = (b_{ij})$, dann
\begin{align*}
    A + B &= (a_{ij} + b_{ij}) =
    \begin{pmatrix}
        a_{11} + b_{11} & a_{12} + b_{12} & \dots & a_{1n} + b_{1n} \\
        a_{21} + b_{21} & a_{22} + b_{22} & \dots & a_{2n} + b_{2n} \\
        \vdots & \vdots & \ddots & \vdots \\
        a_{m1} + b_{m1} & a_{m2} + b_{m2} & \dots & a_{mn} + b_{mn}
    \end{pmatrix}
\end{align*}

\mdf{Lemma}
Bezüglich \glqq{}$+$\grqq{} bilden $m \times n$ Matrizen eine abelsche Gruppe, d.h. es gilt:
\begin{align*}
    A + B &= B + A \\
    A + (B + C) &= (A + B) + C
\end{align*}
Die Nullmatrix $0 = (0)\enspace i = 1,\dots,m\enspace j = 1,\dots,n$ ist neutrales Element der Addition.

Ist $A = (a_{ij})$, so gibt es ein eindeutig bestimmtes $-A = (-a_{ij})$ mit $A + (-A) = 0$. Beweis: Siehe Übungsblatt 5.

Matrizen sind ähnlich wie Vektoren. Für $k, h \in \mathbb{R}$ (oder $\mathbb{C}$):
\begin{itemize}
    \item{$kA = (k a_{ij})$}
    \item{$k(hA) = (kh)A$}
    \item{$k(A+B) = kA + kB$}
    \item{$(k+h)A = kA + hA$}
    \item{$1A = A$}
\end{itemize}

\mdf{Definition}
Zu einer $m \times n$ Matrix $A = (a_{ij})$ heißt die $n \times m$ Matrix $A^T = (a_{ji})$ \begr[Transponierte Matrix]{transponierte Matrix} zu $A$.

\mdf{Beispiel}
\begin{align*}
    A &= \begin{pmatrix}
        3 & 2 \\
        5 & -1 \\
        0 & 7
    \end{pmatrix} & A^T = \begin{pmatrix}
        3 & 5 & 0 \\
        2 & -1 & 7
    \end{pmatrix} \\
    B &= \begin{pmatrix}
        1 \\
        2 \\
        5
    \end{pmatrix} & B^T = \begin{pmatrix}
        1 & 2 & 5
    \end{pmatrix}
\end{align*}

\mdf{Definition}
Gilt für eine $n \times n$ Matrix $A$:
\begin{align*}
    A &= A^T
\end{align*}
so heißt $A$ \begr[Symmetrische Matrix]{symmetrische Matrix}.

\mdf{Definition}
Sei $A \in \mathbb{R}^{m \times n},\enspace B \in \mathbb{R}^{n \times r}$. Das Produkt $A \cdot B$ ist die $m \times r$ Matrix $C$ mit
\begin{align*}
    c_{ij} &= \sum_{k=1}^{n} a_{ik} b_{kj} = a_{i1} b_{1j} + a_{i2} b_{2j} + \dots + a_{in} b_{nj}
\end{align*}
Das Produkt von $A$ und $B$ ist nur definiert, falls $A$ so viele Spalten besitzt, wie $B$ Zeilen hat.

\mdf{Beispiel}
\begin{align*}
    A &= \begin{pmatrix}
        4 & 2 & 0 \\
        -1 & 3 & 5
    \end{pmatrix} \\
    B &= \begin{pmatrix}
        2 & 1 \\
        3 & 7 \\
        1 & 0
    \end{pmatrix} \\
    C &= \begin{pmatrix}
        7 & 1 \\
        2 & 5
    \end{pmatrix} \\
\end{align*}
Definiert sind nur $A\cdot B$, $B\cdot A$, $B\cdot C$ und $C\cdot A$. Nicht definiert sind hingegen $A\cdot C$ und $C\cdot B$.

Das \begr[Falk Schema]{Falk(sche) Schema} kann verwendet werden, um Matrizen einfach zu multiplizieren, hier am Beispiel $A\cdot B$. Man schreibt $A$ in die untere linke Zelle und $B$ in die obere rechte Zelle. Das Ergebnis $C$ steht in der unteren rechten Zelle.
\begin{center}
\begin{tabular}{r|l}
    & $\begin{pmatrix}2 & 1\\3 & 7\\1 & 0\end{pmatrix}$ \\ \hline
    $\begin{pmatrix}4 & 2 & 0\\-1 & 3 & 5\end{pmatrix}$ & $\begin{pmatrix} 14 & 18 \\ 12 & 20 \end{pmatrix}$
\end{tabular}
\end{center}
(Wird noch überarbeitet)

\mdf{Satz}
Seien $A, B, C$ Matrizen mit passender Dimension (passend für die Beispiele), $k \in \mathbb{R}$ (oder $\mathbb{C}$). Dann gelten:
\begin{itemize}
    \item{$(kA)B = k(AB)$}
    \item{$A(BC) = (AB)C$}
    \item{$(A+B)C = AC+BC$}
    \item{$A(B+C) = AB+AC$}
    \item{$(AB)^T = B^TA^T$}
\end{itemize}
Im Allgemeinen gilt $AB \neq BA$.

\mdf{Definition}
Die $n \times n$ Matrix
\begin{align*}
    \mathbbm{1} &= \begin{pmatrix}
        1 & 0 & \dots & 0 \\
        0 & 1 & \dots & 0 \\
        \vdots & \vdots & \ddots & \vdots \\
        0 & 0 & \dots & 1
    \end{pmatrix}
\end{align*}
heißt \begr[Einheitsmatrix]{$n \times n$ Einheitsmatrix}. Ist $A \in \mathbb{R}^{m \times n}$, so gilt $\mathbbm{1}_m A = A \mathbbm{1}_n = A$.

Lineare Gleichungssysteme kann man als Matrixgleichung schreiben. Statt
\begin{align*}
    a_{11}x_1 + a_{12}x_2 + \dots + a_{1n}x_n &= b_1 \\
    a_{21}x_1 + a_{22}x_2 + \dots + a_{2n}x_n &= b_2 \\
    \vdots &= \vdots \\
    a_{m1}x_1 + a_{m2}x_2 + \dots + a_{mn}x_n &= b_m
\end{align*}
schreiben wir
\begin{align*}
    A x = b\text{, mit } A \in \mathbb{R}^{m \times n},\quad x \in \mathbb{R}^{n(\times 1)},\quad b \in \mathbb{R}^{m (\times 1)}
\end{align*}



\termin{29.04.2016}
\subsection{Invertieren}
\mdf{Definition}
Ist $A \in \mathbb{R}^{n \times n}$ eine quadratische Matrix und gibt es zu $A$ eine Matrix $A^{-1}$ mit
\begin{align*}
	A^{-1} A = A A^{-1} = \mathbbm{1}_n
\end{align*}
so heißt $A$ \begr[Invertierbarkeit (Matrix)]{invertierbar} und die Matrix $A^{-1}$ heißt \begr[Inverse Matrix]{inverse Matrix} zu $A$.

\vspace{1cm}

Ist ein LGS mit $n$ Gleichungen und $n$ Unbekannten gegeben
\begin{align*}
	Ax = b\text{ mit }A \in \mathbb{R}^{n \times n},\quad x, b \in \mathbb{R}^n
\end{align*}
und A ist invertierbar, dann ist $x = A^{-1}b$ eindeutige Lösung des LGS.
\begin{align*}
	A^{-1}Ax = \mathbbm{1}_nx = x = A^{-1}b
\end{align*}

\mdf{Satz}
Sind $A, B \in \mathbb{R}^{n \times n}$ invertierbar, so ist auch $A \cdot B$ invertierbar und es gilt
\begin{align*}
	(AB)^{-1} &= B^{-1}A^{-1} \\
	\text{Denn: } (B^{-1}A^{-1})(AB) &= B^{-1}(A^{-1}A)B \\
	&= B^{-1} (\mathbbm{1}_n) B \\
	&= B^{-1}B \\
	&= \mathbbm{1}_n
\end{align*}

\mdf{Beispiel}
Ist $A = \begin{pmatrix}2&4\\-1&3\end{pmatrix}$ invertierbar? Gesucht ist also eine Matrix $B = \begin{pmatrix}b_{11}&b_{12}\\b_{21}&b_{22}\end{pmatrix}$ mit
\begin{align*}
	AB = \mathbbm{1}_2 &= \begin{pmatrix}1&0\\0&1\end{pmatrix} \\
	\begin{pmatrix}2&4\\-1&3\end{pmatrix}\begin{pmatrix}b_{11}&b_{12}\\b_{21}&b_{22}\end{pmatrix} &= \begin{pmatrix}1&0\\0&1\end{pmatrix} \\
	AB = \begin{pmatrix}2b_{11}+4b_{21}&2b_{12}+4b_{22}\\-b_{11}+3b_{21}&-b_{12}+3b_{22}\end{pmatrix} &= \begin{pmatrix}1&0\\0&1\end{pmatrix}
\end{align*}
Als LGS mit $4$ Gleichungen und $4$ Unbekannten
\begin{alignat*}{9}
2b_{11} & \enspace\enspace & & \enspace+\enspace & 4b_{12} & & & \enspace=\enspace & 1 \\
-b_{11} & & & \enspace+\enspace & 3b_{12} & & & \enspace=\enspace & 0 \\
 & \enspace\enspace & 2b_{12} & & & \enspace+\enspace & 4b_{22} & \enspace=\enspace & 0 \\
 & \enspace\enspace & -b_{12} & & & \enspace+\enspace & 3b_{22} & \enspace=\enspace & 1
\end{alignat*}
\begin{align*}
	\begin{pmatrix}
		2 & 0 & 4 & 0 \\
		-1 & 0 & 3 & 0 \\
		0 & 2 & 0 & 4 \\
		0 & -1 & 0 & 3
	\end{pmatrix} \begin{pmatrix}
		b_{11} \\
		b_{12} \\
		b_{21} \\
		b_{22}
	\end{pmatrix} &= \begin{pmatrix}
		1 \\
		0 \\
		0 \\
		1
	\end{pmatrix}
\end{align*}
Als $2$ LGS mit je $2$ Unbekannten
\begin{align*}
	\begin{pmatrix}2&4\\-1&3\end{pmatrix}\begin{pmatrix}b_{11}\\b_{21}\end{pmatrix} = \begin{pmatrix}1\\0\end{pmatrix}\text{ und }
	\begin{pmatrix}2&4\\-1&3\end{pmatrix}\begin{pmatrix}b_{12}\\b_{22}\end{pmatrix} = \begin{pmatrix}0\\1\end{pmatrix}
\end{align*}
Als erweiterte erweiterte (!) Koeffizientenmatrix
\begin{align*}
	\left(\begin{array}{cc|c|c}
		2 & 4 & 1 & 0 \\
		-1 & 3 & 0 & 1
	\end{array}\right) = \left(\begin{array}{c|c}A & \mathbbm{1}_2\end{array}\right)
\end{align*}
Mit Gauß--Jordan auf reduzierte ZSF bringen
\begin{align*}
	\left(\begin{array}{c|c}
		\begin{matrix}1 & 0\\0 & 1\end{matrix} & A^{-1}
	\end{array}\right) &= \left(\begin{array}{c|c}
		\mathbbm{1}_2 & A^{-1}
	\end{array}\right) \\
	\left(\begin{array}{cc|cc}
		2 & 4 & 1 & 0 \\
		-1 & 3 & 0 & 1
	\end{array}\right)\enspace &|\text{I} : 2 \\
	\left(\begin{array}{cc|cc}
		1 & 2 & \frac{1}{2} & 0 \\[0.3em]
		-1 & 3 & 0 & 1
	\end{array}\right)\enspace &|\text{II} + \text{I} \\
	\left(\begin{array}{cc|cc}
		1 & 2 & \frac{1}{2} & 0 \\[0.3em]
		0 & 5 & \frac{1}{2} & 1
	\end{array}\right)\enspace &|\text{II} : 5 \\
	\left(\begin{array}{cc|cc}
		1 & 2 & \frac{1}{2} & 0 \\[0.3em]
		0 & 1 & \frac{1}{10} & \frac{1}{5}
	\end{array}\right)\enspace &|\text{I} - 2\cdot\text{II} \\
	\left(\begin{array}{cc|cc}
		1 & 0 & \frac{3}{10} & -\frac{2}{5} \\[0.3em]
		0 & 1 & \frac{1}{10} & \frac{1}{5}
	\end{array}\right)
\end{align*}
Also $A^{-1} = \frac{1}{10}\begin{pmatrix}3 & -4 \\1 & 2\end{pmatrix}$.
\begin{align*}
	A^{-1}A = \frac{1}{10}\begin{pmatrix}3 & -4 \\1 & 2\end{pmatrix}\begin{pmatrix}2&4\\-1&3\end{pmatrix} = \frac{1}{10}\begin{pmatrix}10&0\\0&10\end{pmatrix} = \begin{pmatrix}1&0\\0&1\end{pmatrix}
\end{align*}

Also: Matrix invertieren durch simultanes Lösung von $n$ LGS. $\left(A|\mathbbm{1}_n\right)$ mit Gauß--Jordan--Algorithmus umformen zu $\left(\mathbbm{1}_n|A^{-1}\right)$. Ist dies möglich, so ist $A$ invertierbar.

\vspace{0.5cm}

\textbf{Frage:} Können wir einer Matrix \glqq{}ansehen\grqq{}, ob sie invertierbar ist?

Betrachte $2 \times 2$ LGS:
\begin{align*}
	a_{11}x_1 + a_{12}x_2 &= b_1\quad |\cdot a_{22} \\
	a_{21}x_1 + a_{22}x_2 &= b_2\quad |\cdot (-a_{12}) \\[0.5cm]
	a_{11}a_{22}x_1 + a_{12}a_{22}x_2 &= a_{22}b_1 \\
	-a_{12}a_{21}x_1 - a_{12}a_{22}x_2 &= -a_{12}b_2 \quad |\text{I} + \text{II} \\[0.5cm]
	a_{11}a_{22}x_1 + a_{12}a_{22}x_2 - a_{12}a_{22}x_2 - a_{12}a_{21}x_1 &= a_{22}b_1 - a_{12}b_2 \\
	\Leftrightarrow (a_{11}a_{22}-a_{12}a_{21}) x_1 &= a_{22}b_1 - a_{12}b_{2}
\end{align*}
Diese Gleichung enthält nur noch $x_1$. Analog erhält man
\begin{align*}
	(a_{11}a_{22}-a_{12}a_{21}) x_2 &= a_{11}b_1 - a_{21}b_{2}
\end{align*}

Ist $(a_{11}a_{22}-a_{12}a_{21}) \neq 0$, ist das LGS eindeutig lösbar.

\mdf{Definition}
Für eine $2 \times 2$ Matrix
\begin{align*}
    A &= \begin{pmatrix}a_{11} & a_{12} \\ a_{21} & a_{22}\end{pmatrix}
\end{align*}
heißt
\begin{align*}
    \text{det}(A) = \begin{vmatrix}a_{11} & a_{12} \\ a_{21} & a_{22}\end{vmatrix}
\end{align*}
\begr{Determinante} von $A$. Die Verallgemeinerung auf allgemeine Matrizen ist schwieriger. Die Herleitung geht über diese Vorlesung hinaus.


\termin{03.05.2016}
\subsection{Laplacescher Entwicklungssatz}
\mdf{Definition}
\begr{Laplacescher Entwicklungssatz}

Für $A = (a_{ij}) \in \mathbb{R}^{n \times n}$ und $i, j \in \{1,\dots,n\}$ gilt
\begin{align*}
	\text{det}(A) &= \sum_{j=1}^{n}(-1)^{i+j} \cdot a_{ij} \cdot \text{det}(A_{ij})
\end{align*}
(Entwicklung nach der $i$-ten Zeile)

bzw.
\begin{align*}
	\text{det}(A) &= \sum_{i=1}^{n}(-1)^{i+j} \cdot a_{ij} \cdot \text{det}(A_{ij})
\end{align*}
(Entwicklung nach der $j$-ten Spalte)

Wobei $A_{ij}$ die Matrix ist, die man erhält, wenn man in $A$ die $i$-te Zeile und die $j$-te Zeile streicht.

\mdf{Beispiel}
\begin{align*}
	A &= \begin{pmatrix}
		4 & 2 & -3 & 4 \\
		5 & 6 & 1 & 4 \\
		0 & 0 & 2 & 0 \\
		-2 & -2 &3 & 6
	\end{pmatrix}
\end{align*}
Entwicklung nach der 3. Zeile (da diese viele Nullen enthält)
\begin{align*}
	\text{det}(A) =\,&(-1)^{3+1} \cdot 0 \cdot \text{det}(A_{31}) \\
	+\,&(-1)^{3+2} \cdot 0 \cdot \text{det}(A_{32}) \\
	+\,&(-1)^{3+3} \cdot 2 \cdot \text{det}(A_{33}) \\
	+\,&(-1)^{3+4} \cdot 0 \cdot \text{det}(A_{34}) \\
	=\,&(-1)^{3+3} \cdot 2 \cdot \text{det}\left(\begin{pmatrix}
		4 & 2 & 4 \\
		5 & 6 & 4 \\
		-2 & -2 & 6
	\end{pmatrix}\right) \\
	=\,&2\cdot\left((-1)^{1+1} \cdot 4 \cdot \begin{vmatrix}
		6 & 4 \\
		-2 & 6
	\end{vmatrix} + (-1)^{1+2} \cdot 2 \cdot \begin{vmatrix}
		5 & 4 \\
		-2 & 6
	\end{vmatrix} + (-1)^{1+3} \cdot 4 \cdot \begin{vmatrix}
		5 & 6 \\
		-2 & -2
	\end{vmatrix}\right) \\
	=\,&2 \cdot (4 \cdot 44 - 2 \cdot 38 + 4 \cdot 2) = 2 \cdot 108 \\
	=\,&216
\end{align*}

\subsubsection*{\textbf{Bei $3 \times 3$ Matrizen}}
Bei $3 \times 3$ Matrizen kann man die \begr{Regel von Sarrus} anwenden. Herleitung:
\begin{align*}
    \begin{vmatrix}
        a_{11} & a_{12} & a_{13} \\
        a_{21} & a_{22} & a_{23} \\
        a_{31} & a_{32} & a_{33}
    \end{vmatrix} &= (-1)^{1+1} a_{11} \begin{vmatrix}
        a_{22} & a_{23} \\
        a_{32} & a_{33}
    \end{vmatrix} + (-1)^{1+2} a_{12} \begin{vmatrix}
        a_{21} & a_{23} \\
        a_{31} & a_{33}
    \end{vmatrix} + (-1)^{1+3} a_{13} \begin{vmatrix}
        a_{21} & a_{22} \\
        a_{31} & a_{32}
    \end{vmatrix} \\
    &= a_{11}(a_{22}a_{33} - a_{23}a_{32}) - a_{12}(a_{21}a_{33} - a_{23}a_{31}) + a_{13}(a_{21}a_{32} - a_{22}a_{31}) \\
    &= a_{11}a_{22}a_{33} - a_{11}a_{23}a_{32} - a_{12}a_{21}a_{33} + a_{12}a_{23}a_{31} + a_{13}a_{21}a_{32} - a_{13}a_{22}a_{31}
\end{align*}\pagebreak

Diese Formel kann man sich einfach grafisch veranschaulichen. Man schreibt rechts von der Matrix die ersten beiden Spalten noch einmal auf:
\tikzset{node style ge/.style={circle}}
\begin{center}
\begin{tikzpicture}[baseline=(A.center)]
\tikzset{Umrandung/.style = {opacity=.4,line width=2 mm,line cap=round,color=#1}}
\tikzset{Plus/.style      = {above left=4mm,opacity=1,circle,fill=#1!50}}
\tikzset{Minus/.style     = {below left=4mm,opacity=1,circle,fill=#1!50}}
\matrix (A) [matrix of math nodes, nodes = {node style ge},,column sep=0 mm] 
{ a_{11} & a_{12} & a_{13} & \textcolor{red}{a_{11}} & \textcolor{red}{a_{12}} \\
  a_{21} & a_{22} & a_{23} & \textcolor{red}{a_{21}} & \textcolor{red}{a_{22}} \\
  a_{31} & a_{32} & a_{33} & \textcolor{red}{a_{31}} & \textcolor{red}{a_{32}} \\
};

\draw [Umrandung=blue] (A-1-1.north west) node[Plus=blue] {$+$} to (A-3-3.south east);
\draw [Umrandung=blue] (A-1-2.north west) node[Plus=blue] {$+$} to (A-3-4.south east);
\draw [Umrandung=blue] (A-1-3.north west) node[Plus=blue] {$+$} to (A-3-5.south east);

\draw [Umrandung=red] (A-3-1.south west) node[Minus=red] {$-$} to (A-1-3.north east);
\draw [Umrandung=red] (A-3-2.south west) node[Minus=red] {$-$} to (A-1-4.north east);
\draw [Umrandung=red] (A-3-3.south west) node[Minus=red] {$-$} to (A-1-5.north east);
\end{tikzpicture}\footnote{Quelle: Alain Matthes, \url{http://www.texample.net/tikz/examples/mnemonic-rule-for-matrix-determinant/}, angepasst.}
\end{center}
Für jede blaue Diagonale bildet man das Produkt der Elemente und summiert dann die Diagonalen auf. Das tut man auch für die roten Diagonalen und subtrahiert diese dann von der Summe der blauen Diagonalen. Das Ergebnis ist die Determinante.

\mdf{Satz}
Sind $A, B \in \mathbb{R}^{n \times n}$ und $a_1,\dots,a_n$ die Zeilen von A, sowie $k \in \mathbb{R}$. Dann gelten:
\begin{itemize}
	\item{$\text{det}(A\cdot B) = \text{det}(A) \cdot \text{det}(B)$}
	\item{$\text{det}\begin{pmatrix}a_1\\\vdots\\a_j\cdot k\\\vdots\\a_n\end{pmatrix} = k \cdot \text{det}(A)$}
	\item{$\text{det}(A^T) = \text{det}(A)$}
	\item{Ist $A$ invertierbar, so gilt $\text{det}(A^{-1}) = \frac{1}{\text{det}(A)}$}
	\item{Vertauscht man in $A$ zwei Zeilen, so wechselt das Vorzeichen der Determinante.}
	\item{\begin{align*}
		&\text{det}\left(\begin{pmatrix}
			a_{11} & \dots & a_{1n} \\
			\vdots & \ddots & \vdots \\
			a_{j1}+b_1 & \dots & a_{jn} + b_n \\
			\vdots & \ddots & \vdots \\
			a_{n1} & \dots & a_{nn}
		\end{pmatrix}\right) =\\
		&\text{det}\left(\begin{pmatrix}
			a_{11} & \dots & a_{1n} \\
			\vdots & \ddots & \vdots \\
			a_{j1} & \dots & a_{jn} \\
			\vdots & \ddots & \vdots \\
			a_{n1} & \dots & a_{nn}
		\end{pmatrix}\right) + \text{det}\left(\begin{pmatrix}
			a_{11} & \dots & a_{1n} \\
			\vdots & \ddots & \vdots \\
			a_{(j-1)1} & \dots & a_{(j-1)n} \\
			b_1 & \dots & b_n \\
			a_{(j+1)1} & \dots & a_{(j+1)n} \\
			\vdots & \ddots & \vdots \\
			a_{n1} & \dots & a_{nn}
		\end{pmatrix}\right)
	\end{align*}}
	\item{$A$ ist genau dann invertierbar, wenn $\text{det}(A) \neq 0$.}
	\item{Sind die Zeilen von $A$ linear abhängig, so ist $\text{det}(A) = 0$.}
\end{itemize}

\newpage
\kapitel{Lineare Abbildungen}
\mdf{Definition}
Seien $U \subseteq \mathbb{R}^n, V \subseteq \mathbb{R}^m$ Unterräume. Die Abbildung $\varphi : U \rightarrow V$ heißt \begr[Lineare Abbildung]{linear}, falls gelten:
\begin{description}
	\item[(A1)]{$\forall u, u' \in U$ gilt $\varphi(u+u') = \varphi(u) + \varphi(u')$}
	\item[(A2)]{$\forall u \in U, \lambda \in \mathbb{R}$ gilt $\varphi(\lambda u) = \lambda \cdot \varphi(u)$}
\end{description}

\mdf{Beispiel}
\begin{align*}
	\varphi :\,\,&\mathbb{R}^2 \rightarrow \mathbb{R}^2 \\
	\begin{pmatrix}x\\y\end{pmatrix} &\mapsto \begin{pmatrix}2x+y\\3x\end{pmatrix}
\end{align*}
ist eine lineare Abbildung. Beweis:
\begin{description}
	\item[(A1)]{Zu zeigen:
	\begin{align*}
		\varphi\left(\begin{pmatrix}x_1\\y_1\end{pmatrix} + \begin{pmatrix}x_2\\y_2\end{pmatrix}\right) &= \varphi\left(\begin{pmatrix}x_1\\y_1\end{pmatrix}\right) + \varphi\left(\begin{pmatrix}x_2\\y_2\end{pmatrix}\right) \\[0.5cm]
		\varphi\left(\begin{pmatrix}x_1+x_2\\y_1+y_2\end{pmatrix}\right) &= \begin{pmatrix}2 (x_1+x_2) + (y_1+y_2)\\3(x_1+x_2)\end{pmatrix} = \begin{pmatrix}2 x_1 + y_1 + 2 x_2 + y_2\\3 x_1 + 3 x_2\end{pmatrix} \\
		&= \begin{pmatrix}2 x_1 + y_1 \\ 3 x_1\end{pmatrix} + \begin{pmatrix}2 x_2 + y_2 \\ 3 x_2\end{pmatrix} = \varphi\left(\begin{pmatrix}x_1\\y_1\end{pmatrix}\right) + \varphi\left(\begin{pmatrix}x_2\\y_2\end{pmatrix}\right)
	\end{align*}
	}
	\item[(A2)]{Zu zeigen:
	\begin{align*}
		\varphi\left(\lambda\begin{pmatrix}x\\y\end{pmatrix}\right) &= \lambda \varphi\left(\begin{pmatrix}x\\y\end{pmatrix}\right) \\[0.5cm]
		\varphi\left(\lambda\begin{pmatrix}x\\y\end{pmatrix}\right) = \varphi\left(\begin{pmatrix}\lambda x\\\lambda y\end{pmatrix}\right) &= \begin{pmatrix}2\lambda x + \lambda y\\3\lambda x\end{pmatrix} = \begin{pmatrix}\lambda(2x+y)\\\lambda(3x)\end{pmatrix} = \lambda \varphi\left(\begin{pmatrix}x\\y\end{pmatrix}\right)
	\end{align*}
	}
\end{description}

\textbf{Gegenbeispiel}
\begin{align*}
	\psi :\,\,&\mathbb{R}^2 \rightarrow \mathbb{R}^2 \\
	\begin{pmatrix}x\\y\end{pmatrix} &\mapsto \begin{pmatrix}x+3\\2x\end{pmatrix}
\end{align*}
ist \textbf{nicht} linear. Denn z.B. $\psi\left(2\begin{pmatrix}1\\0\end{pmatrix}\right) = \psi\left(\begin{pmatrix}2\\0\end{pmatrix}\right) = \begin{pmatrix}5\\4\end{pmatrix}$, aber $2\psi\left(\begin{pmatrix}1\\0\end{pmatrix}\right) = 2 \cdot \begin{pmatrix}4\\2\end{pmatrix} = \begin{pmatrix}8\\4\end{pmatrix} \neq \begin{pmatrix}5\\4\end{pmatrix}$.

\mdf{Satz}
Eine Abbildung $\varphi : \mathbb{R}^n \rightarrow \mathbb{R}^m$ ist genau dann linear, wenn sie in der Form $\varphi(x) = Ax$ mit $A \in \mathbb{R}^{m \times n}$ geschrieben werden kann.

D.h.
\begin{align*}
	\varphi\left(\begin{pmatrix}x_1\\\vdots\\x_n\end{pmatrix}\right) &= \begin{pmatrix}
		a_{11}x_1 + \dots + a_{1n}x_n \\
		\vdots & \ddots & \vdots \\
		a_{m1}x_1 + \dots & a_{mn}x_n
	\end{pmatrix}
\end{align*}
Die Matrix $A$ ist eindeutig bestimmt. Die Spalten von $A$ sind die Bilder der Standardbasisvektoren $e_1,\dots,e_n$. D.h.
\begin{align*}
	A &= (\varphi(e_1),\varphi(e_2),\dots,\varphi(e_n))
\end{align*}

\mdf{Beweis}
\begin{align*}
	\varphi(x + y) &= A(x + y) \\
	&= Ax + Ay \\
	&= \varphi(x) + \varphi(y) \\
	\varphi(\lambda x) &= A(\lambda x) = \lambda(Ax) = \lambda\varphi(x)
\end{align*}
Außerdem:
\begin{align*}
	x &= x_1e_1 + x_2e_2 + \dots + x_ne_n
\end{align*}
Damit:
\begin{align*}
	\varphi(x) &= \varphi(x_1e_1 + \dots + x_ne_n) \\
	&= \varphi(x_1e_1) + \varphi(x_2e_2) + \dots + \varphi(x_ne_n) \\
	&= x_1\varphi(e_1) + x_2\varphi(e_2) + \dots + x_n\varphi(e_n) \\
	&= (\varphi(e_1),\varphi(e_2),\dots,\varphi(e_n))x \\
	&= Ax
\end{align*}

\mdf{Definition}
Die Matrix $A$ aus Satz 3 heißt \begr[Darstellende Matrix]{darstellende Matrix} der linearen Abbildung $\varphi$.

\mdf{Beispiel}
Betrachte Abbildung $\varphi$ aus Beispiel 2. Wie sieht die darstellende Matrix aus?

Standardbasisvektoren einsetzen:
\begin{align*}
	\varphi(e_1) = \varphi\left(\begin{pmatrix}1\\0\end{pmatrix}\right) &= \begin{pmatrix}2\\3\end{pmatrix} \\
	\varphi(e_2) = \varphi\left(\begin{pmatrix}0\\1\end{pmatrix}\right) &= \begin{pmatrix}1\\0\end{pmatrix}
\end{align*}

Also ist $A = \begin{pmatrix}2 & 1 \\ 3 & 0\end{pmatrix}$ die darstellende Matrix von $\varphi$.


\termin{06.05.2016}

An diesem Termin haben wir Fragen zum Stoff vergangener Vorlesungen behandelt. Die daraus neu gewonnenen Erkenntnisse sind an den entsprechenden Stellen im Skript nachgetragen worden.


\termin{10.05.2016}

\mdf{Satz}
Seien $\varphi : \mathbb{R}^n \rightarrow \mathbb{R}^m$ und $\psi : \mathbb{R}^k \rightarrow \mathbb{R}^n$ lineare Abbildungen.
\begin{align*}
	\varphi(y) = Ay \quad \psi(x) = Bx
\end{align*}
Dann ist die \begr{Verkettung} $\varphi \circ \psi : \mathbb{R}^k \rightarrow \mathbb{R}^n$ wieder linear und $(\varphi \cdot \psi)(x) = (AB)x$.

\mdf{Satz}
Eine lineare Abbildung $\varphi : \mathbb{R}^n \rightarrow \mathbb{R}^n$ ist \begr[Umkehrbare lineare Abbildung]{umkehrbar} genau dann wenn die darstellende Matrix $A$ invertierbar ist.

Die Umkehrabbildung ist ebenfalls linear und hat $A^{-1}$ als darstellende Matrix.

\mdf{Definition}
Seien $V \subseteq \mathbb{R}^n, W \subseteq \mathbb{R}^m$ Unterräume und $\varphi : V \rightarrow W$ eine lineare Abbildung. Dann heißen
\begin{align*}
	\text{ker}\varphi := \{x \in V\,|\,\varphi(x) = 0\}
\end{align*}
\begr[Kern einer linearen Abbildung]{Kern} von $\varphi$ und
\begin{align*}
	\text{Bild}\varphi := \varphi(V)
\end{align*}
\begr[Bild einer linearen Abbildung]{Bild} von $\varphi$.

\mdf{Satz}
Sei $\varphi : V \rightarrow W$ eine lineare Abbildung. Dann gilt
\begin{align*}
	\text{dim}\left(\text{ker}(\varphi)\right) + \text{dim}\left(\text{Bild}(\varphi)\right) = \text{dim}(V)
\end{align*}

\mdf{Definition}
Der \begr[Rang einer Matrix]{Rang} $\text{rang}(A)$ einer Matrix $A$ ist die Anzahl der linear unabhängigen Zeilen von $A$.

\mdf{Satz}
Sei $A \in \mathbb{R}^{m \times n}$ dann gilt $\text{rang}(A) = \text{rang}(A^T)$.

\mdf{Satz}
Sei $\varphi : \mathbb{R}^n \rightarrow \mathbb{R}^m$ eine lineare Abbildung mit darstellender Matrix $A$. Dann gilt
\begin{align*}
	\text{dim}\left(\text{Bild}(\varphi)\right) = \text{rang}(A)
\end{align*}

\mdf{Definition}
Sei $\varphi : \mathbb{R}^n \rightarrow \mathbb{R}^n$ eine lineare Abbildung. Ein Vektor $v \in \mathbb{R}^n \setminus \{0\}$ heißt \begr{Eigenvektor} zum \begr{Eigenwert} $\lambda \in \mathbb{R}$ falls gilt: $\varphi(v) = \lambda v$.

Eine reelle Zahl $\lambda$ heißt Eigenwert falls es wenigstens einen Vektor $v \in \mathbb{R}^m \setminus \{0\}$ gibt, der Eigenvektor zu $\lambda$ ist.

\mdf{Beispiel}
Sei $\varphi : \mathbb{R}^2 \rightarrow \mathbb{R}^2$ mit darstellender Matrix $A = \begin{pmatrix} 2 & 0 \\ 0 & 3 \end{pmatrix}$. Dann ist $v = \begin{pmatrix}1\\0\end{pmatrix}$ Eigenvektor von $\varphi$ zum Eigenwert $\lambda = 2$, denn
\begin{align*}
	Av = \begin{pmatrix}2 & 0 \\ 0 & 3\end{pmatrix}\begin{pmatrix}1\\0\end{pmatrix} = \begin{pmatrix}2\\0\end{pmatrix} = 2v
\end{align*}
Außerdem ist $\begin{pmatrix}x\\0\end{pmatrix}$ Eigenvektor (EV) zum Eigenwert (EW) $2 \quad\forall x \in \mathbb{R} \setminus \{0\}$.

Analog: $\begin{pmatrix}0\\y\end{pmatrix}$ ist EV zum Eigenwert $3$ für alle $y \in \mathbb{R} \setminus \{0\}$.

\mdf{Definition}
Eine lineare Abbildung $\varphi : \mathbb{R}^n \rightarrow \mathbb{R}^n$ heißt \begr[Diagonalisierbare Abbildung]{diagonalisierbar} wenn es eine Basis des $\mathbb{R}^n$ aus Eigenvektoren von $\varphi$ gibt.

\mdf{Beispiel}
Die Abbildung aus Beispiel $14$ ist diagonalisierbar da $\begin{pmatrix}1\\0\end{pmatrix}$ und $\begin{pmatrix}0\\1\end{pmatrix}$ EV von $\varphi$ sind.

\mdf{Satz}
Sei $\varphi : \mathbb{R}^n \rightarrow \mathbb{R}^n$ eine lineare Abbildung mit darstellender Matrix $A$. Dann ist $\lambda \in \mathbb{R}$ genau dann EW von $\varphi$, wenn
\begin{align*}
	\text{det}(A - \lambda \cdot \mathbbm{1}_n)x = 0
\end{align*}

\textbf{Beweis:}
\begin{align*}
	&\lambda\text{ ist EW von }\varphi \Leftrightarrow Av = 2v, v \neq 0 \\
	&\Leftrightarrow Av - \lambda v = 0, v \neq 0 \\
	&\Leftrightarrow (A - \lambda \mathbbm{1}_n)v = 0, v \neq 0 \\
	&\Leftrightarrow \text{det}(A - \lambda \mathbbm{1}_n) = 0
\end{align*}
Das heißt $(A - \lambda \mathbbm{1}_n)$ ist nicht invertierbar.

Das Polynom $\chi_A = \text{det}(A - \lambda \mathbbm{1}_n)$ heißt \begr[Charakteristisches Polynom]{charakteristisches Polynom} von $A$ bzw. $\varphi$. (Chi)

\mdf{Beispiel}
Sei $\varphi$ die lineare Abbildung zur darstellenden Matrix $A = \begin{pmatrix}2 & 3 \\ 0 & 3\end{pmatrix}$. Dann ist
\begin{align*}
	\chi_A(A) &= \text{det}\left(\begin{pmatrix}2 & 3 \\ 0 & 3\end{pmatrix} - \lambda\cdot\begin{pmatrix}1 & 0 \\ 0 & 1\end{pmatrix}\right) \\
	&= \text{det}\left(\begin{pmatrix}2-\lambda & 3 \\ 0 & 3-\lambda\end{pmatrix}\right) = (2-\lambda)\cdot(3-\lambda) = \lambda^2 - 5\lambda + 6
\end{align*}
Suche jetzt Nullstellen von $\chi_A(\lambda)$.
\begin{align*}
	\lambda_1 = 2\quad\lambda_2 = 3
\end{align*}
sind Nullstellen von $\chi_A$ und somit EW von $\varphi$.

Um EV zu finden muss man nun das LGS
\begin{align*}
	(A - \lambda\mathbbm{1}_n)x = 0
\end{align*}
lösen für jeden EW $\lambda$ von $\varphi$.

Also für $\lambda_1 = 2$:
\begin{align*}
	&\begin{pmatrix}0 & 3 \\ 0 & 1\end{pmatrix}\begin{pmatrix}x_1 \\ x_2\end{pmatrix} = \begin{pmatrix}0\\0\end{pmatrix} \\
	&\Rightarrow \left(\begin{array}{cc|c}0 & 3 & 0 \\ 0 & 1 & 0\end{array}\right) \\
	&\Downarrow \text{I} - 3 \cdot \text{II} \\
	&\Leftrightarrow \left(\begin{array}{cc|c}0 & 0 & 0 \\ 0 & 1 & 0\end{array}\right)
\end{align*}
$\Rightarrow x_2 = 0, x_1$ beliebig, aber nicht $0$, da $\begin{pmatrix}0\\0\end{pmatrix}$ kein EV. Also $\begin{pmatrix}x_1\\0\end{pmatrix}$ mit $x_1 \neq 0$ EV zum EW $\lambda _1 = 2$.

Für $\lambda_2 = 3$:
\begin{align*}
	&\left(\begin{array}{cc|c}-1 & 3 & 0 \\ 0 & 0 & 0\end{array}\right) \\
	&\Leftrightarrow -1x_1 + 3x_2 = 0 \\
	&\Leftrightarrow 3x_2 = x_1
\end{align*}
Das heißt $x_2 \neq 0$ beliebig wählen. Also ist $\begin{pmatrix}3x_2\\x_2\end{pmatrix}$ mit $x_2 \neq 0$ ein EV zum EW $\lambda _2 = 3$.

\mdf{Definition}
Ist $\lambda$ Eigenwert zur linearen Abbildung $\varphi : \mathbb{R}^n \rightarrow \mathbb{R}^n, x \mapsto Ax$, so heißt die Lösungsmenge des LGS $(A - \lambda \mathbbm{1}_n)x = 0$ \begr{Eigenraum} zum EW $\lambda$ $(\text{Eig}_A(\lambda))$.

Dies ist ein Unterraum des $\mathbb{R}^n$ und jeder Vektor $v \in \text{Eig}_A(\lambda)$ außer $v = 0$ ist EV von $\varphi$ zum EW $\lambda$.






\termin{13.05.2016}

\kapitel{Geometrie Teil 2}
\mdf{Beispiel}

\begin{center}
\begin{tikzpicture}[>=triangle 45,font=\sffamily]
\draw[step=1cm,gray,very thin] (-3.9,-3.9) grid (3.9,3.9);
\draw[thick,->] (-4.5,0) -- (4.5,0) node[anchor=north west] {$x_1$};
\draw[thick,->] (0,-4.5) -- (0,4.5) node[anchor=south east] {$x_2$};
\foreach \x in {-4,-3,-2,-1,1,2,3,4}
        \draw (\x cm,4pt) -- (\x cm,-4pt) node[anchor=north] {$\x$};
\foreach \y in {-4,-3,-2,-1,1,2,3,4}
        \draw (4pt,\y cm) -- (-4pt,\y cm) node[anchor=east] {$\y$};

\draw[black,dashed] (-3.5,-3.5) -- (3.5,3.5) node[anchor=south west] {$\mu \begin{pmatrix}1\\1\end{pmatrix}$};

\draw[thick,->] (0,0) -- (1.2,2.5) node[anchor=south east] {};
\draw[thick,->] (0,0) -- (2.5,1.2) node[anchor=south east] {};

\draw (1.2,2.5) node [cross=4pt,red] {};
\node [align=left] at (1.8,2.5) {$\varphi (x)$};

\draw (2.5,1.2) node [cross=4pt,red] {};
\node [align=left] at (2.8,1.2) {$x$};
\end{tikzpicture}
\end{center}
Es sei $\varphi : \mathbb{R}^2 \rightarrow \mathbb{R}^2$ die lineare Abbildung, die jeden Punkt $x \in \mathbb{R}^2$ an der Geraden
\begin{align*}
    \left\{\mu\begin{pmatrix}1\\1\end{pmatrix}\,|\,\mu \in \mathbb{R}\right\}
\end{align*}
spiegelt.

\begin{enumerate}
    \item{Wie sieht die darstellende Matrix von $\varphi$ aus?}
    \item{Welche Eigenschaften hat sie?}
\end{enumerate}

\begin{align*}
    \varphi (e_1) &= e_2\qquad \varphi (e_2) = e_1 \\
    \Rightarrow A &= \begin{pmatrix} 0 & 1 \\ 1 & 0 \end{pmatrix} = \left(\varphi (e_1), \varphi (e_2)\right)
\end{align*}
Eigenwerte:
\begin{align*}
    \chi_A (\lambda) &= \text{det}\left(\begin{pmatrix} -\lambda & 1 \\ 1 & -\lambda \end{pmatrix}\right) = \lambda ^2 - 1 \\
    \chi_A (\lambda) &= \text{det}(A - \lambda \mathbbm{1}_n) \\
    &\Rightarrow \lambda _1 = 1 \qquad \lambda _2 = -1
\end{align*}
EV:
\begin{align*}
    &\lambda _1 = 1 \\
    &\left(\begin{array}{cc|c}
        -1 & 1 & 0 \\
        1 & -1 & 0
    \end{array}\right) \xRightarrow{\text{II+I}} \left(\begin{array}{cc|c}
        -1 & 1 & 0 \\
        0 & 0 & 0
    \end{array}\right) \\
    &\Rightarrow - x_1 + x_2 = 0 \Leftrightarrow x_1 = x_2 \\
    &\text{Eig}_A(1) = \left\{\begin{pmatrix}x\\x\end{pmatrix}\,|\,x \in \mathbb{R}\right\} \\
    &\text{z.B. Basis: } \left\{\begin{pmatrix}1\\1\end{pmatrix}\right\} \\[0.5cm]
    &\lambda _2 = -1 \\
    &\left(\begin{array}{cc|c}
        1 & 1 & 0 \\
        1 & 1 & 0
    \end{array}\right) \xRightarrow{\text{II-I}} \left(\begin{array}{cc|c}
        1 & 1 & 0 \\
        0 & 0 & 0
    \end{array}\right) \\
    &\Rightarrow x_1 + x_2 = 0 \Leftrightarrow x_1 = -x_2 \\
    &\text{Eig}_A(-1) = \left\{\begin{pmatrix}x\\-x\end{pmatrix}\,|\,x \in \mathbb{R}\right\} \\
    &\text{z.B. Basis: } \left\{\begin{pmatrix}1\\-1\end{pmatrix}\right\} \\[1cm]
    &\Rightarrow \varphi \text{ ist diagonalisierbar, da } \left\{\begin{pmatrix}1\\1\end{pmatrix},\begin{pmatrix}1\\-1\end{pmatrix}\right\} \text{ eine Basis von } \mathbb{R}^2 \text{ darstellt.}
\end{align*}

\mdf{Definition}
\begin{center}
\begin{tikzpicture}[>=triangle 45,font=\sffamily]
\coordinate (v1) at ({2*cos(0.6 r)},{2*sin(0.6 r)});
\coordinate (v2) at ({2*cos(0.6 r)},0);
\coordinate (c) at (0,0);

\pic["$\alpha$", draw=red, thick, -, angle eccentricity=0.6, angle radius=1cm] {angle=v2--c--v1};

\draw [step=2cm,gray,very thin] (-2.5,-2.5) grid (2.5,2.5);
\draw [->] (-3,0) -- (3,0) node[anchor=north west] {$x$};
\draw [->] (0,-2.5) -- (0,3) node[anchor=south east] {$y$};

\draw (0,0) circle (2);

\draw [thick] (0,0) -- (v1);
\draw [thick] (0,0) -- (v2);
\draw [thick] (v2) -- (v1);

\node [below] at (0,-2.5) {Einheitskreis};

\node [below] at (0.9,0) {$x$};
\node [right] at (2,0.5) {$y$};
\end{tikzpicture}
\end{center}
Für ein rechtwinkliges Dreieck im Einheitskreis definiert man für den Winkel $\alpha \in [0^{\circ}, 360^{\circ}]$ bzw. $\alpha \in [0, 2\pi]$
\begin{align*}
    \sina &= y \\
    \cosa &= x
\end{align*}
den \begr{Sinus} und den \begr{Kosinus} von $\alpha$.

Beachte: $x, y$ sind vorzeichenbehaftet, d.h. z.B. für $90^{\circ} < \alpha < 270^{\circ}$ bzw. $\frac{\pi}{2} < \alpha < \frac{3\pi}{2}$ ist $x < 0$.

\mdf{Satz}
Ist eine Gerade durch den Ursprung im $\mathbb{R}^2$ gegeben, die einen Winkel von $\frac{\alpha}{2}$ mit der $x_1$-Achse einschließt, dann ist die lineare Abbildung, die jeden Punkt $x \in \mathbb{R}^2$ an dieser Geraden spiegelt, durch die Matrix
\begin{align*}
    \begin{pmatrix}
        \cosa & \sina \\
        \sina & -\cosa
    \end{pmatrix}
\end{align*}
gegeben.

% TODO: SKIZZE

Wir bezeichnen Spiegelungen mit $\sigalph$.

\mdf{Satz}
Ist $\sigalph$ eine Spiegelung an der Ursprungsgeraden
\begin{align*}
    \left\{\mu\begin{pmatrix}
        \cosah \\[2mm]
        \sinah
    \end{pmatrix}\,|\,\mu \in \mathbb{R}\right\}
\end{align*}
so hat $\sigalph$ die Eigenwerte $\lambda _1 = 1$ und $\lambda _2 = -1$.

\vspace{0.3cm}

\textbf{Beweis:}

Die darstellende Matrix hat nach Satz 3 die Form
\begin{align*}
    A &= \begin{pmatrix}
        \cosa & \sina \\
        \sina & -\cosa
    \end{pmatrix}
\end{align*}
Das characteristische Polynom
\begin{align*}
    \chi _A(\lambda) &= \text{det}(A - \lambda \mathbbm{1}_2) \\
    &= \text{det}\left(\begin{pmatrix}
        \cosa - \lambda & \sina \\
        \sina & -\cosa - \lambda
    \end{pmatrix}\right) \\
    &= -(\cosa - \lambda)(\cosa + \lambda) - \text{sin}^2(\alpha)&\text{Nach 3. binomischer Formel} \\
    &= \lambda ^2 - \text{cos}^2(\alpha) - \text{sin}^2(\alpha) \\
    &= \lambda ^2 - (\text{cos}^2(\alpha) + \text{sin}^2(\alpha))&\text{Nach Satz von Pythagoras} \\
    &= \lambda ^2 - 1 \\
    &= (\lambda - 1)(\lambda + 1) \\
    &\Rightarrow \lambda _1 = 1 \quad \lambda _2 = -1
\end{align*}

\mdf{Satz}
Sei $\sigalph$ wie in Satz $4$. Dann ist die darstellende Matrix diagonalisierbar und die Eigenräume sind orthogonal zueinander.

\vspace{0.3cm}

\textbf{Beweis:}

EW von $\sigalph$
\begin{align*}
    &x_1 = 1 \quad x_2 = -1 \\
    \text{Eig}_A(1) &= \left\{\mu\begin{pmatrix}
        \cosah \\[2mm]
        \sinah
    \end{pmatrix}\,|\,\mu \in \mathbb{R}\right\} \\
    \Rightarrow \text{Basis }&\left\{\begin{pmatrix}
        \cosah \\[2mm]
        \sinah
    \end{pmatrix}\right\} \\[0.5cm]
    \text{Eig}_A(-1) &= \left\{\mu\begin{pmatrix}
        -\sinah \\[2mm]
        \cosah
    \end{pmatrix}\,|\,\mu \in \mathbb{R}\right\} \\
    \Rightarrow \text{Basis }&\left\{\begin{pmatrix}
        -\sinah \\[2mm]
        \cosah
    \end{pmatrix}\right\}
\end{align*}
Ist
\begin{align*}
    \{b_1, b_2\} &= \left\{\begin{pmatrix}
        \cosah \\[2mm]
        \sinah
    \end{pmatrix}, \begin{pmatrix}
        -\sinah \\[2mm]
        \cosah
    \end{pmatrix}\right\}
\end{align*}
linear unabhängig?
\begin{align*}
    &\mu _1 b_1 + \mu _2 b_2 = 0 \\
    &\left(\begin{array}{cc|c}
        \cosah & -\sinah & 0 \\[2mm]
        \sinah & \cosah & 0
    \end{array}\right) \\
    &\quad\Downarrow\quad\text{I}\cdot\sinah \\
    &\quad\Downarrow\quad\text{II}\cdot\cosah \\
    &\left(\begin{array}{cc|c}
		\cosah\cdot\sinah & -\text{sin}^2\left(\frac{\alpha}{2}\right) & 0 \\[2mm]
		\sinah\cdot\cosah & \text{cos}^2\left(\frac{\alpha}{2}\right) & 0
    \end{array}\right) \\
    &\quad\Downarrow\quad\text{II-I} \\
    &\left(\begin{array}{cc|c}
		\cosah\cdot\sinah & -\text{sin}^2\left(\frac{\alpha}{2}\right) & 0 \\[2mm]
		0 & 1 & 0
    \end{array}\right) \\
\end{align*}
Erklärung zum letzten Eintrag:
\begin{align*}
	\text{sin}^2\left(\frac{\alpha}{2}\right) + \text{cos}^2\left(\frac{\alpha}{2}\right) &= 1 \\[0.5cm]
	1\cdot\mu_2 = 0 \Rightarrow \mu_2 &= 0\quad\text{Einsetzen in I} \\
	\sinah\cdot\cosah\cdot\mu_1 = 0 \Rightarrow \mu_1 &= 0
\end{align*}

Die Vektoren sind linear unabhängig. Also bilden
\begin{align*}
	\begin{pmatrix}
		\cosah \\[2mm]
		\sinah
	\end{pmatrix}\text{ und }\begin{pmatrix}
		-\sinah \\[2mm]
		\cosah
	\end{pmatrix}
\end{align*}
eine Basis von $\mathbb{R}^2$. $\Rightarrow \sigalph$ ist diagonalisierbar.

Bestimme Skalarprodukt:
\begin{align*}
	&\left\langle \begin{pmatrix}
		\cosah \\[2mm]
		\sinah
	\end{pmatrix}, \begin{pmatrix}
		-\sinah \\[2mm]
		\cosah
	\end{pmatrix}\right\rangle \\
	&= \cosah\cdot\left(-\sinah\right) + \sinah\cdot\cosah = 0
\end{align*}
$\Rightarrow$ Die Vektoren sind orthogonal zueinander.


\termin{17.05.2016}

\kapitel{Grenzwerte und Stetigkeit}
\mdf{Definition}
Eine Abbildung
\begin{align*}
	&a : \mathbb{N}_{\geq 1} \rightarrow \mathbb{R},\text{ auch geschrieben als } a_1, a_2, a_3, \dots \\
	&n \mapsto a_n
\end{align*}
heißt \begr[Reelle Folge]{reelle Folge}. Die $a_n$ heißen \begr[Glied einer Folge]{Glieder} der Folge, $n$ heißt \begr[Index einer Folge]{Index} der Folge.

Es gibt mehrere Möglichkeiten, Folgen zu beschreiben:
\begin{itemize}
	\item{Rekursiv: Bestimme $a_n$ in Abhängigkeit vorangegangener Folgeglieder.}
	\item{Direkt: Bestimme $a_n$ ohne andere Folgeglieder zu kennen.}
\end{itemize}

\mdf{Definition}
Eine Folge $a_n$ heißt
\begin{itemize}
	\item{\begr[Beschränktheit nach oben, Folge]{Nach oben beschränkt}, wenn es ein $K \in \mathbb{R}$ gibt, mit $a_n \leq K \enspace\forall a$.}
	\item{\begr[Beschränktheit nach unten, Folge]{Nach unten beschränkt}, wenn es ein $K \in \mathbb{R}$ gibt, mit $K \leq a \enspace\forall a$.}
	\item{\begr[Beschränktheit, Folge]{Beschränkt}, wenn sie nach oben und unten beschränkt ist.}
	\item{\begr[Monoton wachsende Folge]{Monoton wachsend}, wenn für alle $n$ gilt: $a_{n+1} \geq a_n$}
	\item{\begr[Streng monoton wachsende Folge]{Streng monoton wachsend}, wenn für alle $n$ gilt: $a_{n+1} > a_n$}
	\item{\begr[Monoton fallende Folge]{Monoton fallend}, wenn für alle $n$ gilt: $a_{n+1} \leq a_n$}
	\item{\begr[Streng monoton fallende Folge]{Streng monoton fallend}, wenn für alle $n$ gilt: $a_{n+1} < a_n$}
\end{itemize}

\mdf{Beispiel}
\begin{itemize}
	\item{$a_n = n\qquad a_1 = 1\enspace a_2 = 2\enspace a_3 = 3\enspace\dots$\\ist streng monoton wachsend und nach unten beschränkt.}
	\item{$a_1 = 2,\enspace a_{n+1} = \frac{1}{2}\cdot\left(a_n + \frac{2}{a_n}\right)\qquad a_1 = 2,\enspace a_2 = 1,5,\enspace a_3 = \frac{17}{12} \approx 1,41666\enspace\dots$\\ist monoton fallend und beschränkt.}
	\item{$a_1 = 2,\enspace a_n = n\cdot a_{n-1}\qquad a_2 = 2,\enspace a_3 = 6,\enspace a_4 = 24\enspace\dots$\\ist streng monoton wachsend und nach unten beschränkt. Direkt angegeben lautet die Formel $a_n = n!$}
	\item{$a_n = (-1)^n\qquad a_1 = -1,\enspace a_2 = 1,\enspace a_3 = -1\enspace\dots$\\Eine Folge, die mit jedem Glied das Vorzeichen wechselt, heißt \begr[Alternierende Folge]{alternierende Folge}.}
\end{itemize}

\mdf{Definition}
Eine Folge $a_n$ heißt \begr[Konvergente Folge]{konvergent} gegen $a$, falls
\begin{align*}
	\forall \varepsilon > 0\enspace\exists\, n_0 \in \mathbb{N}\quad\forall n\geq n_n : |a - a_n| < \varepsilon
\end{align*}
D.h. für jede noch so kleine Zahl $\varepsilon$ gibt es einen Folgenindex $n_0$, so dass alle Folgenglieder mit Index $n \geq n_0$ im Intervall $(a-\varepsilon, a+\varepsilon)$ liegen.

Kurzschreibweise:
\begin{align*}
	\lim\limits_{n \to \infty}a_n = a
\end{align*}
$a_n$ \begr[Konvergieren]{konvergiert} gegen den Grenzwert $a$.

\mdf{Satz}
Der Grenzwert einer Folge ist eindeutig bestimmt und jede konvergente Folge ist beschränkt.

Nicht-konvergente Folgen heißen \begr[Divergente Folge]{divergent}.

\mdf{Beispiel}
\begin{itemize}
	\item{$a_n = 3$\\Konvergent mit Grenzwert 3. Für alle $\varepsilon > 0$ erfüllt $n_0 = 1$ die Bedingung, denn $|3 - a_n| = 0 < \varepsilon$.}
	\item{$a_n = \frac{1}{n}$\\Konvergent mit Grenzwert $a = 0$. Für jedes $\varepsilon > 0$ wähle $n_0$ als kleinste natürliche Zahl mit $n_0 > \frac{1}{\varepsilon}$.}
\end{itemize}
\begin{align*}
	&\text{z.B. }\varepsilon = 0,01 \Rightarrow  n_0 > \frac{1}{0,01} \Rightarrow n_0 = 101 \\
	&\frac{1}{102}, \frac{1}{103}, \dots < \varepsilon \\[2mm]
	&\lim\limits_{n \to \infty}\frac{1}{n} = 0
\end{align*}
\begin{itemize}
	\item{$a_n = (-1)^n \cdot 2^n$\\Ist unbeschränkt und daher divergent.}
\end{itemize}

\mdf{Satz}
Sind $a_n, b_n$ konvergente Folgen mit Grenzwert $a$ bzw. $b$ und $c \in \mathbb{R}$, so gelten
\begin{itemize}
	\item{$c \cdot a_n$ ist konvergent mit $\lim\limits_{n \to \infty}(c \cdot a_n) = c \cdot a$}
	\item{$a_n + b_n$ ist konvergent mit $\lim\limits_{n \to \infty}(a_n + b_n) = a + b$}
	\item{$a_n \cdot b_n$ ist konvergent mit $\lim\limits_{n \to \infty}(a_n \cdot b_n) = a \cdot b$}
	\item{$\frac{a_n}{b_n}$ ist konvergent mit $\lim\limits_{n \to \infty}\left(\frac{a_n}{b_n}\right) = \frac{a}{b}$, falls $b \neq 0$.}
\end{itemize}

\mdf{Satz}
Jede beschränkte und monoton wachsende oder fallende Folge ist konvergent.

Das Produkt einer beschränkten Folge und einer Nullfolge ist eine Nullfolge. Eine Nullfolge ist eine Folge mit Grenzwert $0$.

\mdf{Definition}
Ist $a_n$ eine Folge und $\frac{1}{a_n}$ konvergent gegen $0$, dann heißt $a_n$
\begin{itemize}
	\item{bestimmt divergent gegen $\infty$, falls es ein $n_0$ gibt mit $a_n > 0$ für alle $n \geq n_0$.}
	\item{bestimmt divergent gegen $-\infty$, falls es ein $n_0$ gibt mit $a_n >0$ für alle $n \geq n_0$.}
\end{itemize}
Schreibweise:
\begin{align*}
	\lim\limits_{n \to \infty}a_n &= \infty \\[2mm]
	\lim\limits_{n \to \infty}a_n &= -\infty
\end{align*}

\mdf{Definition}
Eine Folge $a_n$ heißt \begr{Couchy--Folge}, falls
\begin{align*}
	\forall\varepsilon > 0\enspace\exists\,n_0 \in \mathbb{N}\quad\forall n, m \geq n_0 : |a_m - a_n| < \varepsilon
\end{align*}


\printindex

\end{document}
