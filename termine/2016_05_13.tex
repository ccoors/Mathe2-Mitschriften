\termin{13.05.2016}

\kapitel{Geometrie Teil 2}
\mdf{Beispiel}

\begin{center}
\begin{tikzpicture}[>=triangle 45,font=\sffamily]
\draw[step=1cm,gray,very thin] (-3.9,-3.9) grid (3.9,3.9);
\draw[thick,->] (-4.5,0) -- (4.5,0) node[anchor=north west] {$x_1$};
\draw[thick,->] (0,-4.5) -- (0,4.5) node[anchor=south east] {$x_2$};
\foreach \x in {-4,-3,-2,-1,1,2,3,4}
        \draw (\x cm,4pt) -- (\x cm,-4pt) node[anchor=north] {$\x$};
\foreach \y in {-4,-3,-2,-1,1,2,3,4}
        \draw (4pt,\y cm) -- (-4pt,\y cm) node[anchor=east] {$\y$};

\draw[black,dashed] (-3.5,-3.5) -- (3.5,3.5) node[anchor=south west] {$\mu \begin{pmatrix}1\\1\end{pmatrix}$};

\draw[thick,->] (0,0) -- (1.2,2.5) node[anchor=south east] {};
\draw[thick,->] (0,0) -- (2.5,1.2) node[anchor=south east] {};

\draw (1.2,2.5) node [cross=4pt,red] {};
\node [align=left] at (1.8,2.5) {$\varphi (x)$};

\draw (2.5,1.2) node [cross=4pt,red] {};
\node [align=left] at (2.8,1.2) {$x$};
\end{tikzpicture}
\end{center}
Es sei $\varphi : \mathbb{R}^2 \rightarrow \mathbb{R}^2$ die lineare Abbildung, die jeden Punkt $x \in \mathbb{R}^2$ an der Geraden
\begin{align*}
    \left\{\mu\begin{pmatrix}1\\1\end{pmatrix}\,|\,\mu \in \mathbb{R}\right\}
\end{align*}
spiegelt.

\begin{enumerate}
    \item{Wie sieht die darstellende Matrix von $\varphi$ aus?}
    \item{Welche Eigenschaften hat sie?}
\end{enumerate}

\begin{align*}
    \varphi (e_1) &= e_2\qquad \varphi (e_2) = e_1 \\
    \Rightarrow A &= \begin{pmatrix} 0 & 1 \\ 1 & 0 \end{pmatrix} = \left(\varphi (e_1), \varphi (e_2)\right)
\end{align*}
Eigenwerte:
\begin{align*}
    \chi_A (\lambda) &= \text{det}\left(\begin{pmatrix} -\lambda & 1 \\ 1 & -\lambda \end{pmatrix}\right) = \lambda ^2 - 1 \\
    \chi_A (\lambda) &= \text{det}(A - \lambda \mathbbm{1}_n) \\
    &\Rightarrow \lambda _1 = 1 \qquad \lambda _2 = -1
\end{align*}
EV:
\begin{align*}
    &\lambda _1 = 1 \\
    &\left(\begin{array}{cc|c}
        -1 & 1 & 0 \\
        1 & -1 & 0
    \end{array}\right) \xRightarrow{\text{II+I}} \left(\begin{array}{cc|c}
        -1 & 1 & 0 \\
        0 & 0 & 0
    \end{array}\right) \\
    &\Rightarrow - x_1 + x_2 = 0 \Leftrightarrow x_1 = x_2 \\
    &\text{Eig}_A(1) = \left\{\begin{pmatrix}x\\x\end{pmatrix}\,|\,x \in \mathbb{R}\right\} \\
    &\text{z.B. Basis: } \left\{\begin{pmatrix}1\\1\end{pmatrix}\right\} \\[0.5cm]
    &\lambda _2 = -1 \\
    &\left(\begin{array}{cc|c}
        1 & 1 & 0 \\
        1 & 1 & 0
    \end{array}\right) \xRightarrow{\text{II-I}} \left(\begin{array}{cc|c}
        1 & 1 & 0 \\
        0 & 0 & 0
    \end{array}\right) \\
    &\Rightarrow x_1 + x_2 = 0 \Leftrightarrow x_1 = -x_2 \\
    &\text{Eig}_A(-1) = \left\{\begin{pmatrix}x\\-x\end{pmatrix}\,|\,x \in \mathbb{R}\right\} \\
    &\text{z.B. Basis: } \left\{\begin{pmatrix}1\\-1\end{pmatrix}\right\} \\[1cm]
    &\Rightarrow \varphi \text{ ist diagonalisierbar, da } \left\{\begin{pmatrix}1\\1\end{pmatrix},\begin{pmatrix}1\\-1\end{pmatrix}\right\} \text{ eine Basis von } \mathbb{R}^2 \text{ darstellt.}
\end{align*}

\mdf{Definition}
\begin{center}
\begin{tikzpicture}[>=triangle 45,font=\sffamily]
\coordinate (v1) at ({2*cos(0.6 r)},{2*sin(0.6 r)});
\coordinate (v2) at ({2*cos(0.6 r)},0);
\coordinate (c) at (0,0);

\pic["$\alpha$", draw=red, thick, -, angle eccentricity=0.6, angle radius=1cm] {angle=v2--c--v1};

\draw [step=2cm,gray,very thin] (-2.5,-2.5) grid (2.5,2.5);
\draw [->] (-3,0) -- (3,0) node[anchor=north west] {$x$};
\draw [->] (0,-2.5) -- (0,3) node[anchor=south east] {$y$};

\draw (0,0) circle (2);

\draw [thick] (0,0) -- (v1);
\draw [thick] (0,0) -- (v2);
\draw [thick] (v2) -- (v1);

\node [below] at (0,-2.5) {Einheitskreis};

\node [below] at (0.9,0) {$x$};
\node [right] at (2,0.5) {$y$};
\end{tikzpicture}
\end{center}
Für ein rechtwinkliges Dreieck im Einheitskreis definiert man für den Winkel $\alpha \in [0^{\circ}, 360^{\circ}]$ bzw. $\alpha \in [0, 2\pi]$
\begin{align*}
    \sina &= y \\
    \cosa &= x
\end{align*}
den \begr{Sinus} und den \begr{Kosinus} von $\alpha$.

Beachte: $x, y$ sind vorzeichenbehaftet, d.h. z.B. für $90^{\circ} < \alpha < 270^{\circ}$ bzw. $\frac{\pi}{2} < \alpha < \frac{3\pi}{2}$ ist $x < 0$.

\mdf{Satz}
Ist eine Gerade durch den Ursprung im $\mathbb{R}^2$ gegeben, die einen Winkel von $\frac{\alpha}{2}$ mit der $x_1$-Achse einschließt, dann ist die lineare Abbildung, die jeden Punkt $x \in \mathbb{R}^2$ an dieser Geraden spiegelt, durch die Matrix
\begin{align*}
    \begin{pmatrix}
        \cosa & \sina \\
        \sina & -\cosa
    \end{pmatrix}
\end{align*}
gegeben.
\begin{center}
\begin{tikzpicture}[>=triangle 45,font=\sffamily]
\draw[step=1cm,gray,very thin] (-3.9,-3.9) grid (3.9,3.9);
\draw[thick,->] (-4.5,0) -- (4.5,0) node[anchor=north west] {$x_1$};
\draw[thick,->] (0,-4.5) -- (0,4.5) node[anchor=south east] {$x_2$};
\foreach \x in {-4,-3,-2,-1,1,2,3,4}
        \draw (\x cm,4pt) -- (\x cm,-4pt) node[anchor=north] {$\x$};
\foreach \y in {-4,-3,-2,-1,1,2,3,4}
        \draw (4pt,\y cm) -- (-4pt,\y cm) node[anchor=east] {$\y$};

\draw[black,dashed] (-3.5,-3.5) -- (3.5,3.5) node[anchor=south west] {};

\coordinate (a) at (1,1);
\coordinate (b) at (0,0);
\coordinate (c) at (1,0);

\pic["$\frac{\alpha{}}{2}$", draw=red, thick, -, angle eccentricity=0.6, angle radius=1.2cm] {angle=c--b--a};

\draw[thick,->] (0,0) -- (-1.2,2.5) node[anchor=south east] {};
\draw[thick,->] (0,0) -- (2.5,-1.2) node[anchor=south east] {};

\draw (-1.2,2.5) node [cross=4pt,red] {};
\node [align=left] at (-1.8,2.5) {$\varphi (x)$};

\draw (2.5,-1.2) node [cross=4pt,red] {};
\node [align=left] at (2.8,-1.2) {$x$};
\end{tikzpicture}
\end{center}

Wir bezeichnen Spiegelungen mit $\sigalph$.

\mdf{Satz}
Ist $\sigalph$ eine Spiegelung an der Ursprungsgeraden
\begin{align*}
    \left\{\mu\begin{pmatrix}
        \cosah \\[2mm]
        \sinah
    \end{pmatrix}\,|\,\mu \in \mathbb{R}\right\}
\end{align*}
so hat $\sigalph$ die Eigenwerte $\lambda _1 = 1$ und $\lambda _2 = -1$.

\vspace{0.3cm}

\textbf{Beweis:}

Die darstellende Matrix hat nach Satz 3 die Form
\begin{align*}
    A &= \begin{pmatrix}
        \cosa & \sina \\
        \sina & -\cosa
    \end{pmatrix}
\end{align*}
Das characteristische Polynom
\begin{align*}
    \chi _A(\lambda) &= \text{det}(A - \lambda \mathbbm{1}_2) \\
    &= \text{det}\left(\begin{pmatrix}
        \cosa - \lambda & \sina \\
        \sina & -\cosa - \lambda
    \end{pmatrix}\right) \\
    &= -(\cosa - \lambda)(\cosa + \lambda) - \text{sin}^2(\alpha)&\text{Nach 3. binomischer Formel} \\
    &= \lambda ^2 - \text{cos}^2(\alpha) - \text{sin}^2(\alpha) \\
    &= \lambda ^2 - (\text{cos}^2(\alpha) + \text{sin}^2(\alpha))&\text{Nach Satz von Pythagoras} \\
    &= \lambda ^2 - 1 \\
    &= (\lambda - 1)(\lambda + 1) \\
    &\Rightarrow \lambda _1 = 1 \quad \lambda _2 = -1
\end{align*}

\mdf{Satz}
Sei $\sigalph$ wie in Satz $4$. Dann ist die darstellende Matrix diagonalisierbar und die Eigenräume sind orthogonal zueinander.

\vspace{0.3cm}

\textbf{Beweis:}

EW von $\sigalph$
\begin{align*}
    &x_1 = 1 \quad x_2 = -1 \\
    \text{Eig}_A(1) &= \left\{\mu\begin{pmatrix}
        \cosah \\[2mm]
        \sinah
    \end{pmatrix}\,|\,\mu \in \mathbb{R}\right\} \\
    \Rightarrow \text{Basis }&\left\{\begin{pmatrix}
        \cosah \\[2mm]
        \sinah
    \end{pmatrix}\right\} \\[0.5cm]
    \text{Eig}_A(-1) &= \left\{\mu\begin{pmatrix}
        -\sinah \\[2mm]
        \cosah
    \end{pmatrix}\,|\,\mu \in \mathbb{R}\right\} \\
    \Rightarrow \text{Basis }&\left\{\begin{pmatrix}
        -\sinah \\[2mm]
        \cosah
    \end{pmatrix}\right\}
\end{align*}
Ist
\begin{align*}
    \{b_1, b_2\} &= \left\{\begin{pmatrix}
        \cosah \\[2mm]
        \sinah
    \end{pmatrix}, \begin{pmatrix}
        -\sinah \\[2mm]
        \cosah
    \end{pmatrix}\right\}
\end{align*}
linear unabhängig?
\begin{align*}
    &\mu _1 b_1 + \mu _2 b_2 = 0 \\
    &\left(\begin{array}{cc|c}
        \cosah & -\sinah & 0 \\[2mm]
        \sinah & \cosah & 0
    \end{array}\right) \\
    &\quad\Downarrow\quad\text{I}\cdot\sinah \\
    &\quad\Downarrow\quad\text{II}\cdot\cosah \\
    &\left(\begin{array}{cc|c}
		\cosah\cdot\sinah & -\text{sin}^2\left(\frac{\alpha}{2}\right) & 0 \\[2mm]
		\sinah\cdot\cosah & \text{cos}^2\left(\frac{\alpha}{2}\right) & 0
    \end{array}\right) \\
    &\quad\Downarrow\quad\text{II-I} \\
    &\left(\begin{array}{cc|c}
		\cosah\cdot\sinah & -\text{sin}^2\left(\frac{\alpha}{2}\right) & 0 \\[2mm]
		0 & 1 & 0
    \end{array}\right) \\
\end{align*}
Erklärung zum letzten Eintrag:
\begin{align*}
	\text{sin}^2\left(\frac{\alpha}{2}\right) + \text{cos}^2\left(\frac{\alpha}{2}\right) &= 1 \\[0.5cm]
	1\cdot\mu_2 = 0 \Rightarrow \mu_2 &= 0\quad\text{Einsetzen in I} \\
	\sinah\cdot\cosah\cdot\mu_1 = 0 \Rightarrow \mu_1 &= 0
\end{align*}

Die Vektoren sind linear unabhängig. Also bilden
\begin{align*}
	\begin{pmatrix}
		\cosah \\[2mm]
		\sinah
	\end{pmatrix}\text{ und }\begin{pmatrix}
		-\sinah \\[2mm]
		\cosah
	\end{pmatrix}
\end{align*}
eine Basis von $\mathbb{R}^2$. $\Rightarrow \sigalph$ ist diagonalisierbar.

Bestimme Skalarprodukt:
\begin{align*}
	&\left\langle \begin{pmatrix}
		\cosah \\[2mm]
		\sinah
	\end{pmatrix}, \begin{pmatrix}
		-\sinah \\[2mm]
		\cosah
	\end{pmatrix}\right\rangle \\
	&= \cosah\cdot\left(-\sinah\right) + \sinah\cdot\cosah = 0
\end{align*}
$\Rightarrow$ Die Vektoren sind orthogonal zueinander.
