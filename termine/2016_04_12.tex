\termin{12.04.2016}
\kapitel{Geometrie Teil 1}
\begin{center}
\begin{tikzpicture}[>=triangle 45,font=\sffamily]
\draw[step=1cm,gray,very thin] (0.1,0.1) grid (3.9,3.9);
\draw[thick,->] (0,0) -- (4.5,0) node[anchor=north west] {$x$};
\draw[thick,->] (0,0) -- (0,4.5) node[anchor=south east] {$y$};
\foreach \x in {1,2,3,4}
        \draw (\x cm,4pt) -- (\x cm,-4pt) node[anchor=north] {$\x$};
\foreach \y in {1,2,3,4}
        \draw (4pt,\y cm) -- (-4pt,\y cm) node[anchor=east] {$\y$};

\draw (3,2) node [cross=4pt,red] {};
\node [align=left] at (3.4,2) {$P$};
\end{tikzpicture}
\end{center}

Punkte in der Ebene können wie folgt beschrieben werden:
\begin{enumerate}
	\item{Als geordnetes Paar: $P = (3, 2)$.}
	\item{Wir wählen \glqq{}grundlegende Basiselemente\grqq{} z.B. $a$ = \glqq{}gehe\grqq{} 1 Schritt in $x$-Richtung und $b$ = \glqq{}gehe\grqq{} 1 Schritt in $y$-Richtung. Dann können wir $P$ schreiben als $3a + 2b$.}
	\item{Als Vektor: $P = \begin{pmatrix}3 \\ 2\end{pmatrix}$, wie ein geordnetes Paar untereinander.}
\end{enumerate}

Offenbar hängt die Darstellung von $P$ von der Achsenbeschriftung ab, bzw. von den \glqq{}Basiselementen\grqq{}.

Z.B. Setze $a$ = \glqq{}gehe\grqq{} 3 Schritte in $x$-Richtung, $b$ wie eben, dann ist $P = a + 2b$. Oder $a$ = \glqq{}gehe\grqq{} 1 Schritt entgegen der $x$-Richtung, dann $P = -3a + 2b$.

\mdf{Definition}
Die \begr[Reelle Ebene]{reelle Ebene} ist das kartesische Produkt $\mathbb{R}^2 = \mathbb{R} \times \mathbb{R}$ und enthält alle Punkte $(x, y)$ mit $x, y \in \mathbb{R}$, also $\mathbb{R}^2 = \{(x,y)\,|\,x,y \in \mathbb{R}\}$. Statt $(x, y)$ schreiben wir in Zukunft $\begin{pmatrix}x \\ y\end{pmatrix} \in \mathbb{R}^2$ für Punkte in der reellen Ebene.

$\begin{pmatrix}x \\ y\end{pmatrix}$ heißt (2-dimensionaler) \begr{Vektor} oder \begr{Spaltenvektor}.

Analog können wir den dreidimensionalen Raum beschreiben als
\begin{align*}
	\mathbb{R}^3 = \mathbb{R} \times \mathbb{R} \times \mathbb{R} = \{(x,y,z)\,|\,x,y,z \in \mathbb{R}\}
\end{align*}

Offenbar ist $\mathbb{R}^3 = \mathbb{R}^2 \times \mathbb{R}$.

Analog definiert man den $\mathbb{R}^n$, den $n$-dimensionalen reellen Raum:

\begin{align*}
	\mathbb{R}^n = \left\{\begin{pmatrix}x_1 \\ x_2 \\ x_3 \\ \vdots \\ x_n \end{pmatrix}\,\middle|\, x_1, x_2, \dots, x_n \in \mathbb{R} \right\}
\end{align*}

\mdf{Definition}
\begin{align*}
	\text{Für 2 Vektoren }x = \begin{pmatrix}x_1 \\ \vdots \\ x_n \end{pmatrix},\enspace y = \begin{pmatrix}y_1 \\ \vdots \\ y_n \end{pmatrix}\text{ ist } x + y\text{ definiert als }\begin{pmatrix}x_1 \\ \vdots \\ x_n \end{pmatrix} + \begin{pmatrix}y_1 \\ \vdots \\ y_n \end{pmatrix} = \begin{pmatrix}x_1 + y_1 \\ \vdots \\ x_n + y_n \end{pmatrix}
\end{align*}

\mdf{Beispiel}
\begin{align*}
	\begin{pmatrix} -1 \\ 0 \\ 2 \\ 1,5 \end{pmatrix} + \begin{pmatrix} 1 \\ \pi \\ 0 \\ -3 \end{pmatrix} = \begin{pmatrix} 0 \\ \pi \\ 2 \\ -1,5 \end{pmatrix}
\end{align*}
Beachte: Man kann nur gleichdimensionale Vektoren addieren.

\mdf{Definition}
Die \begr{Skalarmultiplikation} eines Vektors $x \in \mathbb{R}^n$ mit einer festen Zahl (Skalar genannt) $\lambda \in \mathbb{R}$ ist gegeben durch
\begin{align*}
	\lambda \cdot x = \begin{pmatrix} \lambda \cdot x_1 \\ \vdots \\ \lambda \cdot x_n \end{pmatrix}
\end{align*}

\mdf{Beispiel}
\begin{align*}
	\lambda = 2, \enspace x = \begin{pmatrix}2 \\ 1\end{pmatrix} \quad \lambda \cdot x = \begin{pmatrix}4 \\ 2\end{pmatrix}
\end{align*}

Mit Vektoren kann man fast wie mit Zahlen rechnen.

Für $x,y,z \in \mathbb{R}^n$ gilt
\begin{itemize}
	\item{$(x+y)+z = x+(y+z)$ Assoziativgesetz}
	\item{$x+y = y+x$ Kommutativgesetz}
	\item{Der Nullvektor $0 = \begin{pmatrix}0 \\ \vdots \\ 0\end{pmatrix} \in \mathbb{R}^n$ ist neutrales Element der Addition}
	\item{Zu jedem $x \in \mathbb{R}^n$ gibt es ein \glqq{}negatives\grqq{} $-x \in \mathbb{R}^n$ mit $x + (-x) = 0 \in \mathbb{R}^n$}
	\item{Für $\lambda,\mu \in \mathbb{R}$ gilt $(\lambda + \mu)\cdot x = \lambda \cdot x + \mu \cdot x$ Distributivgesetz}
	\item{Für $\lambda,\mu \in \mathbb{R}$ gilt $(\lambda \cdot \mu) \cdot x = \lambda \cdot (\mu \cdot x)$ Assoziativgesetz der Skalarmultiplikation}
\end{itemize}

\mdf{Bemerkung}
Für Vektoren gibt es kein \glqq{}sinnvolles\grqq{} Produkt, so dass $x\cdot y$ in $\mathbb{R}^n$ liegt und die Eigenschaften gelten, die man von dem Produkt erwartet.

\mdf{Definition}
Eine Menge von Vektoren $v_1,\dots,v_k \in \mathbb{R}^n$ heißt \begr[Lineare Unabhängigkeit]{linear unabhängig} falls die Gleichung

\begin{align*}
	\lambda _1 \cdot v_1 +\dots+\lambda _k \cdot v_k = 0\quad \lambda _1,\dots,\lambda _k \in \mathbb{R}
\end{align*}

nur die Lösung $\lambda _1,\dots,\lambda _k = 0$ hat.

Nicht linear unabhängige Vektoren heißen linear abhängig.

Für Vektoren $v_1,\dots,v_k \in \mathbb{R}^n$ und Skalare $\lambda _1,\dots\lambda _k \in \mathbb{R}$ heißt $\lambda _1 \cdot v_1+\dots+\lambda _k \cdot v_k$ \begr{Linearkombination}.

\mdf{Definition}
Eine Teilmenge $U \subseteq \mathbb{R}^n$ heißt \begr{Unterraum} von $\mathbb{R}^n$ falls gilt:
\begin{description}
	\item[(U1)]{$0 \in U$}
	\item[(U2)]{$\forall u \in U,\, \forall \lambda \in \mathbb{R}\,:\,\lambda \cdot u \in U$}
	\item[(U3)]{$\forall u,v \in U\,:\,u+v \in U$}
\end{description}

\mdf{Beispiel}
Betrachte $\mathbb{R}^3$
\begin{align*}
	U = \left\{\begin{pmatrix}x \\ y \\ 0\end{pmatrix}\,\middle|\,x,y \in \mathbb{R}\right\}\text{ ist Unterraum des }\mathbb{R}^3
\end{align*}

Betrachte $\mathbb{R}^2$
\begin{align*}
	U = \left\{\begin{pmatrix}x \\ -x\end{pmatrix}\,\middle|\,x \in \mathbb{R}\right\}\,\subseteq\,\mathbb{R}^2\text{ ist Unterraum des }\mathbb{R}^2
\end{align*}

\begin{center}
\begin{tikzpicture}[>=triangle 45,font=\sffamily]
\draw[step=1cm,gray,very thin] (-3.9,-3.9) grid (3.9,3.9);
\draw[thick,->] (-4.5,0) -- (4.5,0) node[anchor=north west] {$x$};
\draw[thick,->] (0,-4.5) -- (0,4.5) node[anchor=south east] {$y$};
\foreach \x in {-4,-3,-2,-1,1,2,3,4}
        \draw (\x cm,4pt) -- (\x cm,-4pt) node[anchor=north] {$\x$};
\foreach \y in {-4,-3,-2,-1,1,2,3,4}
        \draw (4pt,\y cm) -- (-4pt,\y cm) node[anchor=east] {$\y$};

\draw[thick,red] (-3.5,3.5) -- (3.5,-3.5);
\end{tikzpicture}
\end{center}

Betrachte $\mathbb{R}^n$, $U = \left\{\begin{pmatrix}0 \\ \vdots \\ 0\end{pmatrix}\right\}$ ist Unterraum des $\mathbb{R}^n$.

\mdf{Definition}
Seien $v_1,\dots,v_k \in \mathbb{R}^n$. Die Menge
\begin{align*}
	\text{span}(v_1,\dots,v_k) = \{\lambda _1 \cdot v_1+\dots+\lambda _k \cdot v_k\,|\,\lambda _i \in \mathbb{R}\}
\end{align*}

heißt \begr[Lineare Hülle]{lineare Hülle} oder \begr{Span} von $v_1,\dots,v_k$.

Die Vektoren $v_1,\dots,v_k$ heißen Erzeugendensystem eines Unterraums $U$ falls $U = \text{span}(v_1,\dots,v_k)$.

Die Vektoren $v_1,\dots,v_k$ heißen Basis von $U$ falls $U = \text{span}(v_1,\dots,v_k)$ und $v_1,\dots,v_k$ sind linear unabhängig.
