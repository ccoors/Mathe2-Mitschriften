\termin{03.06.2016}

\bigskip
\textbf{Nun:} Bestimmte Integration

\textbf{Anschauung:} Flächeninhalt, der vom Graphen einer Abbildung und der $x$-Achse auf einem Intervall eingeschlossen wird.

\begin{center}
\begin{tikzpicture}[>=triangle 45,font=\sffamily,scale=1.5,decoration=brace]
\draw [schraffiert] (1,0) rectangle (4,1);

\draw[->] (-0.25,0) -- (5,0) node[anchor=north west] {$x$};
\draw[->] (0,-0.25) -- (0,1.5) node[anchor=south east] {$f(x)$};

\draw[-,thick,blue] (-0.15,1) -- (4.85,1) node[anchor=south east] {$f(x) = c$};

\draw (1,0.1) -- (1,-0.1) node[anchor=north] {$a$};
\draw (4,0.1) -- (4,-0.1) node[anchor=north] {$b$};
\end{tikzpicture}

Flächeninhalt: $c \cdot (b - a)$

\bigskip
\begin{tikzpicture}[>=triangle 45,font=\sffamily,scale=1.5,decoration=brace]
\draw [schraffiert] (1,0) -- (1,0.4605) -- (4,1.2705) -- (4,0) -- cycle;

\draw[->] (-0.25,0) -- (5,0) node[anchor=north west] {$x$};
\draw[->] (0,-0.25) -- (0,2) node[anchor=south east] {$f(x)$};

\draw[-,thick,blue] (-0.15,0.15) -- (4.85,1.5) node[anchor=south east] {$f(x) = m\cdot x+c$};

\draw (1,0.1) -- (1,-0.1) node[anchor=north] {$a$};
\draw (4,0.1) -- (4,-0.1) node[anchor=north] {$b$};
\end{tikzpicture}

Flächeninhalt:
\begin{align*}
	(f(a)+f(b))\cdot (b-a)\cdot \frac{1}{2} &= (ma + c) + (mb + c)(b - a)\frac{1}{2} \\
	&= \frac{1}{2} (m(a+b)+2c)(b-a) \\
	\text{Nach 3. Binomischer Formel } &= \frac{1}{2} m (b^2 - a^2)+c(b-a)
\end{align*}

\bigskip
\begin{tikzpicture}[>=triangle 45,font=\sffamily,scale=1.5,decoration=brace]
\draw [schraffiert] (1,0) -- (1,0.51) -- (1.5,0.53375) -- (2,0.58) -- (2.5,0.65625) -- (3,0.77) -- (3.5,0.92875) -- (4,1.14) -- (4,0) -- cycle;
% Jajaja. :P Es funktioniert gut genug, die Unebenheiten werden von der dicken blauen Linie von f(x) überdeckt.

\draw[->] (-0.25,0) -- (5,0) node[anchor=north west] {$x$};
\draw[->] (0,-0.25) -- (0,2) node[anchor=south east] {$f(x)$};

\draw[domain=-0.15:4.85,smooth,variable=\x,blue,thick] plot ({\x},{0.5+(\x*\x*\x)*0.01}) node[anchor=south east] {$f(x) = ax^3 + bx^2 + cx + d$};

\draw (1,0.1) -- (1,-0.1) node[anchor=north] {$a$};
\draw (4,0.1) -- (4,-0.1) node[anchor=north] {$b$};
\end{tikzpicture}

Flächeninhalt?
\end{center}

\bigskip
\textbf{Naive Idee:} Unterteilung des Intervalls $[a, b]$ in $n$ Teilintervalle der Länge $\Delta n = \frac{b-a}{n}$
\begin{align*}
	F_n &= \sum_{k=0}^{n-1} f(x_k) \cdot \Delta n \\
	\text{mit } &x_0 = a \\
	&x_1 = a + \Delta n \\
	&x_2 = a + 2 \Delta n \\
	&\vdots \\
	&x_k = a + k \Delta n \\
	&\vdots \\
	&x_n = a + n \Delta n = b
\end{align*}

\begin{center}
\begin{tikzpicture}[>=triangle 45,font=\sffamily,scale=1.5,decoration=brace]
\filldraw [fill=red, draw=black] (1,0) rectangle (1.5,0.28);
\filldraw [fill=red, draw=black] (1.5,0) rectangle (2,0.35125);
\filldraw [fill=red, draw=black] (2,0) rectangle (2.5,0.49);
\filldraw [fill=red, draw=black] (2.5,0) rectangle (3,0.71875);
\filldraw [fill=red, draw=black] (3,0) rectangle (3.5,1.06);
\filldraw [fill=red, draw=black] (3.5,0) rectangle (4,1.53625);
\draw (4,0) -- (4,2.17);

\node at (2,3) {Fehlbetrag};
\draw [->] (1.5,2.8) -- (1.4,0.3);
\draw [->] (1.7,2.8) -- (1.9,0.42);
\draw [->] (1.9,2.8) -- (2.4,0.6);
\draw [->] (2.1,2.8) -- (2.9,0.85);
\draw [->] (2.3,2.8) -- (3.4,1.2);
\draw [->] (2.5,2.8) -- (3.9,1.65);


\draw[->] (-0.25,0) -- (5,0) node[anchor=north west] {$x$};
\draw[->] (0,-0.25) -- (0,4) node[anchor=south east] {$f(x)$};

\draw[domain=-0.15:4.85,smooth,variable=\x,blue,thick] plot ({\x},{0.25+(\x*\x*\x)*0.03});

\draw (1,0.1) -- (1,-0.1) node[anchor=north] {$a$};
\draw (4,0.1) -- (4,-0.1) node[anchor=north] {$b$};
\end{tikzpicture}
\end{center}
Definiere Folge von Flächeninhalten:
\begin{align*}
	F_n = \sum_{k=0}^{n-1} f(x_k)\frac{b-a}{n}
\end{align*}
und betrachte Grenzwert:
\begin{align*}
	\lim\limits_{n \to \infty} F_n
\end{align*}
ist der gesuchte Flächeninhalt, falls der Grenzwert existiert.

\mdf{Definition}
Sei $f$ eine auf $[a, b]$ stetige Abbildung. Setze $\Delta n = \frac{b-a}{n}$ und $x_k = a + k \Delta n$ für $k = 0,\dots,n$. Dann konvergiert die Folge der Rechtecksflächen
\begin{align*}
	F_n = \sum_{k=0}^{n-1} f(x_k)\Delta n
\end{align*}
Man nennt den Grenzwert \begr[Bestimmtes Integral]{bestimmtes Integral} von $f$ auf dem Intervall $[a, b]$ und man schreibt
\begin{align*}
	\int_a^b \! f(x) \, \mathrm{d}x = \lim\limits_{n \to \infty} \sum_{k=0}^{n-1} f(x_n) \Delta n
\end{align*}
Dabei heißt $f(x)$ \begr{Integrand}, $x$ \begr{Integrationsvariable} und $[a, b]$ \begr{Integrationsintervall}.

\bigskip
Flächen oberhalb der $x$--Achse gehen positiv, Flächeninhalte unterhalb der $x$--Achse negativ in das Integral ein.

\medskip
Es gilt für jedes $c \in (a, b)$:
\begin{align*}
	\int_a^b \! f(x) \, \mathrm{d}x = \int_a^c \! f(x) \, \mathrm{d}x + \int_c^b \! f(x) \, \mathrm{d}x
\end{align*}
Außerdem:
\begin{align*}
	\int_a^a \! f(x) \, \mathrm{d}x = 0
\end{align*}
und
\begin{align*}
	\int_a^b \! f(x) \, \mathrm{d}x = -\int_b^a \! f(x) \, \mathrm{d}x
\end{align*}

\mdf{Satz} (Hauptsatz der Differential-- und Integralrechnung/Fundamentalsatz der Analysis)

Sei $f$ stetig auf $[a, b]$ und $F$ irgendeine Stammfunktion von $f$. Dann gilt
\begin{align*}
	\int_a^b \! f(x) \, \mathrm{d}x = F(b) - F(a) = F(x)\bigg\vert_a^b
\end{align*}

\mdf{Beispiel}
\begin{align*}
	f(x) &= mx+c \\
	F(x) &= \frac{m}{2}x^2 + cx \\
	\int_a^b \! f(x) \, \mathrm{d}x &= F(b) - F(a) = \frac{m}{2}b^2 + cb - \left(\frac{m}{2}a^2 + ca\right) \\
	&= \frac{m(b^2-a^2)}{2} + c(b-a)
\end{align*}

\mdf{Satz}

Partielle Integration:
\begin{align*}
	\int_a^b \! f(x)g'(x) \, \mathrm{d}x &= f(x)g(x)\bigg\vert_a^b - \int_a^b \! f'(x)g(x) \, \mathrm{d}x \\
\end{align*}

Substitution:
\begin{align*}
	\int_a^b \! f(x)g'(x) \, \mathrm{d}x &= \int_{g(a)}^{g(b)} \! f'(u) \, \mathrm{d}u\text{ mit }u = g(x) \\
\end{align*}
\textbf{Wichtig:} Grenzen anpassen nicht vergessen!

\mdf{Beispiel}
\begin{align*}
	\int_0^1 \! \text{exp}(x) \, \mathrm{d}x &= x\cdot \text{exp}(x)\bigg\vert_0^1 - \int_0^1 \! 1\cdot\text{exp}(x) \, \mathrm{d}x \\
	&= 1\cdot\text{exp}(1)-0\cdot\text{exp}(0) - x\cdot\text{exp}(x)\bigg\vert_0^1 \\
	&= \text{exp}(1) - \text{exp}(1) + \text{exp}(0) = \text{exp}(0) = 1
\end{align*}

\bigskip
\begin{align*}
	\int_0^\pi \! \text{sin}(x^2) \, \mathrm{d}x &= \frac{1}{2}\int_0^{\pi^2} \! \text{sin}(u) \, \mathrm{d}u = -\frac{1}{2}\text{cos}(u)\bigg\vert_0^{\pi^2} = \frac{1}{2}-\frac{\text{cos}(\pi^2)}{2} \approx 0,951343
\end{align*}



