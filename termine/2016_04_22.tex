\termin{22.04.2016}
\kapitel{Lineare Gleichungssysteme}
\mdf{Definition}
Ein \begr{Lineares Gleichungssystem}{lineares Gleichungssystem} aus $m$ Gleichungen und $n$ Unbekannten $x_1,\dots,x_n$ hat die Form
\begin{align*}
    a_{11}x_1 + a_{12}x_2 + \dots + a_{1n}x_n &= b_1 \\
    a_{21}x_1 + a_{22}x_2 + \dots + a_{2n}x_n &= b_2 \\
    \vdots & \\
    a_{m1}x_1 + a_{m2}x_2 + \dots + a_{mn}x_n &= b_m
\end{align*}

Dabei sind die $a_{ij}$ und $b_i$ reelle oder komplexe Zahlen. Die $a_{ij}$ heißen \begr[Koeffizienten (LGS)]{Koeffizienten} des LGS. Sind alle $b_i$ gleich Null, so heißt das LGS \begr[Homogenes LGS]{homogen}, sonst \begr[Inhomogenes LGS]{inhomogen}.

Ein LGS zu lösen bedeutet, Zahlen $x_1,\dots,x_n$ zu finden, so dass alle Gleichungen des LGS erfüllt werden.

\mdf{Satz}
Ein inhomogenes LGS hat entweder keine, genau eine oder unendlich viele Lösungen.

\mdf{Beispiel}

\textbf{a)}
\begin{alignat*}{4}
\text{I}\quad & x+y & = 2 & \quad|\,\text{Aus II: } x = y \\
\text{II}\quad & x-y & = 0 & \quad|\,\text{Aus I: } 2x = 2
\end{alignat*}
Also genau eine Lösung: $x = y = 1$. \\

\textbf{b)}
\begin{alignat*}{4}
\text{I}\quad & x+y & = 2 & \quad|\,\text{Aus II: } x = -y \\
\text{II}\quad & x+y & = 0 & \quad|\,-y+y = 0 \Rightarrow 0 = 2 \enspace\lightning
\end{alignat*} \\

\textbf{c)}
\begin{alignat*}{4}
\text{I}\quad & x+y & = 2 & \quad|\,\text{Aus II: } y = 2 - x \\
\text{II}\quad & 2x + 2y & = 4 & \quad|\,2x + 2(2-x) = 4
\end{alignat*}
Also $4 = 4$. Daher kann $x$ oder $y$ frei gewählt werden.
\begin{align*}
    \text{Also }\begin{pmatrix}x\\y\end{pmatrix} = \begin{pmatrix}t\\2-t\end{pmatrix}\text{ mit }t\in \mathbb{R}
\end{align*}

\begin{samepage}
\textbf{Anschauung:} \\
\textbf{a)}
\begin{center}
\begin{tikzpicture}[>=triangle 45,font=\sffamily]
\draw[step=1cm,gray,very thin] (-2.9,-2.9) grid (3.9,3.9);
\draw[thick,->] (-3.5,0) -- (4.5,0) node[anchor=north west] {$x$-Achse};
\draw[thick,->] (0,-3.5) -- (0,4.5) node[anchor=south east] {$y$-Achse};
\foreach \x in {-2,-1,1,2,3,4}
        \draw (\x cm,4pt) -- (\x cm,-4pt) node[anchor=north] {$\x$};
\foreach \y in {-2,-1,1,2,3,4}
        \draw (4pt,\y cm) -- (-4pt,\y cm) node[anchor=east] {$\y$};

\draw[red,thick] (-2.5,-2.5) -- (3.5,3.5) node[anchor=south west] {$x=y$};
\draw[blue,thick] (-2.5,4.5) -- (3.5,-1.5) node[anchor=north west] {$y=2-x$};
\end{tikzpicture}
\end{center}
Die Lösung liegt genau in der Schnittmenge der beiden Geraden, hier also bei $(1, 1)$.
\end{samepage} \\

\begin{samepage}
\textbf{b)}
\begin{center}
\begin{tikzpicture}[>=triangle 45,font=\sffamily]
\draw[step=1cm,gray,very thin] (-2.9,-2.9) grid (3.9,3.9);
\draw[thick,->] (-3.5,0) -- (4.5,0) node[anchor=north west] {$x$-Achse};
\draw[thick,->] (0,-3.5) -- (0,4.5) node[anchor=south east] {$y$-Achse};
\foreach \x in {-2,-1,1,2,3,4}
        \draw (\x cm,4pt) -- (\x cm,-4pt) node[anchor=north] {$\x$};
\foreach \y in {-2,-1,1,2,3,4}
        \draw (4pt,\y cm) -- (-4pt,\y cm) node[anchor=east] {$\y$};

\draw[red,thick] (-2.5,2.5) -- (3.5,-3.5) node[anchor=south west] {$x=-y$};
\draw[blue,thick] (-2.5,4.5) -- (3.5,-1.5) node[anchor=north west] {$y=2-x$};
\end{tikzpicture}
\end{center}
Die Geraden sind hier parallel und schneiden sich nicht, es gibt also keine Lösung des LGS, die Lösungsmenge ist leer.
\end{samepage} \\

\begin{samepage}
\textbf{c)}
\begin{center}
\begin{tikzpicture}[>=triangle 45,font=\sffamily]
\draw[step=1cm,gray,very thin] (-2.9,-2.9) grid (3.9,3.9);
\draw[thick,->] (-3.5,0) -- (4.5,0) node[anchor=north west] {$x$-Achse};
\draw[thick,->] (0,-3.5) -- (0,4.5) node[anchor=south east] {$y$-Achse};
\foreach \x in {-2,-1,1,2,3,4}
        \draw (\x cm,4pt) -- (\x cm,-4pt) node[anchor=north] {$\x$};
\foreach \y in {-2,-1,1,2,3,4}
        \draw (4pt,\y cm) -- (-4pt,\y cm) node[anchor=east] {$\y$};


\draw[blue,thick] (-2.5,4.5) -- (3.5,-1.5) node[anchor=north west] {$y=2-x$};
\draw[red,thick,dashed] (-2.5,4.5) -- (3.5,-1.5) node[anchor=south west] {$x=4-2y$};
\end{tikzpicture}
\end{center}
Die Geraden liegen aufeinander, es gibt unendlich viele Lösungen.
\end{samepage}

\mdf{Bemerkung}
Ein homogenes LGS hat immer mindestens eine Lösung: $x_1 = \dots = x_n = 0$ (triviale Lösung)

\subsection{Erlaubte Umformungen eines LGS}
\mdf{Satz}
Folgende Umformungen verändern die Lösung eines LGS nicht:

\begin{description}
\item[i)]{Vertauschung zweier Zeilen}
\item[ii)]{Multiplikation einer Zeile mit einer Zahl ungleich Null}
\item[iii)]{Addition des Vielfachen einer Zeile zu einer anderen Zeile}
\end{description}

\subsection{Erweiterte Koeffizientenmatrix}
\mdf{Bemerkung}
Statt ein LGS als System von Gleichungen darzustellen nutzt man oft die \begr[Erweiterte Koeffizientenmatrix]{erweiterte Koeffizientenmatrix}:
\begin{align*}
    \left(\begin{array}{cccc|c}
        a_{11} & a_{12} & \dots & a_{1n} & b_1 \\
        a_{21} & a_{22} & \dots & a_{2n} & b_2 \\
        \vdots & \vdots & \ddots & \vdots & \vdots \\
        a_{m1} & a_{m2} & \dots & a_{mn} & b_m
    \end{array}\right)
\end{align*}

\mdf{Definition}
Eine erweiterte Koeffizientenmatrix ist in \begr{Zeilenstufenform} (ZSF), falls

\begin{description}
\item[i)]{In jeder Zeile ist die erste Zahl ungleich Null eine Eins und steht weiter rechts als die erste nicht-Null-Zahl der Zeile darüber}
\item[ii)]{Alle Nullzeilen befinden sich am unteren Ende der Matrix}
\end{description}
Sie ist in reduzierter Zeilenstufenform, falls zusätzlich gilt:
\begin{description}
\item[iii)]{Über jeder führenden Eins stehen nur Nullen}
\end{description}

\begin{align*}
    \left(\begin{array}{ccccc|c}
        0 & 1 & 1 & 3 & 5 & 1 \\
        0 & 0 & 1 & 1 & 0 & 0 \\
        0 & 0 & 0 & 0 & 1 & 3
    \end{array}\right) \text{ ist in ZSF}
\end{align*}

\begin{align*}
    \left(\begin{array}{ccccc|c}
        1 & 0 & 0 & 0 & 0 & 2 \\
        0 & 0 & 0 & 1 & 0 & 3 \\
        0 & 0 & 0 & 0 & 1 & 1 \\
        0 & 0 & 0 & 0 & 0 & 0
    \end{array}\right) \text{ ist in red. ZSF}
\end{align*}

Wenn eine erweiterte Koeffizientenmatrix in ZSF oder red. ZSF ist (und damit auch das zugehörige LGS), lässt sich die Lösung des LGS leicht ablesen.

Der folgende Algorithmus überführt eine erweiterte Koeffizientenmatrix in ZSF:

\mdf{Satz}
Jede Matrix kann durch endlich viele Umformungen aus Satz 5 in ZSF gebracht werden.

\subsection{Gauß--Jordan--Algorithmus}
\mdf{Algorithmus}

Anmerkung: \texttt{a[x][y]} bezeichnet hier $a_{xy}$.
\begin{verbatim}
1   i = 1
2   j = 1
3   Gauß(i, j):
4       Falls i = m oder j = n+1:
5           Ende
6       Falls a[i][j] = 0:
7           Suche r > i mit a[r][j] != 0
8           Falls r existiert:
9               Tausche Zeilen r und i
10          Sonst:
11              Gauß(i, j+1)
12      Teile i-te Zeile durch a[i][j]
13      Für alle k > i: (Zeile k) - a[k][j] * (Zeile i)
14      Gauß(i+1, j+1)
\end{verbatim}

\mdf{Bemerkung}
Beim Lösen von Hand ist es oft einfacher, notwendige Umformungen durch \glqq{}scharfes Hingucken\grqq{} zu erkennen.

Beim Implementieren des Gauß--Jordan--Algorithmus treten oft Effekte auf, die mit der Repräsentation reeller Zahlen im Rechner zusammenhängen.

\mdf{Beispiel}
\begin{align*}
    \left(\begin{array}{ccccc|c}
        1 & 2 & 1 & 1 & 1 & 2 \\
        -1 & -2 & -2 & 2 & 1 & 1 \\
        2 & 4 & 3 & -1 & 0 & -1 \\
        1 & 2 & 2 & -2 & 1 & 1
    \end{array}\right)
\end{align*}
Löse das zugehörige LGS. Es gibt keine Lösung.
