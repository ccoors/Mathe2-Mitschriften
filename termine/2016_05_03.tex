\termin{03.05.2016}
\subsection{Laplacescher Entwicklungssatz}
\mdf{Definition}
\begr{Laplacescher Entwicklungssatz}

Für $A = (a_{ij}) \in \mathbb{R}^{n \times n}$ und $i, j \in \{1,\dots,n\}$ gilt
\begin{align*}
	\text{det}(A) &= \sum_{j=1}^{n}(-1)^{i+j} \cdot a_{ij} \cdot \text{det}(A_{ij})
\end{align*}
(Entwicklung nach der $i$-ten Zeile)

bzw.
\begin{align*}
	\text{det}(A) &= \sum_{i=1}^{n}(-1)^{i+j} \cdot a_{ij} \cdot \text{det}(A_{ij})
\end{align*}
(Entwicklung nach der $j$-ten Spalte)

Wobei $A_{ij}$ die Matrix ist, die man erhält, wenn man in $A$ die $i$-te Zeile und die $j$-te Zeile streicht.

\mdf{Beispiel}
\begin{align*}
	A &= \begin{pmatrix}
		4 & 2 & -3 & 4 \\
		5 & 6 & 1 & 4 \\
		0 & 0 & 2 & 0 \\
		-2 & -2 &3 & 6
	\end{pmatrix}
\end{align*}
Entwicklung nach der 3. Zeile (da diese viele Nullen enthält)
\begin{align*}
	\text{det}(A) =\,&(-1)^{3+1} \cdot 0 \cdot \text{det}(A_{31}) \\
	+\,&(-1)^{3+2} \cdot 0 \cdot \text{det}(A_{32}) \\
	+\,&(-1)^{3+3} \cdot 2 \cdot \text{det}(A_{33}) \\
	+\,&(-1)^{3+4} \cdot 0 \cdot \text{det}(A_{34}) \\
	=\,&(-1)^{3+3} \cdot 2 \cdot \text{det}\left(\begin{pmatrix}
		4 & 2 & 4 \\
		5 & 6 & 4 \\
		-2 & -2 & 6
	\end{pmatrix}\right) \\
	=\,&2\cdot\left((-1)^{1+1} \cdot 4 \cdot \begin{vmatrix}
		6 & 4 \\
		-2 & 6
	\end{vmatrix} + (-1)^{1+2} \cdot 2 \cdot \begin{vmatrix}
		5 & 4 \\
		-2 & 6
	\end{vmatrix} + (-1)^{1+3} \cdot 4 \cdot \begin{vmatrix}
		5 & 6 \\
		-2 & -2
	\end{vmatrix}\right) \\
	=\,&2 \cdot (4 \cdot 44 - 2 \cdot 38 + 4 \cdot 2) = 2 \cdot 108 \\
	=\,&216
\end{align*}

\subsubsection*{\textbf{Bei $3 \times 3$ Matrizen}}
Bei $3 \times 3$ Matrizen kann man die \begr{Regel von Sarrus} anwenden. Herleitung:
\begin{align*}
    \begin{vmatrix}
        a_{11} & a_{12} & a_{13} \\
        a_{21} & a_{22} & a_{23} \\
        a_{31} & a_{32} & a_{33}
    \end{vmatrix} &= (-1)^{1+1} a_{11} \begin{vmatrix}
        a_{22} & a_{23} \\
        a_{32} & a_{33}
    \end{vmatrix} + (-1)^{1+2} a_{12} \begin{vmatrix}
        a_{21} & a_{23} \\
        a_{31} & a_{33}
    \end{vmatrix} + (-1)^{1+3} a_{13} \begin{vmatrix}
        a_{21} & a_{22} \\
        a_{31} & a_{32}
    \end{vmatrix} \\
    &= a_{11}(a_{22}a_{33} - a_{23}a_{32}) - a_{12}(a_{21}a_{33} - a_{23}a_{31}) + a_{13}(a_{21}a_{32} - a_{22}a_{31}) \\
    &= a_{11}a_{22}a_{33} - a_{11}a_{23}a_{32} - a_{12}a_{21}a_{33} + a_{12}a_{23}a_{31} + a_{13}a_{21}a_{32} - a_{13}a_{22}a_{31}
\end{align*}\pagebreak

Diese Formel kann man sich einfach grafisch veranschaulichen. Man schreibt rechts von der Matrix die ersten beiden Spalten noch einmal auf:
\tikzset{node style ge/.style={circle}}
\begin{center}
\begin{tikzpicture}[baseline=(A.center)]
\tikzset{Umrandung/.style = {opacity=.4,line width=2 mm,line cap=round,color=#1}}
\tikzset{Plus/.style      = {above left=4mm,opacity=1,circle,fill=#1!50}}
\tikzset{Minus/.style     = {below left=4mm,opacity=1,circle,fill=#1!50}}
\matrix (A) [matrix of math nodes, nodes = {node style ge},,column sep=0 mm] 
{ a_{11} & a_{12} & a_{13} & \textcolor{red}{a_{11}} & \textcolor{red}{a_{12}} \\
  a_{21} & a_{22} & a_{23} & \textcolor{red}{a_{21}} & \textcolor{red}{a_{22}} \\
  a_{31} & a_{32} & a_{33} & \textcolor{red}{a_{31}} & \textcolor{red}{a_{32}} \\
};

\draw [Umrandung=blue] (A-1-1.north west) node[Plus=blue] {$+$} to (A-3-3.south east);
\draw [Umrandung=blue] (A-1-2.north west) node[Plus=blue] {$+$} to (A-3-4.south east);
\draw [Umrandung=blue] (A-1-3.north west) node[Plus=blue] {$+$} to (A-3-5.south east);

\draw [Umrandung=red] (A-3-1.south west) node[Minus=red] {$-$} to (A-1-3.north east);
\draw [Umrandung=red] (A-3-2.south west) node[Minus=red] {$-$} to (A-1-4.north east);
\draw [Umrandung=red] (A-3-3.south west) node[Minus=red] {$-$} to (A-1-5.north east);
\end{tikzpicture}\footnote{Quelle: Alain Matthes, \url{http://www.texample.net/tikz/examples/mnemonic-rule-for-matrix-determinant/}, angepasst.}
\end{center}
Für jede blaue Diagonale bildet man das Produkt der Elemente und summiert dann die Diagonalen auf. Das tut man auch für die roten Diagonalen und subtrahiert diese dann von der Summe der blauen Diagonalen. Das Ergebnis ist die Determinante.

\mdf{Satz}
Sind $A, B \in \mathbb{R}^{n \times n}$ und $a_1,\dots,a_n$ die Zeilen von A, sowie $k \in \mathbb{R}$. Dann gelten:
\begin{itemize}
	\item{$\text{det}(A\cdot B) = \text{det}(A) \cdot \text{det}(B)$}
	\item{$\text{det}\begin{pmatrix}a_1\\\vdots\\a_j\cdot k\\\vdots\\a_n\end{pmatrix} = k \cdot \text{det}(A)$}
	\item{$\text{det}(A^T) = \text{det}(A)$}
	\item{Ist $A$ invertierbar, so gilt $\text{det}(A^{-1}) = \frac{1}{\text{det}(A)}$}
	\item{Vertauscht man in $A$ zwei Zeilen, so wechselt das Vorzeichen der Determinante.}
	\item{\begin{align*}
		&\text{det}\left(\begin{pmatrix}
			a_{11} & \dots & a_{1n} \\
			\vdots & \ddots & \vdots \\
			a_{j1}+b_1 & \dots & a_{jn} + b_n \\
			\vdots & \ddots & \vdots \\
			a_{n1} & \dots & a_{nn}
		\end{pmatrix}\right) =\\
		&\text{det}\left(\begin{pmatrix}
			a_{11} & \dots & a_{1n} \\
			\vdots & \ddots & \vdots \\
			a_{j1} & \dots & a_{jn} \\
			\vdots & \ddots & \vdots \\
			a_{n1} & \dots & a_{nn}
		\end{pmatrix}\right) + \text{det}\left(\begin{pmatrix}
			a_{11} & \dots & a_{1n} \\
			\vdots & \ddots & \vdots \\
			a_{(j-1)1} & \dots & a_{(j-1)n} \\
			b_1 & \dots & b_n \\
			a_{(j+1)1} & \dots & a_{(j+1)n} \\
			\vdots & \ddots & \vdots \\
			a_{n1} & \dots & a_{nn}
		\end{pmatrix}\right)
	\end{align*}}
	\item{$A$ ist genau dann invertierbar, wenn $\text{det}(A) \neq 0$.}
	\item{Sind die Zeilen von $A$ linear abhängig, so ist $\text{det}(A) = 0$.}
\end{itemize}

\newpage
\kapitel{Lineare Abbildungen}
\mdf{Definition}
Seien $U \subseteq \mathbb{R}^n, V \subseteq \mathbb{R}^m$ Unterräume. Die Abbildung $\varphi : U \rightarrow V$ heißt \begr[Lineare Abbildung]{linear}, falls gelten:
\begin{description}
	\item[(A1)]{$\forall u, u' \in U$ gilt $\varphi(u+u') = \varphi(u) + \varphi(u')$}
	\item[(A2)]{$\forall u \in U, \lambda \in \mathbb{R}$ gilt $\varphi(\lambda u) = \lambda \cdot \varphi(u)$}
\end{description}

\mdf{Beispiel}
\begin{align*}
	\varphi :\,\,&\mathbb{R}^2 \rightarrow \mathbb{R}^2 \\
	\begin{pmatrix}x\\y\end{pmatrix} &\mapsto \begin{pmatrix}2x+y\\3x\end{pmatrix}
\end{align*}
ist eine lineare Abbildung. Beweis:
\begin{description}
	\item[(A1)]{Zu zeigen:
	\begin{align*}
		\varphi\left(\begin{pmatrix}x_1\\y_1\end{pmatrix} + \begin{pmatrix}x_2\\y_2\end{pmatrix}\right) &= \varphi\left(\begin{pmatrix}x_1\\y_1\end{pmatrix}\right) + \varphi\left(\begin{pmatrix}x_2\\y_2\end{pmatrix}\right) \\[0.5cm]
		\varphi\left(\begin{pmatrix}x_1+x_2\\y_1+y_2\end{pmatrix}\right) &= \begin{pmatrix}2 (x_1+x_2) + (y_1+y_2)\\3(x_1+x_2)\end{pmatrix} = \begin{pmatrix}2 x_1 + y_1 + 2 x_2 + y_2\\3 x_1 + 3 x_2\end{pmatrix} \\
		&= \begin{pmatrix}2 x_1 + y_1 \\ 3 x_1\end{pmatrix} + \begin{pmatrix}2 x_2 + y_2 \\ 3 x_2\end{pmatrix} = \varphi\left(\begin{pmatrix}x_1\\y_1\end{pmatrix}\right) + \varphi\left(\begin{pmatrix}x_2\\y_2\end{pmatrix}\right)
	\end{align*}
	}
	\item[(A2)]{Zu zeigen:
	\begin{align*}
		\varphi\left(\lambda\begin{pmatrix}x\\y\end{pmatrix}\right) &= \lambda \varphi\left(\begin{pmatrix}x\\y\end{pmatrix}\right) \\[0.5cm]
		\varphi\left(\lambda\begin{pmatrix}x\\y\end{pmatrix}\right) = \varphi\left(\begin{pmatrix}\lambda x\\\lambda y\end{pmatrix}\right) &= \begin{pmatrix}2\lambda x + \lambda y\\3\lambda x\end{pmatrix} = \begin{pmatrix}\lambda(2x+y)\\\lambda(3x)\end{pmatrix} = \lambda \varphi\left(\begin{pmatrix}x\\y\end{pmatrix}\right)
	\end{align*}
	}
\end{description}

\textbf{Gegenbeispiel}
\begin{align*}
	\psi :\,\,&\mathbb{R}^2 \rightarrow \mathbb{R}^2 \\
	\begin{pmatrix}x\\y\end{pmatrix} &\mapsto \begin{pmatrix}x+3\\2x\end{pmatrix}
\end{align*}
ist \textbf{nicht} linear. Denn z.B. $\psi\left(2\begin{pmatrix}1\\0\end{pmatrix}\right) = \psi\left(\begin{pmatrix}2\\0\end{pmatrix}\right) = \begin{pmatrix}5\\4\end{pmatrix}$, aber $2\psi\left(\begin{pmatrix}1\\0\end{pmatrix}\right) = 2 \cdot \begin{pmatrix}4\\2\end{pmatrix} = \begin{pmatrix}8\\4\end{pmatrix} \neq \begin{pmatrix}5\\4\end{pmatrix}$.

\mdf{Satz}
Eine Abbildung $\varphi : \mathbb{R}^n \rightarrow \mathbb{R}^m$ ist genau dann linear, wenn sie in der Form $\varphi(x) = Ax$ mit $A \in \mathbb{R}^{m \times n}$ geschrieben werden kann.

D.h.
\begin{align*}
	\varphi\left(\begin{pmatrix}x_1\\\vdots\\x_n\end{pmatrix}\right) &= \begin{pmatrix}
		a_{11}x_1 + \dots + a_{1n}x_n \\
		\vdots & \ddots & \vdots \\
		a_{m1}x_1 + \dots & a_{mn}x_n
	\end{pmatrix}
\end{align*}
Die Matrix $A$ ist eindeutig bestimmt. Die Spalten von $A$ sind die Bilder der Standardbasisvektoren $e_1,\dots,e_n$. D.h.
\begin{align*}
	A &= (\varphi(e_1),\varphi(e_2),\dots,\varphi(e_n))
\end{align*}

\mdf{Beweis}
\begin{align*}
	\varphi(x + y) &= A(x + y) \\
	&= Ax + Ay \\
	&= \varphi(x) + \varphi(y) \\
	\varphi(\lambda x) &= A(\lambda x) = \lambda(Ax) = \lambda\varphi(x)
\end{align*}
Außerdem:
\begin{align*}
	x &= x_1e_1 + x_2e_2 + \dots + x_ne_n
\end{align*}
Damit:
\begin{align*}
	\varphi(x) &= \varphi(x_1e_1 + \dots + x_ne_n) \\
	&= \varphi(x_1e_1) + \varphi(x_2e_2) + \dots + \varphi(x_ne_n) \\
	&= x_1\varphi(e_1) + x_2\varphi(e_2) + \dots + x_n\varphi(e_n) \\
	&= (\varphi(e_1),\varphi(e_2),\dots,\varphi(e_n))x \\
	&= Ax
\end{align*}

\mdf{Definition}
Die Matrix $A$ aus Satz 3 heißt \begr[Darstellende Matrix]{darstellende Matrix} der linearen Abbildung $\varphi$.

\mdf{Beispiel}
Betrachte Abbildung $\varphi$ aus Beispiel 2. Wie sieht die darstellende Matrix aus?

Standardbasisvektoren einsetzen:
\begin{align*}
	\varphi(e_1) = \varphi\left(\begin{pmatrix}1\\0\end{pmatrix}\right) &= \begin{pmatrix}2\\3\end{pmatrix} \\
	\varphi(e_2) = \varphi\left(\begin{pmatrix}0\\1\end{pmatrix}\right) &= \begin{pmatrix}1\\0\end{pmatrix}
\end{align*}

Also ist $A = \begin{pmatrix}2 & 1 \\ 3 & 0\end{pmatrix}$ die darstellende Matrix von $\varphi$.
