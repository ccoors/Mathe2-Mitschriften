\termin{05.04.2016}
\kapitel{Zahlen}
\subsection{Zahlenmengen}
\begin{itemize}
\item{Die Menge $\mathbb{N} = \{0,1,2,3,...\}$ heißt \begr[Natürliche Zahlen]{Menge der natürlichen Zahlen}. Für uns beinhalten die natürlichen Zahlen die $0$.}
\item{Die Menge $\mathbb{Z} = \{...,-3,-2,-1,0,1,2,3,...\}$ heißt \begr[Ganze Zahlen]{Menge der ganzen Zahlen}.}
\item{Die Menge $\mathbb{Q} = \left\{\frac{p}{q}\,|\,p,q \in \mathbb{Z}, q \neq 0\right\}$ heißt \begr[Rationale Zahlen]{Menge der rationalen Zahlen}.}
\end{itemize}

\subsection{Satz von Euklid}
Es gibt keine rationale Zahl $q \in \mathbb{Q}$ mit $q^2 = 2$.

\subsection{Reelle Zahlen}
Die Menge $\mathbb{R}$ der \begr[Reelle Zahlen]{reellen Zahlen} ist die Vereinigung der Menge der rationalen Zahlen mit allen Zahlen, die sich durch rationalen Zahlen beliebig approximieren lassen.

\subsection{Wurzel}
Ist $b^n = a$ für $a, b \in \mathbb{R} > 0, n \in \mathbb{N}$, so heißt b die \begr[Wurzel]{$n$-te Wurzel} von $a$: $b = \sqrt[n]{a}=a^\frac{1}{n}$. $b$ ist für alle $a > 0$ eindeutig. Der Vorgang des Wurzelziehens heißt auch \begr[Radizieren]{radizieren}.

\subsection{Kurzschreibweisen}
Wir führen Kurzschreibweisen ein:
\begin{align*}
[a,b] &:= \{x\in\mathbb{R}\,|\,a \leq x \leq b\}\text{ heißt \begr[Intervalle]{abgeschlossenes Intervall},} \\
[a,b) &:= \{x\in\mathbb{R}\,|\,a \leq x < b\}\text{ und }(a,b] := \{x\in\mathbb{R}\,|\,a < x \leq b\}\text{ heißen halboffene Intervalle und} \\
(a,b) &:= \{x\in\mathbb{R}\,|\,a < x < b\}\text{ heißt offenes Intervall.} \\
&\text{Sonderfälle:} \\
[a, \infty) &:= \{x\in\mathbb{R}\,|\,a \leq x\} \\
(-\infty, b] &:= \{x\in\mathbb{R}\,|\,x \leq b\} \\
(a, \infty) &:= \{x\in\mathbb{R}\,|\,a < x\} \\
(-\infty, b) &:= \{x\in\mathbb{R}\,|\,x < b\}
\end{align*}

\subsection{Beschränktheit nach oben}
Eine Menge $M \subseteq \mathbb{R}$ heißt nach oben \begr[Beschränktheit]{beschränkt}, falls es ein $k \in \mathbb{R}$ gibt, so dass $x \le k$ für alle $x \in M$. $k$ heißt obere Schranke von $M$. Ist $M$ nach oben beschränkt, gibt es mehrere obere Schranken von $M$: $M = [3,5)$. Offenbar ist $k = 17$ eine obere Schranke von $M$. Die kleinste obere Schranke von $M$ heißt \begr{Supremum} von $M$, kurz, $\text{sup}(M)$.

\subsection{Satz über die Vollständigkeit von $\mathbb{R}$}
Jede nach oben beschränkte Menge $M \subseteq \mathbb{R}$ besitzt ein Supremum in $\mathbb{R}$. Für $\mathbb{Q}$ gilt dies nicht. Das Supremum der Menge $[0,\sqrt{3}) \subseteq \mathbb{Q}$ wäre in $\mathbb{R}$ $\sqrt{3}$, diese Zahl liegt aber nicht in $\mathbb{Q}$. Dies unterscheidet die reellen Zahlen fundamental von den rationalen Zahlen.

\subsection{Beschränktheit nach unten}
Siehe Beschränktheit nach oben. Wichtig: Die größte untere Schranke heißt \begr{Infimum}, kurz $\text{inf}(M)$. Es gilt: $\text{inf}(M) = -\text{sup}(-M)$. ($-M = \{-m\,|\,m \in M\}$).

\subsection{Satz zu Supremum und Infimum} Supremum und Infimum einer Menge M müssen nicht in der Menge liegen. Ist $\text{sup}(M) \in M$, so ist es auch das größte Element (Maximum) von $M$, kurz: $\text{max}(M)$. Analog dazu: Ist $\text{inf}(M) \in M$, so ist es auch das kleinste Element (Minimum) von $M$, kurz: $\text{min}(M)$.

\subsection{Literatur}
Alles bis hierher ist zu finden in:

\href{http://www.mat.univie.ac.at/~gerald/ftp/book-mfi/mfi1.pdf}{Mathematik für Informatiker -- Band 1: Diskrete Mathematik und Lineare Algebra. Springer. Gerald Teschl, Susanne Teschl. Kapitel 2.1.}

