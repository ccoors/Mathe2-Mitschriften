\termin{29.04.2016}
\subsection{Invertieren}
\mdf{Definition}
Ist $A \in \mathbb{R}^{n \times n}$ eine quadratische Matrix und gibt es zu $A$ eine Matrix $A^{-1}$ mit
\begin{align*}
	A^{-1} A = A A^{-1} = \mathbbm{1}_n
\end{align*}
so heißt $A$ \begr[Invertierbarkeit (Matrix)]{invertierbar} und die Matrix $A^{-1}$ heißt \begr[Inverse Matrix]{inverse Matrix} zu $A$.

\vspace{1cm}

Ist ein LGS mit $n$ Gleichungen und $n$ Unbekannten gegeben
\begin{align*}
	Ax = b\text{ mit }A \in \mathbb{R}^{n \times n},\quad x, b \in \mathbb{R}^n
\end{align*}
und A ist invertierbar, dann ist $x = A^{-1}b$ eindeutige Lösung des LGS.
\begin{align*}
	A^{-1}Ax = \mathbbm{1}_nx = x = A^{-1}b
\end{align*}

\mdf{Satz}
Sind $A, B \in \mathbb{R}^{n \times n}$ invertierbar, so ist auch $A \cdot B$ invertierbar und es gilt
\begin{align*}
	(AB)^{-1} &= B^{-1}A^{-1} \\
	\text{Denn: } (B^{-1}A^{-1})(AB) &= B^{-1}(A^{-1}A)B \\
	&= B^{-1} (\mathbbm{1}_n) B \\
	&= B^{-1}B \\
	&= \mathbbm{1}_n
\end{align*}

\mdf{Beispiel}
Ist $A = \begin{pmatrix}2&4\\-1&3\end{pmatrix}$ invertierbar? Gesucht ist also eine Matrix $B = \begin{pmatrix}b_{11}&b_{12}\\b_{21}&b_{22}\end{pmatrix}$ mit
\begin{align*}
	AB = \mathbbm{1}_2 &= \begin{pmatrix}1&0\\0&1\end{pmatrix} \\
	\begin{pmatrix}2&4\\-1&3\end{pmatrix}\begin{pmatrix}b_{11}&b_{12}\\b_{21}&b_{22}\end{pmatrix} &= \begin{pmatrix}1&0\\0&1\end{pmatrix} \\
	AB = \begin{pmatrix}2b_{11}+4b_{21}&2b_{12}+4b_{22}\\-b_{11}+3b_{21}&-b_{12}+3b_{22}\end{pmatrix} &= \begin{pmatrix}1&0\\0&1\end{pmatrix}
\end{align*}
Als LGS mit $4$ Gleichungen und $4$ Unbekannten
\begin{alignat*}{9}
2b_{11} & \enspace\enspace & & \enspace+\enspace & 4b_{21} & & & \enspace=\enspace & 1 \\
-b_{11} & & & \enspace+\enspace & 3b_{21} & & & \enspace=\enspace & 0 \\
 & \enspace\enspace & 2b_{12} & & & \enspace+\enspace & 4b_{22} & \enspace=\enspace & 0 \\
 & \enspace\enspace & -b_{12} & & & \enspace+\enspace & 3b_{22} & \enspace=\enspace & 1
\end{alignat*}
\begin{align*}
	\begin{pmatrix}
		2 & 0 & 4 & 0 \\
		-1 & 0 & 3 & 0 \\
		0 & 2 & 0 & 4 \\
		0 & -1 & 0 & 3
	\end{pmatrix} \begin{pmatrix}
		b_{11} \\
		b_{12} \\
		b_{21} \\
		b_{22}
	\end{pmatrix} &= \begin{pmatrix}
		1 \\
		0 \\
		0 \\
		1
	\end{pmatrix}
\end{align*}
Als $2$ LGS mit je $2$ Unbekannten
\begin{align*}
	\begin{pmatrix}2&4\\-1&3\end{pmatrix}\begin{pmatrix}b_{11}\\b_{21}\end{pmatrix} = \begin{pmatrix}1\\0\end{pmatrix}\text{ und }
	\begin{pmatrix}2&4\\-1&3\end{pmatrix}\begin{pmatrix}b_{12}\\b_{22}\end{pmatrix} = \begin{pmatrix}0\\1\end{pmatrix}
\end{align*}
Als erweiterte erweiterte (!) Koeffizientenmatrix
\begin{align*}
	\left(\begin{array}{cc|c|c}
		2 & 4 & 1 & 0 \\
		-1 & 3 & 0 & 1
	\end{array}\right) = \left(\begin{array}{c|c}A & \mathbbm{1}_2\end{array}\right)
\end{align*}
Mit Gauß--Jordan auf reduzierte ZSF bringen
\begin{align*}
	\left(\begin{array}{c|c}
		\begin{matrix}1 & 0\\0 & 1\end{matrix} & A^{-1}
	\end{array}\right) &= \left(\begin{array}{c|c}
		\mathbbm{1}_2 & A^{-1}
	\end{array}\right) \\
	\left(\begin{array}{cc|cc}
		2 & 4 & 1 & 0 \\
		-1 & 3 & 0 & 1
	\end{array}\right)\enspace &|\text{I} : 2 \\
	\left(\begin{array}{cc|cc}
		1 & 2 & \frac{1}{2} & 0 \\[0.3em]
		-1 & 3 & 0 & 1
	\end{array}\right)\enspace &|\text{II} + \text{I} \\
	\left(\begin{array}{cc|cc}
		1 & 2 & \frac{1}{2} & 0 \\[0.3em]
		0 & 5 & \frac{1}{2} & 1
	\end{array}\right)\enspace &|\text{II} : 5 \\
	\left(\begin{array}{cc|cc}
		1 & 2 & \frac{1}{2} & 0 \\[0.3em]
		0 & 1 & \frac{1}{10} & \frac{1}{5}
	\end{array}\right)\enspace &|\text{I} - 2\cdot\text{II} \\
	\left(\begin{array}{cc|cc}
		1 & 0 & \frac{3}{10} & -\frac{2}{5} \\[0.3em]
		0 & 1 & \frac{1}{10} & \frac{1}{5}
	\end{array}\right)
\end{align*}
Also $A^{-1} = \frac{1}{10}\begin{pmatrix}3 & -4 \\1 & 2\end{pmatrix}$.
\begin{align*}
	A^{-1}A = \frac{1}{10}\begin{pmatrix}3 & -4 \\1 & 2\end{pmatrix}\begin{pmatrix}2&4\\-1&3\end{pmatrix} = \frac{1}{10}\begin{pmatrix}10&0\\0&10\end{pmatrix} = \begin{pmatrix}1&0\\0&1\end{pmatrix}
\end{align*}

Also: Matrix invertieren durch simultanes Lösung von $n$ LGS. $\left(A|\mathbbm{1}_n\right)$ mit Gauß--Jordan--Algorithmus umformen zu $\left(\mathbbm{1}_n|A^{-1}\right)$. Ist dies möglich, so ist $A$ invertierbar.

\vspace{0.5cm}

\textbf{Frage:} Können wir einer Matrix \glqq{}ansehen\grqq{}, ob sie invertierbar ist?

Betrachte $2 \times 2$ LGS:
\begin{align*}
	a_{11}x_1 + a_{12}x_2 &= b_1\quad |\cdot a_{22} \\
	a_{21}x_1 + a_{22}x_2 &= b_2\quad |\cdot (-a_{12}) \\[0.5cm]
	a_{11}a_{22}x_1 + a_{12}a_{22}x_2 &= a_{22}b_1 \\
	-a_{12}a_{21}x_1 - a_{12}a_{22}x_2 &= -a_{12}b_2 \quad |\text{I} + \text{II} \\[0.5cm]
	a_{11}a_{22}x_1 + a_{12}a_{22}x_2 - a_{12}a_{22}x_2 - a_{12}a_{21}x_1 &= a_{22}b_1 - a_{12}b_2 \\
	\Leftrightarrow (a_{11}a_{22}-a_{12}a_{21}) x_1 &= a_{22}b_1 - a_{12}b_{2}
\end{align*}
Diese Gleichung enthält nur noch $x_1$. Analog erhält man
\begin{align*}
	(a_{11}a_{22}-a_{12}a_{21}) x_2 &= a_{11}b_1 - a_{21}b_{2}
\end{align*}

Ist $(a_{11}a_{22}-a_{12}a_{21}) \neq 0$, ist das LGS eindeutig lösbar.

\mdf{Definition}
Für eine $2 \times 2$ Matrix
\begin{align*}
    A &= \begin{pmatrix}a_{11} & a_{12} \\ a_{21} & a_{22}\end{pmatrix}
\end{align*}
heißt
\begin{align*}
    \text{det}(A) = \begin{vmatrix}a_{11} & a_{12} \\ a_{21} & a_{22}\end{vmatrix}
\end{align*}
\begr{Determinante} von $A$. Die Verallgemeinerung auf allgemeine Matrizen ist schwieriger. Die Herleitung geht über diese Vorlesung hinaus.
